\chapter{Computing with Register Machines}

\section{Designing Register Machines}

\begin{exe}[5.1]
    \label{5.1}
    The data-path and the controller diagrams for the iterative factorial 
    machine are shown on figure \ref{5.1fig}.

    \begin{figure}
        \centering
        \begin{tikzpicture}[>=Stealth]
            % Data-path diagram.
            \matrix[data matrix] (dp) {
                & \node[reg] (n) {n}; &
                \node[test] (>) {>}; \\

                & \node[const] (c1) {1}; & \\

                \node[reg] (p) {product}; &&
                \node[reg] (c) {counter}; \\

                & \node[op] (+) {+}; & \\

                & \node[op] (*) {*}; & \\
            };

            \draw[arg] (n) -- (>);
            \draw[arg] (c) -- (>);
            \draw[button=0.82] (c1) -| node[near start, above] {p<-1} (p);
            \draw[button=0.82] (c1) -| node[near start, above] {c<-1}
                ($ (c.north) + (-1em, 0) $);
            \draw[arg] (c1) -- (+);
            \draw[button=0.8] ($ (+.north) + (1em, 0) $) |-
                node[near end, above] {c++} (c);
            \draw[arg] (c) |- (+);
            \draw[arg] ($ (p.south) + (-1em, 0) $) |-
                ($ (*.south) + (0, -1.5em) $) -- (*);
            \draw[arg] (c) |- (*);
            \draw[button=0.8] (*) -| node[near end, right]{p<-*}
                ($ (p.south) + (1em, 0) $);

            % Controller diagram
            \matrix[controller matrix, right=5em of dp] {
                \node[cio] (s) {start}; \\
                \node[cbutton] (pi) {p<-1}; \\
                \node[cbutton] (ci) {c<-1}; \\
                \node[ctest] (ct) {>}; & \node[cio] (cd) {done}; \\[+1em]
                \node[cbutton] (cp) {p<-*}; \\
                \node[cbutton] (cc) {c++}; \\
            };

            \draw[flow] (s) -- (pi);
            \draw[flow] (pi) -- (ci);
            \draw[flow] (ci) -- (ct);
            \draw[flow] (ct) --node[right] {no} (cp);
            \draw[flow] (ct) --node[above] {yes} (cd);
            \draw[flow] (cp) -- (cc);
            \draw[flow] (cc) -| ($ (ct.west) - (2.5em, 0) $) -- (ct);
        \end{tikzpicture}
        \caption{The data-path and controller diagrams for the iterative 
        factorial machine.}
        \label{5.1fig}
    \end{figure}
\end{exe}

\subsection{A Language for Describing Register Machines}

\begin{exe}[5.2]
    Anticipating on the next section to use the register-machine simulator, we 
    can define the iterative factorial machine of exercise \ref{5.1} as:
    \scm{ch5/5.02.scm}
\end{exe}
