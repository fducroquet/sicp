\documentclass{scrreprt}

% Language-specific settings
\usepackage{polyglossia}
\setdefaultlanguage{american}

% Fonts
\usepackage{fontspec,microtype}
\setmainfont{Linux Libertine O}
\setsansfont{Linux Libertine O}
\setmonofont[Scale=MatchLowercase]{FantasqueSansMono-Regular.ttf}

% Hyperlinks
\usepackage{xcolor}
\definecolor{darkblue}{rgb}{0,0,0.5}
\definecolor{darkgreen}{rgb}{0,0.3,0}
\usepackage[unicode,
pdftitle={Solutions to Structure and Interpretation of Computer Programs},
pdfauthor={Fabienne Ducroquet},
urlcolor=darkblue,
linkcolor=darkgreen,
breaklinks,
colorlinks]{hyperref}

% Headers
\usepackage{fancyhdr}
\pagestyle{fancy}
\fancyhead[RO,LE]{\thepage}
\fancyhead[LO]{\rightmark}
\fancyhead[RE]{\leftmark}

% Source code highlighting
\usepackage{minted}
\newminted[cscm]{scheme}{autogobble,style=lovelace}
\newmintedfile[scm]{scheme}{style=lovelace}
\newmintinline[vscm]{scheme}{style=bw}

% Show subsubsections in the table of contents.
\setcounter{tocdepth}{4}

% Theorem environments
\usepackage{amssymb,amsmath,amsthm}
\newtheoremstyle{break}{}{}{}{-12pt}{\bfseries}{}{\newline}{}
\theoremstyle{break}
\newtheorem{exeaux}{Exercise}[chapter]

% For exercise 2.24, and some exercises in chapter 3
\usepackage{tikz}
\usetikzlibrary{positioning, arrows.meta, calc, fit, matrix, backgrounds, 
decorations.markings, shapes.misc, shapes.geometric}
\newcommand\var[2]{%
    \newlength{#1}
    \setlength{#1}{#2}
}

% Definitions for the box-and-pointer diagrams.
\var{\boxsize}{2em}
\var{\round}{3pt}

\tikzset{
    box/.style={
        minimum size = \boxsize,
        rectangle,
        rounded corners=\round,
        draw,
        fill=brown!15,
    },
    struct name/.style={
        minimum height=\boxsize,
    },
    car/.style={
        minimum size = \boxsize,
        append after command={
        \pgfextra
        \draw[sharp corners,fill=brown!15]
            (\tikzlastnode.south)
            -| (\tikzlastnode.east)
            |- (\tikzlastnode.north)
            [rounded corners=\round]
            -| (\tikzlastnode.west)
            |- (\tikzlastnode.south east);
        \endpgfextra},
    },
    cdr/.style={
        minimum size = \boxsize,
        append after command={
        \pgfextra
            \draw[sharp corners,fill=brown!15] (\tikzlastnode.south)
                -| (\tikzlastnode.west) |- (\tikzlastnode.north)
                [rounded corners=\round] -| (\tikzlastnode.east)
                |- (\tikzlastnode.south west);
        \endpgfextra},
    },
    cell matrix/.style={
        row sep = 1.2\boxsize,
        column sep = {\boxsize,between origins}
    },
    box and pointer/.style={
        very thick,
        font = \ttfamily,
        decoration = {markings,
            mark = at position 0 with { \fill circle (3pt); }
        },
    },
    pointer/.style = {
        -Stealth
    },
    box pointer/.style = {
        pointer,
        postaction = decorate
    },
    nil/.style = {
        Triangle Cap[]-,
        shorten >=.8pt,
    }
}

\newcommand\nil[1]{
    \draw[nil] (#1.south west) -- (#1.north east);
}
\newcommand\link[2]{
    \draw[box pointer] (#1.base) -- (#2);
}
% Box-and-pointer end.

% Tikz styles and commands for the environments of section 3.2.
\var{\nametoenv}{4mm}
\tikzset{
    env/.style={
        rectangle,
        rounded corners=2pt,
        % border
        very thick,
        draw=teal!80!black,
        % filling
        fill=teal!10,
        inner sep = 2mm,
        font=\ttfamily\small,
        align=left,
    },
    global env/.style={
        env,
        align=left,
        inner xsep=1ex,
        text width=13cm-2*\pgfkeysvalueof{/pgf/inner xsep},
        minimum width=13cm,
        minimum height=1.5cm
    },
    code/.style={
        align=left,
        font=\ttfamily\small,
    }
}

\var{\circleradius}{3mm}
\var{\smallcircleradius}{.8mm}
\newcommand\pointer[1]{
    \draw (#1) circle (\circleradius);
    \filldraw (#1) circle (\smallcircleradius);
}
\newcommand\procedure[3]{
    \coordinate[left=\circleradius of #1]  (#2);
    \coordinate[right=\circleradius of #1]  (#3);
    \pointer{#2}
    \pointer{#3}
}
% Environments end.

% Settings for timing diagrams in section 3.4.
\tikzset{
    bank/.style={
        circle,
        draw,
        minimum size=1.2cm,
        fill=orange!10,
    },
    action/.style={
        rounded rectangle,
        draw,
        minimum height=1.5em,
        fill=blue!5,
    },
    time matrix/.style={
        column sep=1em,
    },
    arrow/.style={
        rounded corners=10pt,
        thick,
        ->,
    },
}

% Settings for the data-path and controller diagrams in section 5.1.
\var{\regheight}{1cm}
\var{\push}{1.5ex}
\tikzset{
    % Node matrix
    data matrix/.style={
        column sep=1.5em,
        row sep=2.5em,
        nodes = { anchor = center, align = center },
    },
    controller matrix/.style={
        data matrix,
        column sep = 3em,
    },
    % Nodes
    data/.style={
        draw,
        thick,
        rounded corners=2pt,
        minimum height=\regheight,
    },
    reg/.style={
        data,
        rectangle,
        inner xsep=.5cm,
        fill=gray!12!blue!5,
    },
    const/.style={
        data,
        regular polygon,
        regular polygon sides=3,
        fill=gray!12!green!5,
        minimum height=1.7\regheight,
    },
    op/.style={
        data,
        trapezium,
        trapezium left angle=120,
        trapezium right angle=120,
        fill=pink!22!yellow!20
    },
    test/.style={
        data,
        circle,
        fill=gray!12!pink!15,
    },
    % Controller diagrams
    cio/.style={
        thick,
        rounded corners=2pt,
    },
    ctest/.style={
        cio,
        draw,
        inner xsep=.4cm,
        diamond,
        fill=gray!12!pink!15,
    },
    cbutton/.style={
        cio,
        draw,
        inner xsep=.5cm,
        rectangle,
        fill=gray!12!blue!5,
        minimum height=\regheight,
    },
    % Paths
    path/.style={
        thick,
        rounded corners=3pt,
    },
    button/.style={
        path,
        decoration = {markings,
            mark = at position #1 with {
                \fill[white] circle (.7071\push);
                \draw[-] ++(-.5\push, -.5\push) -- ++(\push, \push);
                \draw[-] ++(-.5\push, .5\push) -- ++(\push, -\push);
                \clip[draw] circle (.7071\push); }
        },
        postaction = decorate,
        ->,
        font=\ttfamily,
    },
    arg/.style={
        path,
        ->,
    },
    flow/.style={
        arg,
    },
}

% Include the exercises in the table of contents.
\NewDocumentEnvironment{exe}{o}
{\IfNoValueTF{#1}
    {\exeaux\addcontentsline{toc}{paragraph}%
        {Exercise}}
    {\exeaux\addcontentsline{toc}{paragraph}%
        {Exercise #1}}%
        \ignorespaces}
{\endexeaux}

\newtheoremstyle{example}{}{}{}{}{\itshape}{:}{ }{}
\theoremstyle{example}
\newtheorem*{example}{Example}

\newtheoremstyle{remark}{}{}{}{}{\bfseries}{.}{ }{}
\theoremstyle{remark}
\newtheorem*{remark}{Remark}

\newtheoremstyle{comp}{}{}{\small}{}{\itshape}{:}{\newline}{}
\theoremstyle{comp}
\newtheorem*{comp}{Complement}

% For inline fractions
\usepackage{xfrac}

\title{Solutions to Structure and Interpretation of Computer Programs}
\author{Fabienne \textsc{Ducroquet}}

\begin{document}

\maketitle

\tableofcontents

\chapter*{Introduction}

Solutions to most of the exercises of \emph{Structure and Interpretation of 
Computer Programs}, second edition, by Harold \textsc{Abelson} and Gerald Jay 
\textsc{Sussman} with Julie \textsc{Sussman}.

The answers to these exercises have been tested with the Scheme interpreter from 
\href{http://gambitscheme.org}{Gambit Scheme}, at the exception of the exercises 
from section \ref{2.2.4} about the picture language, which have been written 
with \href{http://racket-lang.org}{Racket}, using the \vscm{graphics.ss} 
library.

\chapter{Building Abstractions with Procedures}

\section{The Elements of Programming}

\subsection{Expressions}

This subsection contains no exercises.

\subsection{Naming and the Environment}

This subsection contains no exercises.

\subsection{Evaluationg Combinations}

This subsection contains no exercises.

\subsection{Compound Procedures}

This subsection contains no exercises.

\subsection{The Substitution Model for Procedure Application}

This subsection contains no exercises.

\subsection{Conditional Expressions and Predicates}

\begin{exe}[1.1]
    Check with a scheme interpreter.
\end{exe}

\begin{exe}[1.2]
    A minimally indented version:
    \scm{ch1/1.02a.scm}
    A heavily indented version:
    \scm{ch1/1.02b.scm}
\end{exe}

\begin{exe}[1.3]
    \ \vspace{-20pt}
    \scm{ch1/1.03.scm}
\end{exe}

\begin{exe}[1.4]
    If $b > 0$, \vscm{(a-plus-abs-b a b)} returns $a + b$; otherwise it returns 
    $a - b$. In other words, \vscm{(a-plus-abs-b a b)} returns $a + |b|$.
\end{exe}

\begin{exe}[1.5]
    With applicative-order evaluation, the interpreter tries to evaluate 
    \vscm{(p)}, which results in an infinite loop, so the interpreter never 
    returns (or returns an error).

    With normal-order evaluation, the interpreter doesn't try to evaluate 
    \vscm{(p)} until it's really needed, but that never happens since
    \vscm{(= x 0)} returns true, so the call returns 0.
\end{exe}

\subsection{Example: Square Roots by Newton’s Method}
\label{1.1.7}

\begin{exe}[1.6]
    When Alyssa attempts to use this to compute square roots, the program never 
    returns.

    Explication: \vscm{new-if} is an ordinary procedure, so each time it is 
    called, the evaluator tries to evaluate all of its arguments. In particular, 
    each call to \vscm{sqrt-iter} will cause one more call to \vscm{sqrt-iter}, 
    whether \vscm{(good-enough? guess x)} returns true or not, so the evaluator 
    ends up in an infinite loop.
\end{exe}

\begin{exe}[1.7]
    \label{1.7}
    A possible solution:
    \scm{ch1/1.07.scm}

    Let's call $x$ the number whose root we want to compute.

    With the initial \vscm{good-enough?} test:
    \begin{itemize}
        \item If $x$ is very small, the difference between the guess and $x$ 
        becomes smaller than $0.001$ (or any number we would replace $0.001$ 
        with, for small enough numbers) while the guess is still several times 
        larger than $\sqrt{x}$, or even orders of magnitude away from it.
        \begin{example}
            \vscm{(sqrt 0.0001)} returns $0.03230844833048122$ instead of $0.01$ 
            because\linebreak
            \vscm{(abs (- (square 0.03230844833048122) 0.0001))} 
            returns\linebreak
            $9.438358335233747e-4$.
        \end{example}
    \item If $x$ is very large, the difference between the guess and $x$ will 
        always be found to be larger than $0.001$ (or any number we would 
        replace $0.001$ with, for large enough numbers) because
        ($x - \text{any number}$) can not be expressed to the precision required 
        to compare it to $0.001$, so the call never returns.
        \begin{example}
            \vscm{(sqrt 1e+129)} does not return, while \vscm{(sqrt 1e+128)} 
            returns a correct answer almost instantly\footnote{These values are 
            implementation-dependent.}.
        \end{example}
    \end{itemize}

    With the modified versions of \vscm{good-enough?} and \vscm{sqrt-iter}, the 
    above examples work.
\end{exe}

\begin{exe}[1.8]
    Here is a solution based on the solution of exercise \ref{1.7}.
    \scm{ch1/1.08.scm}
\end{exe}

\subsection{Procedures as Black-Box Abstractions}

This subsection contains no exercises.

\section{Procedures and the Processes They Generate}

\subsection{Linear Recursion and Iteration}

\begin{exe}[1.9]
    With the first procedure:
    \scm{ch1/1.09a.scm}

    With the second procedure:
    \scm{ch1/1.09b.scm}

    The first process is recursive, the second is iterative.
\end{exe}

\begin{exe}[1.10]
    Using the interpreter, we obtain $1024 = 2^{10}$ for \vscm{(A 1 10)}, and 
    $65536 = 2^{16}$ for \vscm{(A 2 4)} and \vscm{(A 3 3)}.

    By definition of the Ackermann function, \vscm{(A 0 n)}, i.e.
    \vscm{(f n)} computes $2n$.

    \medskip

    If $n > 0$, \vscm{(g n)} computes $2^n$.

    \begin{proof}
        By definition, \vscm{(g 1)} equals $2$, and for $n > 1$, \vscm{(A 1 n)} 
        equals \vscm{(A 0 (A 1 (- n 1)))}. Since \vscm{(A 0 n)} computes $2n$,
        the result follows by mathematical induction.
    \end{proof}

    \medskip

    If $n > 0$, \vscm{(h n)} computes $2 \uparrow \uparrow n$, that is, 
    $2^{2^{2^{…}}}$ with $n$ copies of $2$.

    \begin{proof}
        This is true for $n = 1$ by definition. For $n > 1$, \vscm{(A 2 n)} is 
        equal to\linebreak
        \vscm{(A 1 (A 2 (- n 1)))}. The result follows by mathematical induction 
        using the previous result.
    \end{proof}

    \medskip

    \begin{remark}
        \vscm{(A 3 3)} returns $2^{16} = 2^{2^{2^2}}$ as well. The recursion 
        beginning with \vscm{(A 3 n)} with\linebreak
        $n > 1$ gives \vscm{(A 2 (A 3 (- n 1)))}, so according to the previous 
        result, \vscm{(A 3 n)} is obtained from \vscm{(A 3 (- n 1))} by 
        computing $2^{2^{2^{…}}}$ with a tower of
        \vscm{(A 3 (- n 1))} 2s, so since \vscm{(A 3 1)} is $2$, \vscm{(A 3 2)} 
        is $2^2 = 4$, and \vscm{(A 3 3)} is $2^{2^{2^2}}$. The general value of 
        \vscm{(A 3 n)} can be noted $2 \uparrow \uparrow \uparrow n$, or $2 
        \uparrow^3 n$.

        This notation can be extended for all $m$s, so \vscm{(A m n)} computes 
        $2 \uparrow^m n$.
    \end{remark}
\end{exe}

\subsection{Tree Recursion}

\subsubsection{Example: Counting change}

\begin{exe}[1.11]
    Procedure computing $f$ by means of a recursive process :
    \scm{ch1/1.11a.scm}
    Procedure computing $f$ by means of a iterative process :
    \scm{ch1/1.11b.scm}
\end{exe}

\begin{exe}[1.12]
    Here is an example of a solution :
    \scm{ch1/1.12.scm}
    The argument $n$ is the line number from the top starting from $0$, and 
    $k$ is the column number from the left starting from $0$.
\end{exe}

\begin{exe}[1.13]
    \label{1.13}
    Let’s prove that for any $n \geq 0$, $\text{Fib}(n) = \sfrac{\left( \phi^n 
    - \psi^n \right)}{\sqrt{5}}$, where $\phi = \sfrac{\left( 1 + \sqrt{5} 
    \right)}{2}$ and $\psi = \sfrac{\left( 1 - \sqrt{5} \right)}{2}$.

    It’s true for $n = 0$ and $n = 1$.

    Let’s assume that it’s true for any $k < n$. We have :

    \begin{align*}
        \text{Fib}(n) &= \text{Fib}(n - 1) + \text{Fib}(n - 2) \\
        &= \frac{\phi^{n - 1} - \psi^{n - 1}}{\sqrt{5}} + \frac{\phi^{n - 2} 
        - \psi^{n - 2}}{\sqrt{5}} \\
        &= \frac{1}{\sqrt{5}}\left(\phi^{n - 2}\left(\phi + 1\right) - \psi^{n 
        - 2}\left(\psi + 1\right) \right) \\
    \end{align*}

    But $\phi$ and $\psi$ are the roots of the equation $x^2 - x - 1 = 0$, in 
    other words, $\phi^2 = \phi + 1$ and $\psi^2 = \psi + 1$, hence 
    $\text{Fib}(n) = \sfrac{\left(\phi^n - \psi^n \right)}{\sqrt{5}}$.

    Furthermore, $ \lvert 1 - \sqrt{5} \rvert < 2 $, so for any $n \geq 0$, 
    $ \lvert 1 - \sqrt{5} \rvert^n < 2^n $, so dividing by $2^n$, we get 
    $ \lvert \psi^n \rvert < 1 $, and by dividing by $\sqrt{5}$, $ \lvert 
    \sfrac{\psi^n}{\sqrt{5}} \rvert < \sfrac{1}{\sqrt{5}} $. Since 
    $ \sfrac{1}{\sqrt{5}} < \sfrac{1}{2} $, we have $ \lvert 
    \sfrac{\psi^n}{\sqrt{5}} \rvert < \sfrac{1}{2} $ for any $n \geq 0$, which 
    means that $\text{Fib}(n)$ is the closest integer to $\sfrac{\phi^n}{5}$.

\end{exe}

\subsection{Orders of Growth}

\begin{exe}[1.14]
    The space required is proportional to the maximum depth of the tree, so it 
    grows as $\Theta(n)$.

    For the time complexity, let’s use the mathematical notation $\text{cc}(n, 
    k)$ rather than \vscm{(cc n k)}.

    The time complexity for $\text{cc}(n, 1)$ grows as $\Theta(n)$.

    If we note $v$ the denomination of the $k$-th coin, we have:
    \begin{align*}
        \text{cc}(n, k) &= \text{cc}(n - v, k) + \text{cc}(n, k - 1) \\
                        &= \text{cc}(n - 2v, k) + 2\, \text{cc}(n, k - 1) \\
                        &= … \\
                        &= \text{cc}(n - \left\lceil \frac{n}{v} \right\rceil v, 
                        k) + \left\lceil \frac{n}{v} \right\rceil \text{cc}(n, 
                        k - 1)
    \end{align*}

    Since $n - \left\lceil \frac{n}{v} \right\rceil v \leq 0$, the time 
    complexity of $\text{cc}(n, k)$ is proportional to $n$ times the time 
    complexity of $\text{cc}(n, k - 1)$. As a consequence, the time complexity 
    for $5$ kinds of coins grows as $\Theta(n^5)$.
\end{exe}

\begin{exe}[1.15]
    \ \vspace{-20pt}
    \begin{itemize}
        \item[a.] If the argument is greater than $0.1$, \vscm{p} is called 
            once, and the argument is divided by three. So the number of steps 
            required is the smallest integer $n$ such that $\sfrac{12.15}{3^n} 
            < 0.1$, or equivalently $121.5 < 3^n$. The smallest such $n$ is 5, 
            so \vscm{p} is called 5 times when \vscm{(sine 12.15)} is evaluated.
        \item[b.] By the same calculation as above, if $a > 0.1$, the number of 
            steps is the smallest $n$ such that $10 \, a < 3^n$. By taking the 
            logarithm, we get $\log(10) + \log(a) < n \log(3)$, so $n 
            = \left\lceil \sfrac{(\log(10) + \log(a))}{\log(3)} \right\rceil$.

            Therefore, the number of steps has order of growth 
            $\Theta(\log(n))$. The space required is proportional to the number 
            of steps, so its order of growth is the same.
    \end{itemize}
\end{exe}

\subsection{Exponentiation}

\begin{exe}[1.16]
    A possible solution to compute exponentials in a logarithmic number of steps 
    iteratively:
    \scm{ch1/1.16.scm}
\end{exe}

\begin{exe}[1.17]
    A recursive process that multiplies two non-negative integers using 
    a logarithmic number of steps.
    \scm{ch1/1.17.scm}
\end{exe}

\begin{exe}[1.18]
    An iterative process that multiplies two non-negative integers using 
    a logarithmic number of steps.

    We keep a state variable $c$ such that $ab + c$ is constant at each call 
    of the inner function.
    \scm{ch1/1.18.scm}
\end{exe}

\begin{exe}[1.19]
    By calculation, we get $p' = p^2 + q^2$ and $q' = q^2 + 2pq$, so the 
    procedure becomes:
    \scm{ch1/1.19.scm}
\end{exe}

\subsection{Greatest Common Divisors}

\begin{exe}[1.20]
    With normal-order evaluation, \vscm{(gcd 206 40)} expands to
    \vscm{(gcd 40 (remainder 206 40))}, a \vscm{remainder} operation is 
    performed to test whether the remainder is null, then the expression expands 
    to \vscm{(gcd (remainder 206 40) (remainder 40 (remainder 206 40)))}. Two 
    \vscm{remainder} operations are performed to test whether the second 
    argument is null, and the expression expands to
    \begin{cscm}
        (gcd (remainder 40 (remainder 206 40))
             (remainder (remainder 206 40) (remainder 40 (remainder 206 40))))
    \end{cscm}
    Four new executions of \vscm{remainder} are necessary to determine that the 
    second argument is not null, and the expression becomes:
    \begin{cscm}
        (gcd (remainder (remainder 206 40) (remainder 40 (remainder 206 40)))
             (remainder (remainder 40 (remainder 206 40))
                        (remainder (remainder 206 40)
                                   (remainder 40 (remainder 206 40)))))
    \end{cscm}
    Seven executions of \vscm{remainder} are necessary to determine that the 
    second argument is null, and the GCD is computed with four executions of 
    \vscm{remainder}. In total, 18 \vscm{remainder} operations are performed in 
    the normal-order evaluation.

    \bigskip

    With applicative-order evaluation, \vscm{(gcd 206 40)} expands to
    \vscm{(gcd 40 6)}, then to \vscm{(gcd 6 4)}, \vscm{(gcd 4 2)},
    \vscm{(gcd 2 0)} and to 2. One \vscm{remainder} operation is performed each 
    time \vscm{b} is not null, so four such operations are performed.
\end{exe}

\subsection{Example: Testing for Primality}

\begin{exe}[1.21]
    The smallest divisors of 199, 1999 and 19999 are 199, 1999 and 
    7 respectively.
\end{exe}

\begin{exe}[1.22]
    Example solution:
    \scm{ch1/1.22.scm}

    Nowadays, it’s necessary to use numbers much larger than those suggested in 
    the book to test the prediction about the timing, but the data support the 
    $\sqrt{n}$ prediction.

    The result is compatible with the notion that programs run in time 
    proportional to the number of steps required for the computation.
\end{exe}

\begin{exe}[1.23]
    The next procedure is:
    \scm{ch1/1.23a.scm}
    and \vscm{smallest-divisor} becomes:
    \scm{ch1/1.23b.scm}

    The modified version does not run twice as fast, but only about 1.7 times 
    as fast as the original version. This is because time is necessary to 
    apply the \vscm{next} procedure at each step.
\end{exe}

\begin{exe}[1.24]
    The only change needed is to replace \vscm{prime?} with \vscm{fast-prime?} 
    using an arbitrary number of tests (100 here) in \vscm{start-prime-test}.
    \scm{ch1/1.24.scm}

    When the number of digits is doubled, the time needed should be doubled as 
    well since the Fermat test has logarithmic growth. This is what is found 
    experimentally, though again, it’s necessary to use numbers much larger 
    than 1000 and 1,000,000 to get significant results.
\end{exe}

\begin{exe}[1.25]
    Alyssa’s procedure computes the correct result but it is much slower 
    because it deals with huge numbers, whereas by taking the remainder at 
    each recursion step, the numbers remain smaller than the tested number.
\end{exe}

\begin{exe}[1.26]
    If we use an explicit multiplication rather than calling square, 
    \vscm{(expmod base (/ exp 2) m)} is computed twice rather than once at 
    each recursive call with \vscm{exp} even, so that the process becomes 
    linear again.
\end{exe}

\begin{exe}[1.27]
    Here is an example of a procedure that tells whether $a^n$ is congruent to 
    $a$ modulo $n$ for every $a < n$:
    \scm{ch1/1.27.scm}

    It returns true for the given Carmichael numbers.
\end{exe}

\begin{exe}[1.28]
    A possible way to implement the Miller-Rabin test is:
    \scm{ch1/1.28.scm}
    It returns false on the Carmichael numbers listed in footnote 47.
\end{exe}

\section{Formulating Abstractions with Higher-Order Procedures}

\subsection{Procedures as Arguments}

\begin{exe}[1.29]
    Here is a solution:
    \scm{ch1/1.29.scm}
\end{exe}

\begin{exe}[1.30]
    A sum procedure generating an iterative process:
    \scm{ch1/1.30.scm}
\end{exe}

\begin{exe}[1.31]
    \ \vspace{-20pt}
    \begin{itemize}
        \item[a.] Here is a procedure analogous to \vscm{sum} that computes 
            a product, generating a recursive process, and examples of its use 
            to define \vscm{factorial} and to compute approximations of $\pi$. 
            In the latter case, we use $\sfrac{(i - 1)(i + 1)}{i^2} = \sfrac{i^2 
            - 1}{i^2}$ as the general term.
            \scm{ch1/1.31a.scm}
        \item[b.] A \vscm{product} procedure generating an iterative process.
            \scm{ch1/1.31b.scm}
    \end{itemize}
\end{exe}

\begin{exe}[1.32]
    \ \vspace{-20pt}
    \begin{itemize}
        \item[a.] A recursive \vscm{accumulate} procedure and the definition of 
            \vscm{sum} and \vscm{product} using that procedure:
            \scm{ch1/1.32a.scm}
        \item[b.] An iterative version of \vscm{accumulate}:
            \scm{ch1/1.32b.scm}
    \end{itemize}
\end{exe}

\begin{exe}[1.33]
    A \vscm{filtered-accumulate} procedure generating a recursive process:
    \scm{ch1/1.33.1.scm}
    A \vscm{filtered-accumulate} procedure generating an iterative process:
    \scm{ch1/1.33.2.scm}
    \begin{itemize}
        \item[a.] Assuming \vscm{prime?} is already written, the sum of the 
            squares of the prime numbers in the interval $a$ to $b$ can be 
            computed with:
            \scm{ch1/1.33a.scm}
        \item[b.] The product of all positive integers less than $n$ that are 
            relatively prime to $n$ can be computed with:
            \scm{ch1/1.33b.scm}
    \end{itemize}
\end{exe}

\subsection{Construction Procedures Using \vscm{Lambda}}

\begin{exe}[1.34]
    If we try to evaluate \vscm{(f f)}, we get an error saying that the operator 
    is not a procedure. The reason is that \vscm{(f f)} evaluates to
    \vscm{(f 2)}, which itself evaluates to \vscm{(2 2)}, and this operation is 
    impossible since 2 is not a procedure.
\end{exe}

\subsection{Procedures as General Methods}
\label{1.3.3}

\begin{exe}[1.35]
    We already noticed in exercise \ref{1.13} that $\phi^2 = \phi + 1$. By 
    dividing this equation by $\phi$, we get $\phi = 1 + \sfrac{1}{\phi}$.

    We can then compute $\phi$ with the command:
    \scm{ch1/1.35.scm}
\end{exe}

\begin{exe}[1.36]
    Modified version of \vscm{fixed-points}:
    \scm{ch1/1.36a.scm}
    We can find a solution to $x^x = 1000$ with, for instance, without average 
    damping:
    \scm{ch1/1.36b.scm}
    And with average damping:
    \scm{ch1/1.36c.scm}

    The former takes 35 steps while the latter takes 9 steps, so average damping 
    makes the search much faster here.
\end{exe}

\begin{exe}[1.37]
    \ \vspace{-20pt}
    \begin{itemize}
        \item[a.] A procedure \vscm{cont-frac} generating an iterative process, 
            doing the computation starting from \vscm{k}:
            \scm{ch1/1.37a.scm}

            11 steps are necessary to get an approximation that is accurate to 
            4 decimal places.

        \item[b.] A procedure \vscm{cont-frac} generating a recursive process, 
            doing the computation starting from 1:
            \scm{ch1/1.37b.scm}
    \end{itemize}
\end{exe}

\begin{exe}[1.38]
    The following procedure computes an approximation of $e$ using a $k$-term 
    finite continued fraction.
    \scm{ch1/1.38.scm}
\end{exe}

\begin{exe}[1.39]
    A possible solution for \vscm{(tan-cf x k)}:
    \scm{ch1/1.39.scm}
\end{exe}

\subsection{Procedures as Returned Values}

\begin{exe}[1.40]
    The procedure \vscm{cubic} is:
    \scm{ch1/1.40.scm}
\end{exe}

\begin{exe}[1.41]
    The procedure \vscm{double}:
    \scm{ch1/1.41.scm}

    \vscm{(double double)} is a procedure that takes a procedure of one argument 
    as argument and returns a procedure that applies the original procedure four 
    times.

    \vscm{((double (double double)) f)} evaluates to
    \vscm{((double double) ((double double) f))}, so it returns a procedures 
    that applies \vscm{f} $4 \times 4 = 16$ times.

    So the value returned by \vscm{(((double (double double)) inc) 5)} is 21.
\end{exe}

\begin{exe}[1.42]
    Here is a procedure \vscm{compose}:
    \scm{ch1/1.42.scm}
\end{exe}

\begin{exe}[1.43]
    \label{1.43}
    A solution generationg a recursive process:
    \scm{ch1/1.43a.scm}

    A solution generationg an iterative process:
    \scm{ch1/1.43b.scm}
\end{exe}

\begin{exe}[1.44]
    A possible solution is:
    \scm{ch1/1.44.scm}

    The $n$-fold smoothed function of a function \vscm{f} can be obtained with
    \vscm{((repeated smooth n) f)}.
\end{exe}

\begin{exe}[1.45]
    Experimentally, we find that the number of average dampings necessary to 
    compute $n$th roots in this way is $\lfloor \log n \rfloor$, so the 
    procedure to compute $n$th roots is:
    \scm{ch1/1.45.scm}
\end{exe}

\begin{exe}[1.46]
    A solution for \vscm{iterative-improve}:
    \scm{ch1/1.46a.scm}

    The \vscm{sqrt} procedure of section \ref{1.1.7} becomes:
    \scm{ch1/1.46b.scm}

    The \vscm{fixed-point} procedure of section \ref{1.3.3} becomes:
    \scm{ch1/1.46c.scm}
\end{exe}

% vim:filetype=tex:set expandtab

\chapter{Building Abstractions with Data}

\section{Introduction to Data Abstraction}

\subsection{Example: Arithmetic Operations for Rational Numbers}

\begin{exe}[2.1]
    A possibility for a \vscm{make-rat} handling both positive and negative 
    arguments:
    \scm{ch2/2.01.scm}
\end{exe}

\subsection{Abstraction Barriers}

\begin{exe}[2.2]
    Exemple implementation for the representation of segments in a plane:
    \scm{ch2/2.02.scm}
\end{exe}

\begin{exe}[2.3]
    In the following implementation, a rectangle is represented by its two 
    opposite sides, which must have the same orientation. The code makes use of 
    auxiliary procedures defined below.

    I added selectors to access each of the vertices of the rectangle to be 
    able to print rectangles in a uniform format.
    \scm{ch2/2.03.1.scm}

    The procedures that compute the perimeter and area of a rectangle, and the 
    procedure that prints a rectangle, are defined thus:
    \scm{ch2/2.03.perim-area.scm}

    Another possibility is to represent a rectangle by its four vertices:
    \scm{ch2/2.03.2.scm}

    Yet another possibility is to represent a rectangle by two perpendicular 
    segments with the same origin:
    \scm{ch2/2.03.3.scm}

    The procedures \vscm{perim-rect}, \vscm{area-rect} and \vscm{print-rect} 
    work in all three cases.

    \begin{comp}
        The code above makes use of the following auxiliary procedures to check 
        that the input is correct, compute the length of a segment, and print 
        points inline for use in \vscm{print-rect}:
        \scm{ch2/2.03.aux.scm}
    \end{comp}
\end{exe}

\subsection{What Is Meant by Data?}

\begin{exe}[2.4]
    With the representation of pairs given in the exercise, \vscm{(cons x y)} is 
    a procedure that takes as its argument a procedure \vscm{m} with two 
    arguments and returns the result of the application of \vscm{m} to \vscm{x} 
    and \vscm{y}.

    \vscm{(car z)} applies the procedure \vscm{(cons x y)} to the procedure that 
    returns the first of its arguments, so \vscm{(car (cons x y))} yields 
    \vscm{x}.

    Using the substitution model, the successive steps are:
    \scm{ch2/2.04a.scm}

    The corresponding definition of \vscm{cdr} is:
    \scm{ch2/2.04b.scm}
    The method to prove that \vscm{(cdr (cons x y))} yields \vscm{y} is the same 
    as with \vscm{car}.
\end{exe}

\begin{exe}[2.5]
    If $a$ and $b$ are known, we can compute $2^a 3^b$, and since the 
    decomposition of integers as a product of primes is unique, it’s possible to 
    find $a$ and $b$ from the value of $2^a 3^b$.

    The procedures \vscm{cons}, \vscm{car}, and \vscm{cdr} corresponding to this 
    representation can be defined as:
    \scm{ch2/2.05.scm}
\end{exe}

\begin{exe}[2.6]
    The successive substitution steps to evaluate \vscm{(add-1 zero)} are:
    \scm{ch2/2.06a.scm}

    In other words, \vscm{one} is a procedure that takes a one-argument 
    procedure as its argument and returns it.

    The substitution steps to evaluate \vscm{(add-1 one)} are:
    \scm{ch2/2.06b.scm}

    In other words, \vscm{two} is a procedure that takes a one-argument 
    procedure $f$ as its argument and returns the procedure $f \circ f$ ($f$ 
    applied twice).

    From these observations, and after remarking that \vscm{zero} is a procedure 
    that takes one argument and always returns the identity procedure, we can 
    make the hypothesis that the $n$th Church numeral is a procedure that takes 
    a one-argument procedure $f$ as its argument and returns the $n$th repeated 
    application of $f$ (see exercise~\ref{1.43}). This can be proved by 
    induction.

    \begin{proof}
        We’ve already shown that it’s true for $0$, $1$ and $2$.
        Let’s assume that it’s true for a positive integer $n$.

        From the induction hypothesis, \vscm{(n f)} is the $n$th repeated 
        application of $f$, so it’s obvious that
        \vscm{(lambda (x) (f ((n f) x)))} is the $(n + 1)$th repeatead 
        application of $f$, so the result is true for $n + 1$, hence it’s true 
        for any positive integer $n$.
    \end{proof}

    To apply a function $n + m$ times, we just need to apply it $m$ times, and 
    then $n$ times more, so \vscm{+} can be defined directly as:
    \scm{ch2/2.06c.scm}
\end{exe}

\subsection{Extended Exercise: Interval Arithmetic}

Let’s first define a function that prints intervals:
\scm{ch2/2.07pre2.scm}

\begin{exe}[2.7]
    Since \vscm{make-interval} has been defined as \vscm{cons}, 
    \vscm{upper-bound} and \vscm{lower-bound} can be defined as \vscm{cdr} and 
    \vscm{car} respectively.
    \scm{ch2/2.07.scm}
\end{exe}

\begin{exe}[2.8]
    With the same reasoning as for division, the subtraction of two intervals is 
    the addition of the first with the opposite of the second.
    The subtraction procedure can thus be defined:
    \scm{ch2/2.08.scm}
\end{exe}

\begin{exe}[2.9]
    Let $[a;b]$ and $[c;d]$ be two intervals.

    Their sum is $[a + c; b + d]$. Its width is $\sfrac{(b + d) - (a + c)}{2} 
    = \sfrac{(b - a)}{2} + \sfrac{(d - c)}{2}$, in other words, the sum’s 
    width is the sum of the widths, so it depends only on the widths of the 
    intervals being added.

    The difference can be defined as the sum with the opposite, and taking the 
    opposite doesn’t change the width, so this is also true for differences.

    For multiplication and division, the width of the result also depends on 
    the values of the bounds. For instance, $[1; 2] \times [2; 3] = [2; 6]$, 
    but $[0; 1] \times [2; 3] = [0; 3]$. In both cases, we multiply two 
    intervals of width $\sfrac{1}{2}$, but the former product has a width of 
    $2$ while the latter has a width of $\sfrac{3}{2}$, so the width of the 
    product is not a function of the widths of the intervals being multiplied.

    Since division can be defined as a multiplication, this is also true for 
    division.
\end{exe}

\begin{exe}[2.10]
    The new code of \vscm{div-interval} could be:
    \scm{ch2/2.10.scm}
\end{exe}

\begin{exe}[2.11]
    There are three cases for each interval:
    \begin{itemize}
        \item the lower bound is positive or null;
        \item the upper bound is negative or null;
        \item the lower bound is negative and the upper bound is positive.
    \end{itemize}
    This results in a total of nine cases, and the only case where the smallest 
    and greatest products can’t be deduced from the signs is when both intervals 
    span zero.

    A procedure taking this suggestion into account is:
    \scm{ch2/2.11.scm}
\end{exe}

\begin{exe}[2.12]
    The procedures \vscm{make-center-percent} and \vscm{percent} can be defined 
    as:
    \scm{ch2/2.12.scm}
\end{exe}

\begin{exe}[2.13]
    Let $c_1, c_2, w_1$ and $w_2$ be the centers and widths of two intervals. We 
    assume that all numbers are positive. The lower and upper bounds of the 
    product are
    $(c_1 \pm w_1) \times (c_2 \pm w_2) = c_1 c_2 \pm (c_1 w_2 + c_2 w_1) + w_1 
    w_2$.

    Since the percentages are small, $w_1 w_2$ is negligible, and the product’s 
    width is $w \approx c_1 w_2 + c_2 w_1$.

    Additionally, if we call the percentage tolerances $p_1$ and $p_2$ 
    respectively, we have $w_i = c_i \times \sfrac{p_i}{100}$ for $i = 1, 2$.

    From there, $w \approx c_1 c_2 \times \sfrac{p_1 + p_2}{100}$, and $c_1 c_2$ 
    is the product’s center, so under the given conditions, the approximate 
    percentage tolerance of the product is the sum of the tolerances of the 
    factors.
\end{exe}

\begin{exe}[2.14]
    TODO
\end{exe}

\begin{exe}[2.15]
    TODO
\end{exe}

\begin{exe}[2.16]
    TODO
\end{exe}

\section{Hierarchical Data and the Closure Property}

\subsection{Representing Sequences}

\begin{exe}[2.17]
    The \vscm{last-pair} procedure can be defined as:
    \scm{ch2/2.17.scm}
\end{exe}

\begin{exe}[2.18]
    The \vscm{reverse} procedure can be defined as:
    \scm{ch2/2.18.scm}
\end{exe}

\begin{exe}[2.19]
    The procedures can be defined respectively as \vscm{car}, \vscm{cdr} and 
    \vscm{null?}.
    \scm{ch2/2.19.scm}

    The order of the list \vscm{coin-values} does not affect the answer produced 
    by \vscm{cc}, because \vscm{cc} gives the total number of combinations, and 
    the relation used for the computation does not depend on a particular order.
\end{exe}

\begin{exe}[2.20]
    A possible solution is:
    \scm{ch2/2.20.scm}
\end{exe}

\begin{exe}[2.21]
    The completed procedures are:
    \scm{ch2/2.21.scm}
\end{exe}

\begin{exe}[2.22]
    With the first procedure, the answer list is in reverse order because the 
    elements are added to it starting from the beginning of the initial list, 
    and the first element added to a list is at its end.

    With the second procedure, the result is not a list because \vscm{cons} is 
    called with a list as its first argument and the element to add as its 
    second argument. To add an element to a list, the order of the arguments 
    should be the opposite.
\end{exe}

\begin{exe}[2.23]
    Here is a possible implementation of \vscm{for-each}:
    \scm{ch2/2.23.scm}
\end{exe}

\subsection{Hierarchical Structures}

\begin{exe}[2.24]
    The result given by the interpreter is \vscm{(1 (2 (3 4)))}. To represent 
    the corresponding box-and-pointer structure in terms of pairs, one must use 
    the equality of \vscm{(list 1 (list 2 (list 3 4)))} and
    \vscm{(cons 1 (cons (list 2 (list 3 4)) nil))}, and similar equalities for 
    the two other lists.

    \begin{figure}
        \begin{center}
            TODO (too time-consuming…)
        \end{center}
        \caption{Box-and-pointer-structure of \vscm{(1 (2 (3 4)))}.}
    \end{figure}

    \vspace*{1cm}
    \begin{figure}
        \begin{center}
            \pstree[nodesepB=5pt]{\Tdot[tnpos=a]~{\vscm{(1 (2 (3 4)))}}}{%
            \TR[tnpos=b]{1}%
            \pstree{\Tdot[tnpos=r]~{\vscm{(2 (3 4))}}}{%
            \TR[tnpos=b]{2}%
            \pstree{\Tdot[tnpos=r]~{\vscm{(3 4)}}}{%
            \TR[tnpos=b]{3}\TR[tnpos=b]{4}}}}
        \end{center}
        \caption{Tree representation of \vscm{(1 (2 (3 4)))}.}
    \end{figure}
\end{exe}

\begin{exe}[2.25]
    If we call the three lists $x$, $y$ and $z$ respectively, the combinations 
    \vscm{(car (cdr (car (cdr (cdr x)))))}, \vscm{(car (car y))} (which can be 
    shortened to \vscm{(caar y)}), and\linebreak
    \vscm{(car (cdr (car (cdr (car (cdr (car (cdr (car (cdr (car (cdr z))))))))))))}
    or, more simply,
    \vscm{(cadr (cadr (cadr (cadr (cadr (cadr z))))))},
    all pick 7 from the lists.
\end{exe}

\begin{exe}[2.26]
    The results printed by the interpreter are \vscm{(1 2 3 4 5 6)} for
    \vscm{(append x y)}, \vscm{((1 2 3) 4 5 6)} for \vscm{(cons x y)} and
    \vscm{((1 2 3) (4 5 6))} for \vscm{(list x y)}.
\end{exe}

\begin{exe}[2.27]
    Here is a possible solution for \vscm{deep-reverse}:
    \scm{ch2/2.27.scm}
\end{exe}

\begin{exe}[2.28]
    A possible solution for \vscm{fringe} is:
    \scm{ch2/2.28.scm}
\end{exe}

\begin{exe}[2.29]
    \ \vspace{-20pt}
    \begin{itemize}
        \item[a.] The selectors can be defined as:
            \scm{ch2/2.29a.scm}
        \item[b.] The total weight can be computed with:
            \scm{ch2/2.29b.scm}
        \item[c.] Since we defined a \vscm{branch-weight} procedure in b., 
            \vscm{balanced?} can be defined simply with:
            \scm{ch2/2.29c.scm}
        \item[d.] To convert to the new representation, the only things that 
            need to be changed are the \vscm{right-branch} and 
            \vscm{branch-structure} selectors:
            \scm{ch2/2.29d.scm}
    \end{itemize}
\end{exe}

\begin{exe}[2.30]
    The two \vscm{square-list} procedures are identical to the \vscm{scale-tree} 
    procedures defined in the text, except that there is only one argument and 
    \vscm{(* tree factor)} is replaced with \vscm{(square tree)}.

    Direct definiton:
    \scm{ch2/2.30a.scm}

    Using \vscm{map} and recursion:
    \scm{ch2/2.30b.scm}
\end{exe}

\begin{exe}[2.31]
    The procedure \vscm{tree-map} can be defined without using \vscm{map}:
    \scm{ch2/2.31a.scm}
    or using it:
    \scm{ch2/2.31b.scm}
\end{exe}

\begin{exe}[2.32]
    The completed procedure is:
    \scm{ch2/2.32.scm}

    The empty set has only one subset: the empty set.

    For a non-empty (finite) set, with elements $\{a_1, …, a_n\}$, the set of 
    subsets is the reunion of the subsets not containing $a_1$ and the subsets 
    containing $a_1$, and the application $S \mapsto S \cup \{a_1\}$ is 
    a bijection between these two sets.
\end{exe}

\subsection{Sequences as Conventional Interfaces}

\begin{exe}[2.33]
    The operations can be redefined as:
    \scm{ch2/2.33.scm}
\end{exe}

\begin{exe}[2.34]
    A polynomial can be evaluated using Horner’s rule with the procedure:
    \scm{ch2/2.34.scm}
\end{exe}

\begin{exe}[2.35]
    This can be done with or without \vscm{enumate-tree}. Without that function:
    \scm{ch2/2.35.scm}
    The mapped function associates to each subtree its number of leaves: 1 if 
    the subtree has no children, i.e.\ is a leaf, \vscm{(count-leaves subtree)} 
    otherwise.
\end{exe}

\begin{exe}[2.36]
    The procedure \vscm{accumulate-n} can be defined as:
    \scm{ch2/2.36.scm}
\end{exe}

\begin{exe}[2.37]
    The matrix operation can be define as:
    \scm{ch2/2.37.scm}
\end{exe}

\chapter{Modularity, Objects, and State}

\section{Assignment and Local State}

\subsection{Local State Variables}

\begin{exe}[3.1]
    The procedure \vscm{make-accumulator} can be written:
    \scm{ch3/3.01.scm}
\end{exe}

\begin{exe}[3.2]
    The \vscm{make-monitored} procedure can be written:
    \scm{ch3/3.02.scm}
\end{exe}

\begin{exe}[3.3]
    The \vscm{make-account} procedure can be modified in the following way:
    \scm{ch3/3.03.scm}
\end{exe}

\begin{exe}[3.4]
    The procedure can be rewritten as:
    \scm{ch3/3.04.scm}
\end{exe}

\subsection{The Benefits of Introducing Assignment}

\begin{exe}[3.5]
    Using Gambit Scheme’s \vscm{random-real} procedure, that generates a random 
    real number between 0 and 1, \vscm{random-in-range} and the other procedures 
    can be written:
    \scm{ch3/3.05.scm}
\end{exe}

\begin{exe}[3.6]
    The \vscm{rand} procedure can be rewritten as:
    \scm{ch3/3.06.scm}
\end{exe}

\subsection{The Costs of Introducing Assignment}

\begin{exe}[3.7]
    I simply added a \vscm{join} action to the account returned by 
    \vscm{make-account} that creates an access with another password. I also 
    make \vscm{incorrect-password} throw an error instead of simply returning 
    a string, otherwise a call such as
    \vscm{(define new-acc (make-join account curr-pass new-pass))} with an 
    incorrect current password will affect a string value to \vscm{new-acc} 
    without reporting an error, and subsequent uses of the account will throw 
    errors because \vscm{"Incorrect password"} is not a procedure.
    \scm{ch3/3.07.scm}
\end{exe}

\begin{exe}[3.8]
    The procedure \vscm{f} returns:
    \begin{itemize}
        \item 0 if it is the first time it is called;
        \item the previous argument it was called with otherwise.
    \end{itemize}
    Thus, if we evaluate \vscm{(f 0)}, then \vscm{(f 1)}, we get 0 both times, 
    but if we evaluate \vscm{(f 1)}, then \vscm{(f 0)}, we get 0 the first 
    time and 1 the second.
    \scm{ch3/3.08.scm}
\end{exe}

\section{The Environment Model of Evaluation}

\subsection{The Rules for Evaluation}

This subsection contains no exercises.

\subsection{Applying Simple Procedures}

\var{\nametoenv}{4mm}
\var{\envtonext}{3mm}
\begin{exe}[3.9]
    \begin{figure}
        \begin{tikzpicture}[>=Stealth, thick]
            \node[text width=1cm, align=right] (ge) {global\\ env};
            \node[global env, right=\nametoenv of ge] (g) {factorial};
            \node[text width=1cm, align=right, below=of ge] (e1) {E1};
            \node[env, right=\nametoenv of e1] (n1) {n:\,6};
            \node[right=\envtonext of n1] (e2) {E2};
            \node[env, right=\nametoenv of e2] (n2) {n:\,5};
            \node[right=\envtonext of n2] (e3) {E3};
            \node[env, right=\nametoenv of e3] (n3) {n:\,4};
            \node[right=\envtonext of n3] (e4) {E4};
            \node[env, right=\nametoenv of e4] (n4) {n:\,3};
            \node[right=\envtonext of n4] (e5) {E5};
            \node[env, right=\nametoenv of e5] (n5) {n:\,2};
            \node[right=\envtonext of n5] (e6) {E6};
            \node[env, right=\nametoenv of e6] (n6) {n:\,1};

            \draw[->] (ge) -- (ge.west -| g.west);
            \foreach \i in {1, ..., 6} {
                \draw[->] (e\i) -- (n\i);
                \draw[->] (n\i.north) -- (n\i.north|-g.south);
                \node[code,below=2mm of n\i] { \vscm{(if ...)} };
            }
        \end{tikzpicture}
        \caption{Environments created by evaluating \vscm{(factorial 6)} with 
        the recursive procedure. In all the environments created, the code to 
        evaluate corresponds to the body of the \vscm{factorial} procedure.}
        \label{fact_rec}
    \end{figure}

    \begin{figure}
        \begin{tikzpicture}[>=Stealth, thick]
            \node[text width=1cm, align=right] (ge) {global\\ env};
            \node[global env, right=\nametoenv of ge] (g)
            {factorial\\ fact-iter};
            \node[text width=1cm, align=right, below=of ge] (e1) {E1};
            \node[env, right=\nametoenv of e1] (n1) {n:\,6};
            \node[right=\envtonext of n1] (e2) {E2};
            \node[env, right=\nametoenv of e2] (n2) {p:\,1\\ c:\,1\\ m:\,6};
            \node[right=\envtonext of n2] (e3) {E3};
            \node[env, right=\nametoenv of e3] (n3) {p:\,2\\ c:\,2\\ m:\,6};
            \node[right=(\envtonext+.6cm) of n3] (e4) {E4};
            \node[env, right=\nametoenv of e4] (n4) {p:\,6\\ c:\,3\\ m:\,6};
            \node[right=(\envtonext+.3cm) of n4] (e5) {E5};
            \node[env, right=\nametoenv of e5] (n5) {p:\,24\\ c:\,4\\ m:\,6};
            \node[below=1.9cm of n3] (e6) {E6};
            \node[env, right=\nametoenv of e6] (n6) {p:\,24\\ c:\,5\\ m:\,6};
            \node[right=\envtonext of n6] (e7) {E7};
            \node[env, right=\nametoenv of e7] (n7) {p:\,120\\ c:\,6\\ m:\,6};
            \node[right=\envtonext of n7] (e8) {E8};
            \node[env, right=\nametoenv of e8] (n8) {p:\,720\\ c:\,7\\ m:\,6};

            \draw[->] (ge) -- (ge.west -| g.west);
            \foreach \i in {1, ..., 8} {
                \draw[->] (e\i) -- (n\i);
                \draw[->] (n\i.north) -- (n\i.north|-g.south);
            }

            \node[code, below = 2mm of n1] { \vscm{(fact-iter 1 1 n)} };
            \foreach \i in {2, ..., 8} {
                \node[code, below = 2mm of n\i] { \vscm{(if ...)} };
            }
        \end{tikzpicture}
        \caption{Environments created by evaluating \vscm{(factorial 6)} with 
        the iterative procedure. In environments E2 to E8, the code to evaluate 
        corresponds to the body of the \vscm{fact-iter} procedure.}
        \label{fact_iter}
    \end{figure}

    The environment structure created by evaluating \vscm{(factorial 6)} with 
    both versions of the procedure are shown in figures \ref{fact_rec} and 
    \ref{fact_iter}.
\end{exe}

\subsection{Frames as the Repository of Local State}

\begin{exe}[3.10]
    \label{3.10}
    \begin{figure}
        \begin{tikzpicture}[>=Stealth, thick]
            \matrix[matrix anchor=north west, column sep=2\nametoenv, column 
            2/.style={anchor=base west}] at (0,0) {
            \node[text width=1cm, align=right] (ge) {global\\ env}; &
            \node[code] (mkw) {make-withdraw: ...}; \\[-8pt]
            & \node[code] (w1) {W1: }; \\
            };
            \begin{scope}[on background layer]
                \coordinate (w1b) at ($ (w1.south) - (0, 10pt) $);
                \node[global env, fit=(mkw) (w1b), right=\nametoenv of ge] (g) 
                {};
            \end{scope}
            \node[text width=1cm, align=right, below=of g] (e1) {E1};
            \node[env, right=\nametoenv of e1] (n1) {initial-amount:\,100};
            \node[env, below=of n1] (n2) {balance:\,100};
            \node[left=\nametoenv of n2] (e2) {E2};
            \node[env, below=of n2] (n3) {amount:\,50};

            \coordinate[left=3cm of e1] (p1);
            \procedure{p1}{c1}{c2}
            \node[code, below=of p1] (code) {
            parameters: amount\\
            body: (if ...)
            };
            \node[code, below=1ex of n3] { (if ...) };

            \draw[->] (c1) -- (c1 |- code.north);
            \draw[->] (c2) -- ++(0, -8mm) -| ($ (n2.north) - (.5cm, 0) $);

            \draw[->] (ge) -- (ge.west -| g.west);
            \foreach \i in {1, 2}
                \draw[->] (e\i) -- (n\i);
            \draw[->] (n1.north) -- (n1.north |- g.south);
            \draw[->] (n2) -- (n1);
            \draw[->] (n3) -- (n2);
            \draw[->] (w1) -| ($ (p1) + (0, \circleradius) $);
        \end{tikzpicture}
        \caption{Environments created when executing
        \vscm{(W1 50)} after executing
        \vscm{(define (W1 (make-withdraw 100))}. The environment E1 is created 
        by the call to \vscm{make-withdraw}, E2 is created when the lambda 
        procedure created by the \vscm{let} is executed. E2 is referenced by the 
        procedure returned by \vscm{make-withdraw}. When \vscm{(W1 50)} is 
        called, a new environment pointing to E2 is created, in which the body 
        of \vscm{W1} is evaluated.}
        \label{3.10figa}
    \end{figure}

    \begin{figure}
        \begin{tikzpicture}[>=Stealth, thick]
            \matrix[matrix anchor=north west, column sep=2\nametoenv, column 
            2/.style={anchor=base west}] at (0,0) {
            & \node[code] (mkw) {make-withdraw: ...}; \\
            \node[text width=1cm, align=right] (ge) {global\\ env};& \node[code] 
            (w1) {W1: }; \\[-8pt]
            & \node[code] (w2) {W2: }; \\
            };
            \begin{scope}[on background layer]
                \coordinate (mkwb) at ($ (mkw.north) + (0, 20pt) $);
                \node[global env, fit=(mkwb) (w1), right=\nametoenv of ge] (g) 
            {};
            \end{scope}
            \coordinate (e1pos) at ($ (g.south west)!.65!(g.south east) $);
            \node[text width=1cm, align=right, below=of e1pos] (e1) {E1};
            \node[env, right=\nametoenv of e1] (n1) {initial-amount:\,100};
            \node[env, below=of n1] (n2) {balance:\,50};
            \node[left=\nametoenv of n2] (e2) {E2};

            \coordinate[left=.8cm of e1] (p1);
            \procedure{p1}{c1}{c2}
            \node[code, below=of p1] (code) {
            parameters: amount\\
            body: (if ...)
            };

            \coordinate (e3pos) at (ge |- g.south east);
            \node[text width=1cm, align=right, below=of e3pos] (e3) {E3};
            \node[env, right=\nametoenv of e3] (n3) {initial-amount:\,100};
            \node[env, below=of n3] (n4) {balance:\,100};
            \node[left=\nametoenv of n4] (e4) {E4};

            \draw[->] (c1) -- (c1 |- code.north);
            \draw[->] (c2) -- ++(0, -8mm) -| ($ (n2.north) - (.5cm, 0) $);
            \draw[->] (w1) -| ($ (p1) + (0, \circleradius) $);

            \coordinate[left=2cm of p1] (p2);
            \procedure{p2}{c3}{c4}
            \draw[->] (c3) |- (code.west);
            \draw[->] (c4) -- ++(0, -8mm) -| ($ (n4.north) + (.5cm, 0) $);
            \draw[->] (w2) -| ($ (p2) + (0, \circleradius) $);

            \draw[->] (ge) -- (ge.west -| g.west);
            \foreach \i in {1, ..., 4}
                \draw[->] (e\i) -- (n\i);
            \draw[->] (n1.north) -- (n1.north |- g.south);
            \draw[->] (n3.north) -- (n3.north |- g.south);
            \draw[->] (n2) -- (n1);
            \draw[->] (n4) -- (n3);
        \end{tikzpicture}
        \caption{Environments created after the execution of
        \vscm{(define W1 (make-withdraw 100))}, followed by \vscm{(W1 50)}, then 
        \vscm{(define W2 (make-withdraw 100))}. The call to \vscm{W1} modified 
        the value of \vscm{balance} in E2, but \vscm{W2} uses the \vscm{balance} 
        variable of environment E4.}
        \label{3.10figb}
    \end{figure}

    The environments created after the execution of the three commands are shown 
    in figures \ref{3.10figa} and \ref{3.10figb}, see the captions for some 
    details. As with the first version of \vscm{make-version}, each object 
    created with a call to \vscm{make-version} uses a \vscm{balance} binding 
    situated in an environment specific to the object.

    In the second version, two environments are created instead of one, and the 
    value of \vscm{initial-value} is unchanged.
\end{exe}

\subsection{Internal Definitions}

\begin{exe}[3.11]
    \begin{figure}
        \begin{tikzpicture}[>=Stealth, thick]
            \matrix[matrix anchor=base west, column sep=2\nametoenv, column 
            2/.style={anchor=base west}] at (0,0) {
            \node[text width=1cm, align=right] (ge) {global\\ env};
            & \node[code] (mkw) {make-account: ...}; \\[-8pt]
            & \node[code] (acc) {acc: }; \\
            };
            \coordinate (e1pos) at ($ (g.south west)!.15!(g.south east) $);
            \matrix[matrix anchor=north west, column sep=2\nametoenv,
            below=1mm of e1pos, column 2/.style={anchor=base west}] {
            \node[text width=1cm, align=right] (e1) {E1};
            & \node[code] (balance) { balance: 50}; \\[-11pt]
            & \node[code] { withdraw: ... }; \\
            & \node[code] { deposit: ... }; \\
            & \node[code] (dispatch) { dispatch: }; \\
            };
            \begin{scope}[on background layer]
                \coordinate (accb) at ($ (acc.south) - (0, 20pt) $);
                \node[global env, fit=(mkw) (accb), right=\nametoenv of ge] (g) 
                    {};
                \node[env, minimum width=4cm, fit=(balance) (dispatch), 
                right=\nametoenv of e1, yshift=-3mm] (n1) {};
            \end{scope}

            \coordinate (xp1) at ($ (n1.south)!.50!(n1.south east) $);
            \coordinate[below=of xp1] (p1);
            \procedure{p1}{c1}{c2}
            \node[code, below=of p1] (code) {
            parameters: m\\
            body: (cond ...)
            };

            \draw[->] (c1) -- (c1 |- code.north);
            \draw[->] (c2) -| ($ (n1.south east) - (3mm, 0) $);

            \draw[->] (ge) -- (ge.west -| g.west);
            \draw[->] (e1) -- (e1.east -| n1.west);
            \draw[->] (n1.north) -- (n1.north |- g.south);
            \draw[->] (dispatch) -| ($ (p1) + (0, \circleradius) $);

            \draw[->] (acc) -- ($ (acc) + (1cm, 0) $)
                |- ($ (p1) + (-2\circleradius, 0) $);

            % Evaluation of ((acc 'deposit) 40)
            \coordinate[right=2cm of n1.east] (middle);
            \node[env, above=1em of middle] (m)
            {m: 'deposit};
            \node[env, below=1em of middle] (amount) {amount: 40};
            \node[right=of m] (e2) {E2};
            \node[right=of amount] (e3) {E3};
            \draw[->] (e2) -- (m);
            \draw[->] (e3) -- (amount);
            \draw[->] (m) -- (m -| n1.east);
            \draw[->] (amount) -- (amount -| n1.east);
        \end{tikzpicture}

        \caption{Environments when evaluating \vscm{((acc 'deposit) 40)}
        after evaluating \vscm{(define acc (make-account 50))}. E1 is created 
        when defining \vscm{acc}, then the evaluation of \vscm{(acc 'deposit)}
        causes the creation of an environment referencing E1, and since the 
        result of the evaluation is the procedure \vscm{deposit}, \vscm{(deposit 
        40)} is then evaluated in a new environment.}
        \label{3.11figa}
    \end{figure}

    \begin{figure}
        \begin{tikzpicture}[>=Stealth, thick]
            \matrix[matrix anchor=base west, column sep=2\nametoenv, column 
            2/.style={anchor=base west}] at (0,0) {
            \node[text width=1cm, align=right] (ge) {global\\ env};
            & \node[code] (mkw) {make-account: ...}; \\[-8pt]
            & \node[code] (acc) {acc: }; \\
            };
            \coordinate (e1pos) at ($ (g.south west)!.15!(g.south east) $);
            \matrix[matrix anchor=north west, column sep=2\nametoenv,
            below=1cm of e1pos, column 2/.style={anchor=base west}] {
            \node[text width=1cm, align=right] (e1) {E1};
            & \node[code] (balance) { balance: 90}; \\[-11pt]
            & \node[code] { withdraw: ... }; \\
            & \node[code] { deposit: ... }; \\
            & \node[code] (dispatch) { dispatch: }; \\
            };
            \begin{scope}[on background layer]
                \coordinate (accb) at ($ (acc.south) - (0, 20pt) $);
                \node[global env, fit=(mkw) (accb), right=\nametoenv of ge] (g) 
                    {};
                \node[env, minimum width=4cm, fit=(balance) (dispatch), 
                right=\nametoenv of e1, yshift=-3mm] (n1) {};
            \end{scope}

            \coordinate (xp1) at ($ (n1.south)!.50!(n1.south east) $);
            \coordinate[below=of xp1] (p1);
            \procedure{p1}{c1}{c2}
            \node[code, below=of p1] (code) {
            parameters: m\\
            body: (cond ...)
            };

            \draw[->] (c1) -- (c1 |- code.north);
            \draw[->] (c2) -| ($ (n1.south east) - (3mm, 0) $);

            \draw[->] (ge) -- (ge.west -| g.west);
            \draw[->] (e1) -- (e1.east -| n1.west);
            \draw[->] (n1.north) -- (n1.north |- g.south);
            \draw[->] (dispatch) -| ($ (p1) + (0, \circleradius) $);

            \draw[->] (acc) -- ($ (acc) + (1cm, 0) $)
                |- ($ (p1) + (-2\circleradius, 0) $);

            % Evaluation of ((acc 'withdraw) 60)
            \coordinate[right=2cm of n1.east] (middle);
            \node[env, above=1em of middle] (m)
            {m: 'withdraw};
            \node[env, below=1em of middle] (amount) {amount: 60};
            \node[right=of m] (e4) {E4};
            \node[right=of amount] (e5) {E5};
            \draw[->] (e4) -- (m);
            \draw[->] (e5) -- (amount);
            \draw[->] (m) -- (m -| n1.east);
            \draw[->] (amount) -- (amount -| n1.east);
        \end{tikzpicture}

        \caption{Environments during the evaluation of
        \vscm{((acc 'withdraw) 60)}.}
        \label{3.11figb}
    \end{figure}

    The environments generated by the evaluation of
    \vscm{(define acc (make-account 50))},\\
    \vscm{((acc 'deposit) 40)} and \vscm{((acc 'withdraw) 60)} are shown in 
    figures \ref{3.11figa} and \ref{3.11figb}. The local state for \vscm{acc} is 
    kept in the local environment referenced by the procedure object referenced 
    by \vscm{acc}.

    If a second environment is created, its local state is kept in a new 
    environment created when evaluating the \vscm{make-account} procedure, so it 
    does not interfere with \vscm{acc}’s local environment.

    The environment structures of \vscm{acc} and \vscm{acc2} share the global 
    environment.
\end{exe}

\section{Modeling with Mutable Data}

\subsection{Mutable List Structure}

\begin{exe}[3.12]
    \begin{figure}
        \centering
        \begin{tikzpicture}[box and pointer]
            \matrix[cell matrix] {
            % x
            \node[struct name] (x) {x}; &[+2\boxsize]
            \node[car] (carx) {}; & \node[cdr] (cdrx) {}; &[+\boxsize]
            \node[car] (cadrx) {}; & \node[cdr] (cddrx) {}; \\

            % Values
            & \node[box] (a) {a}; &&
            \node[box] (b) {b}; &&[+2\boxsize]
            \node[box] (c) {c}; &&[+\boxsize]
            \node[box] (d) {d}; & \\

            % z and y
            \node[struct name] (z) {z}; &[+2\boxsize]
            \node[car] (carz) {}; & \node[cdr] (cdrz) {}; &
            \node[car] (cadrz) {}; & \node[cdr] (cddrz) {}; &
            \node[car] (cary) {}; & \node[cdr] (cdry) {}; &
            \node[car] (cadry) {}; & \node[cdr] (cddry) {}; &
            \\
            };

            \draw[pointer] (x) -- (carx);
            \link{carx}{a}
            \link{cdrx}{cadrx}
            \link{cadrx}{b}
            \nil{cddrx}

            \draw[pointer] (z) -- (carz);
            \link{carz}{a}
            \link{cdrz}{cadrz}
            \link{cadrz}{b}
            \link{cddrz}{cary}
            \link{cary}{c}
            \link{cdry}{cadry}
            \link{cadry}{d}
            \nil{cddry}

            \coordinate(ypos) at ($ (cary.north west)!.4!(cary.west) $);
            \node[struct name, left=1.5\boxsize of ypos] (y) {y};
            \draw[pointer] (y) -- (ypos);
        \end{tikzpicture}
        \caption{The lists \vscm{x}, \vscm{y} and \vscm{z} right after the 
        definition of \vscm{z}.}
        \label{3.12z}
    \end{figure}

    \begin{figure}
        \centering
        \begin{tikzpicture}[box and pointer]
            \matrix[cell matrix] {
            % z and y
            \node[struct name] (x) {x}; &[+2\boxsize]
            \node[car] (carx) {}; & \node[cdr] (cdrx) {}; &[+\boxsize]
            \node[car] (cadrx) {}; & \node[cdr] (cddrx) {}; &[+3\boxsize]
            \node[car] (cary) {}; & \node[cdr] (cdry) {}; &[+\boxsize]
            \node[car] (cadry) {}; & \node[cdr] (cddry) {}; & \\

            % Values
            & \node[box] (a) {a}; &&
            \node[box] (b) {b}; &&[+2\boxsize]
            \node[box] (c) {c}; &&[+\boxsize]
            \node[box] (d) {d}; & \\
            };

            \draw[pointer] (x) -- (carx);
            \link{carx}{a}
            \link{cdrx}{cadrx}
            \link{cadrx}{b}
            \link{cddrx}{cary}

            \link{cary}{c}
            \link{cdry}{cadry}
            \link{cadry}{d}
            \nil{cddry}

            \coordinate(wpos) at ($ (carx.north west)!.4!(carx.west) $);
            \node[struct name, left=3\boxsize of wpos] (w) {w};
            \draw[pointer] (w) -- (wpos);

            \coordinate(ypos) at ($ (cary.north west)!.4!(cary.west) $);
            \node[struct name, left=1.5\boxsize of ypos] (y) {y};
            \draw[pointer] (y) -- (ypos);
        \end{tikzpicture}
        \caption{The lists \vscm{x}, \vscm{y} and \vscm{w} right after the 
        definition of \vscm{w}.}
        \label{3.12w}
    \end{figure}
    The first response is \vscm{(b)}, the second response is \vscm{(b c d)}. The 
    figure \ref{3.12z} shows the lists \vscm{x}, \vscm{y} and \vscm{z} right 
    after the definition of \vscm{z}. The figure \ref{3.12w} shows the lists 
    \vscm{x}, \vscm{y} and \vscm{w} after the definition of \vscm{w}.
\end{exe}

\begin{exe}[3.13]
    \label{3.13}
    \begin{figure}
        \centering
        \begin{tikzpicture}[box and pointer]
            \matrix[cell matrix] {
            \node[car] (carz) {}; & \node[cdr] (cdrz) {}; &[+\boxsize]
            \node[car] (cadrz) {}; & \node[cdr] (cddrz) {}; &[+\boxsize]
            \node[car] (caddrz) {}; & \node[cdr] (cdddrz) {}; \\

            % Values
            \node[box] (a) {a}; &&
            \node[box] (b) {b}; &&[+\boxsize]
            \node[box] (c) {c}; \\
            };

            \link{carz}{a}
            \link{cdrz}{cadrz}
            \link{cadrz}{b}
            \link{cddrz}{caddrz}
            \link{caddrz}{c}

            \coordinate(zpos) at ($ (carz.north west)!.6!(carz.west) $);
            \node[struct name, left=2\boxsize of zpos] (z) {z};
            \draw[pointer] (z) -- (zpos);

            \coordinate(in) at ($ (carz.south west)!.6!(carz.west) $);
            \draw[box pointer] (cdddrz.base) -- ++(0, -3.5\boxsize) -|
                ($ (in) - (\boxsize, 0) $) -- (in);
        \end{tikzpicture}
        \caption{The structure created by
        \vscm{(define z (make-cycle (list 'a 'b 'c))}.}
        \label{3.13fig}
    \end{figure}
    The structure \vscm{z} is shown in figure \ref{3.13fig}.

    If we try to compute \vscm{(last-pair z)}, we get an infinite loop since 
    \vscm{z} has a cycle.
\end{exe}

\begin{exe}[3.14]
    \begin{figure}
        \centering
        \begin{tikzpicture}[box and pointer]
            \matrix[cell matrix] {
            \node[struct name] (v) {v}; &[+2\boxsize]
            \node[car] (c11) {}; & \node[cdr] (c12) {}; &[+\boxsize]
            \node[car] (c21) {}; & \node[cdr] (c22) {}; &[+\boxsize]
            \node[car] (c31) {}; & \node[cdr] (c32) {}; &[+\boxsize]
            \node[car] (c41) {}; & \node[cdr] (c42) {}; & \\

            % Values
            & \node[box] (a) {a}; &&
            \node[box] (b) {b}; &&[+\boxsize]
            \node[box] (c) {c}; &&[+\boxsize]
            \node[box] (d) {d}; & \\
            };

            \draw[pointer] (v) -- (c11);
            \link{c11}{a}
            \link{c12}{c21}
            \link{c21}{b}
            \link{c22}{c31}
            \link{c31}{c}
            \link{c32}{c41}
            \link{c41}{d}
            \nil{c42}
        \end{tikzpicture}
        \caption{The list to which \vscm{v} is bound initially.}
        \label{3.14v}
    \end{figure}

    \begin{figure}
        \centering
        \begin{tikzpicture}[box and pointer]
            \matrix[cell matrix] {
            \node[struct name] (w) {w}; &[+2\boxsize]
            \node[car] (c11) {}; & \node[cdr] (c12) {}; &[+\boxsize]
            \node[car] (c21) {}; & \node[cdr] (c22) {}; &[+\boxsize]
            \node[car] (c31) {}; & \node[cdr] (c32) {}; &[+3\boxsize]
            \node[car] (c41) {}; & \node[cdr] (c42) {}; & \\

            % Values
            & \node[box] (d) {d}; &&
            \node[box] (c) {c}; &&[+\boxsize]
            \node[box] (b) {b}; &&[+3\boxsize]
            \node[box] (a) {a}; & \\
            };

            \draw[pointer] (w) -- (c11);
            \link{c11}{d}
            \link{c12}{c21}
            \link{c21}{c}
            \link{c22}{c31}
            \link{c31}{b}
            \link{c32}{c41}
            \link{c41}{a}
            \nil{c42}

            \coordinate(vpos) at ($ (c41.north west)!.6!(c41.west) $);
            \node[struct name, left=2\boxsize of vpos] (v) {v};
            \draw[pointer] (v) -- (vpos);
        \end{tikzpicture}
        \caption{The lists \vscm{v} and \vscm{w} after calling \vscm{mystery}.}
        \label{3.14w}
    \end{figure}
    The \vscm{mystery} procedure reverses the elements of the list. Figure 
    \ref{3.14v} shows the list \vscm{v} as it is initially, and figure 
    \ref{3.14w} shows the lists \vscm{v} and \vscm{w} after evaluating 
    \vscm{(define w (mystery v))}. The values of \vscm{v} and \vscm{w} would be 
    \vscm{(a)} and \vscm{(d c b a)}.
\end{exe}

\begin{exe}[3.15]
    \begin{figure}
        \centering
        \begin{tikzpicture}[box and pointer]
            \matrix[cell matrix] {
            \node[struct name] (z1) {z1}; &[+2\boxsize]
            \node[car] (c11) {}; & \node[cdr] (c12) {}; \\

            &
            \node[car] (c21) {}; & \node[cdr] (c22) {}; &[+\boxsize]
            \node[car] (c23) {}; & \node[cdr] (c24) {}; \\

            &
            \node[box] (wow) {wow}; &&
            \node[box] (b) {b}; \\
            };

            \draw[pointer] (z1) -- (c11);
            \link{c11}{c21}
            \link{c12}{c22}
            \link{c21}{wow}
            \link{c22}{c23}
            \link{c23}{b}
            \nil{c24}
        \end{tikzpicture}
        \caption{The list \vscm{z1} after applying \vscm{set-to-wow!} to it.}
        \label{3.15z1}
    \end{figure}

    \begin{figure}
        \centering
        \begin{tikzpicture}[box and pointer]
            \matrix[cell matrix] {
            \node[struct name] (z2) {z2}; &[+2\boxsize]
            \node[car] (c11) {}; & \node[cdr] (c12) {}; &[+\boxsize]
            \node[car] (c13) {}; & \node[cdr] (c14) {}; &[+\boxsize]
            \node[car] (c15) {}; & \node[cdr] (c16) {}; \\

            &&&
            \node[box] (a) {a}; &&
            \node[box] (b) {b}; \\

            &&&
            \node[car] (c21) {}; & \node[cdr] (c22) {}; &
            \node[car] (c23) {}; & \node[cdr] (c24) {}; \\

            &&&
            \node[box] (wow) {wow}; \\
            };

            \draw[pointer] (z2) -- (c11);
            \draw[box pointer] (c11.base) |- (c21);
            \link{c12}{c13}
            \link{c13}{a}
            \link{c14}{c15}
            \link{c15}{b}
            \nil{c16}
            \link{c21}{wow}
            \link{c22}{c23}
            \link{c23}{b}
            \nil{c24}
        \end{tikzpicture}
        \caption{The list \vscm{z2} after applying \vscm{set-to-wow!} to it.}
        \label{3.15z2}
    \end{figure}
    The figures \ref{3.15z1} and \ref{3.15z2} show the effect of 
    \vscm{set-to-wow!} on \vscm{z1} and \vscm{z2}.
\end{exe}

\begin{exe}[3.16]
    \begin{figure}
        \centering
        \begin{tikzpicture}[box and pointer,
            every label/.style={font=\sffamily}]
            \hspace{-1.3cm} % To center the figure better.
            % Example returning 3.
            \matrix[cell matrix, label=below:{List structure returning 3.}] (e1) 
            {
            \node[struct name] (x) {x}; &[+\boxsize]
            \node[car] (x11) {}; & \node[cdr] (x12) {}; &[+\boxsize]
            \node[car] (x21) {}; & \node[cdr] (x22) {}; &[+\boxsize]
            \node[car] (x31) {}; & \node[cdr] (x32) {}; \\

            & \node[box] (xa) {a}; &&
            \node[box] (xb) {b}; &&[+\boxsize]
            \node[box] (xc) {c}; \\
            };

            \draw[pointer] (x) -- (x11);
            \link{x11}{xa}
            \link{x12}{x21}
            \link{x21}{xb}
            \link{x22}{x31}
            \link{x31}{xc}
            \nil{x32}

            % Example returning 4.
            \matrix[cell matrix, right=of e1, label=below:{List structure 
            returning 4.}] (e2) {
            \node[struct name] (y) {y}; &[+\boxsize]
            \node[car] (y11) {}; & \node[cdr] (y12) {}; &[+\boxsize]
            \node[car] (y21) {}; & \node[cdr] (y22) {}; &[+\boxsize]
            \node[car] (y31) {}; & \node[cdr] (y32) {}; \\

            &&&
            \node[box] (ya) {a}; &&[+\boxsize]
            \node[box] (yb) {b}; \\
            };

            \draw[pointer] (y) -- (y11);
            \draw[box pointer] (y11.base) -- ++(0, 1.2\boxsize) -| (y31);
            \link{y12}{y21}
            \link{y21}{ya}
            \link{y22}{y31}
            \link{y31}{yb}
            \nil{y32}

            % Example returning 7.
            \matrix[cell matrix, below=4em of e1, label=below:{List structure 
            returning 7.}] (e3) {
            \node[struct name] (z) {z}; &[+\boxsize]
            \node[car] (z11) {}; & \node[cdr] (z12) {}; &[+\boxsize]
            \node[car] (z21) {}; & \node[cdr] (z22) {}; &[+\boxsize]
            \node[car] (z31) {}; & \node[cdr] (z32) {}; \\

            &&&&&
            \node[box] (za) {a}; \\
            };

            \draw[pointer] (z) -- (z11);
            \draw[box pointer] (z11.base) -- ++(0, -1.2\boxsize) -| (z21);
            \link{z12}{z21}
            \draw[box pointer] (z21.base) -- ++(0, 1.2\boxsize) -| (z31);
            \link{z22}{z31}
            \link{z31}{za}
            \nil{z32}

            % Example never returning.
            \matrix[cell matrix, right=of e3, label=below:{List structure never 
            returning.}] (e4) {
            \node[struct name] (t) {t}; &[+\boxsize]
            \node[car] (t11) {}; & \node[cdr] (t12) {}; &[+\boxsize]
            \node[car] (t21) {}; & \node[cdr] (t22) {}; &[+\boxsize]
            \node[car] (t31) {}; & \node[cdr] (t32) {}; \\

            & \node[box] (ta) {a}; &&
            \node[box] (tb) {b}; &&[+\boxsize]
            \node[box] (tc) {c}; \\
            };

            \draw[pointer] (t) -- (t11);
            \link{t11}{ta}
            \link{t12}{t21}
            \link{t21}{tb}
            \link{t22}{t31}
            \link{t31}{tc}
            \draw[box pointer] (t32.base) -- ++(0, 1.2\boxsize) -| (t11);
        \end{tikzpicture}
        \caption{Structures made of exactly three pairs for which Ben’s 
            procedure returns different values.}
        \label{3.16ex}
    \end{figure}
    Figure \ref{3.16ex} shows examples of list structures made up of exactly 
    three pairs for which Ben’s procedure returns 3, 4, 7, or never at all.
    These structures can be defined in the following way, using 
    \vscm{make-cycle} from exercise \ref{3.13} for the last one.
    \scm{ch3/3.16.scm}
\end{exe}

\begin{exe}[3.17]
    A possible solution is:
    \scm{ch3/3.17.scm}
\end{exe}

\begin{exe}[3.18]
    Here is a possible solution:
    \scm{ch3/3.18.scm}
\end{exe}

\begin{exe}[3.19]
    We go through the list with two pointers: one advancing one step at a time, 
    the other advancing two steps at a time. If the list contains a cycle, 
    they’ll end up pointing to the same pair after a while. Otherwise, the 
    second one will reach the end of the list.
    \scm{ch3/3.19.scm}
\end{exe}

\begin{exe}[3.20]
    \begin{figure}
        \begin{tikzpicture}[>=Stealth, thick]
            \matrix[matrix anchor=base west, column sep=2\nametoenv, column 
            2/.style={anchor=base west}] at (0,0) {
            & \node[code] (cons) {cons: ...}; \\[-3pt]
            \node[text width=1cm, align=right] (ge) {global\\ env};
            & \node[code] (z) {z: }; \\[-8pt]
            & \node[code] (x) {x: }; \\
            };

            % E1
            \coordinate (e1pos) at ($ (g.south west)!.10!(g.south east) $);
            \matrix[matrix anchor=north west, column sep=2\nametoenv,
            below=of e1pos, column 2/.style={anchor=base west}] {
            \node[text width=1cm, align=right] (e1) {E1};
            & \node[code] (x1) { x: 1}; \\[-11pt]
            & \node[code] (y1) { y: 2}; \\
            & \node[code] { set-x!: ... }; \\
            & \node[code] { set-y!: ... }; \\
            & \node[code] (dispatch1) { dispatch: }; \\
            };

            % E2
            \coordinate (e2pos) at ($ (g.south west)!.55!(g.south east) $);
            \matrix[matrix anchor=north west, column sep=2\nametoenv,
            below=of e2pos, column 2/.style={anchor=base west}] {
            \node[text width=1cm, align=right] (e2) {E2};
            & \node[code] (x2) { x: x}; \\[-11pt]
            & \node[code] (y2) { y: x}; \\
            & \node[code] { set-x!: ... }; \\
            & \node[code] { set-y!: ... }; \\
            & \node[code] (dispatch2) { dispatch: }; \\
            };

            % Backgrounds
            \begin{scope}[on background layer]
                \node[global env, fit=(cons) (x), right=\nametoenv of ge, 
                yshift=1mm] (g) {};
                \node[env, minimum width=4cm, fit=(x1) (dispatch1), 
                right=\nametoenv of e1, yshift=-6mm] (n1) {};
                \node[env, minimum width=4cm, fit=(x2) (dispatch2), 
                right=\nametoenv of e2, yshift=-6mm] (n2) {};
            \end{scope}

            \coordinate (xp1) at ($ (n1.south)!.50!(n1.south east) $);
            \coordinate[below=of xp1] (p1);
            \procedure{p1}{c1}{c2}
            \node[code, below=of p1] (code) {
            parameters: m\\
            body: (cond ...)
            };

            \coordinate (xp2) at ($ (n2.south)!.50!(n2.south east) $);
            \coordinate[below=of xp2] (p2);
            \procedure{p2}{c3}{c4}

            \draw[->] (c1) -- (c1 |- code.north);
            \draw[->] (c2) -| ($ (n1.south east) - (3mm, 0) $);

            \draw[->] (ge) -- (ge.west -| g.west);
            \draw[->] (e1) -- (e1.east -| n1.west);
            \draw[->] (n1.north) -- (n1.north |- g.south);
            \draw[->] (dispatch1) -| ($ (p1) + (0, \circleradius) $);

            \draw[->] (e2) -- (e2.east -| n2.west);
            \draw[->] (c3) |- (code);
            \draw[->] (c4) -| ($ (n2.south east) - (3mm, 0) $);
            \draw[->] (n2.north) -- (n2.north |- g.south);
            \draw[->] (dispatch2) -| ($ (p2) + (0, \circleradius) $);
            \draw[->] (x) -- ($ (x) + (1cm, 0) $)
                |- ($ (p1) + (-2\circleradius, 0) $);
            \draw[->] (z) -- ($ (z) + (7cm, 0) $)
                |- ($ (p2) + (-2\circleradius, 0) $);
        \end{tikzpicture}
        \caption{Environment structure after the definitions of \vscm{x} and 
        \vscm{z}. In E2, the values of \vscm{x} and \vscm{y} correspond to the 
        \vscm{x} defined in the global environment.}
        \label{3.20fig}
    \end{figure}
    Figure \ref{3.20fig} shows the environments created by the definitions of 
    \vscm{x} and \vscm{z}. When\\
    \vscm{(set-car! (cdr z) 17)} is evaluated, \vscm{(cdr z)} is evaluated 
    first. This creates an environment E3 pointing to E2, where \vscm{(z 'cdr)} 
    is evaluated, returning the \vscm{x} defined in the global environment. So 
    the expression becomes \vscm{(set-car! x 17)}, evaluated in the global 
    environment. The expression
    \vscm{((x 'set-car!) 17)} is then evaluated. The evaluation of
    \vscm{(x 'set-car!)} leads to the creation of an environment E4 pointing to 
    E1, in which the evaluation returns the \vscm{set-x!} procedure from 
    environment E1. The evaluation of the expression obtained \vscm{(set-x! 17)} 
    leads to the modification of the value of \vscm{x} in environment E1. 
    Lastly, \vscm{(car x)} is evaluated in an environment pointing to E1, so the 
    value returned is 17.
\end{exe}

\subsection{Representing Queues}

\begin{exe}[3.21]
    The elements actually contained in the queue are only the contents of the 
    queue’s \vscm{car}. The queue’s \vscm{cdr} points to the last element of the 
    queue, so it is printed twice. The rear pointer is not updated when the last 
    element from the queue is deleted, so the former last element is still 
    printed although the queue is empty.
    \scm{ch3/3.21.scm}
\end{exe}

\begin{exe}[3.22]
    The constructor, selectors and mutators can be defined in the following way. 
    The implementation of the queue operations doesn’t need to be modified.
    \scm{ch3/3.22.scm}
\end{exe}

\begin{exe}[3.23]
    To respect the requirement that all operations should be accomplished in 
    $\Theta(1)$ steps, it’s necessary to use a doubly-linked list instead of 
    a singly-linked list. The deque is represented as a pair containing 
    a pointer to the first element of a list and a pointer to the last element 
    of this list just like the queue. Each element of the list is a pair 
    containing the value and a pointer to the previous element of the list. 
    Since such a structure can’t be printed since it contains infinite loops as 
    soon as the deque contains at least 2 elements, the insertion and deletion 
    procedures return a list representation of the contents of the deque.
    \scm{ch3/3.23.scm}
\end{exe}

\subsection{Representing Tables}

\begin{exe}[3.24]
    The only necessary change is to define an \vscm{assoc} procedure that uses 
    the provided \vscm{same-key?} instead of \vscm{equal?}. The code below is 
    a possible solution for a one-dimensional table. For multi-dimensional 
    tables, there is no reason to assume that the successive keys are of the 
    same type or that the same equality test must be used at every level, so 
    multiple comparison procedures should be provided, and the right procedure 
    should be passed as an argument to \vscm{assoc} at each level of the table.
    \scm{ch3/3.24.scm}
    \scm{ch3/3.24test.scm}
\end{exe}

\begin{exe}[3.25]
    It would be possible to use the lists as keys directly but I don’t think 
    that’s the point of the exercise. The solution I implemented allows 
    different numbers of keys for different records, however it does not allow 
    keys that are prefixes of each other: if a value is stored under the key
    \vscm{'(a b c)} and another value is then stored under \vscm{'(a b)}, the 
    record for \vscm{'(a b c)} is silently deleted, and vice versa. It would 
    also be inefficient for large tables since it checks whether the record 
    found really contains a table by going through the whole record before 
    looking for the following key in it.
    \scm{ch3/3.25.scm}
\end{exe}

\begin{exe}[3.26]
    Here is an example of a one-dimensional table where the keys are ordered 
    with the given comparison procedure \vscm{<?}. The local table is stored as 
    a binary tree of records instead of a headed list. I used mutable trees 
    instead of using the \vscm{adjoin-tree} procedure for binary trees of 
    section \ref{2.3.3} to avoid stacking recursive calls and creating multiple 
    intermediate trees.
    \scm{ch3/3.26.scm}
\end{exe}

\begin{exe}[3.27]
    \begin{figure}
        \begin{tikzpicture}[>=Stealth, thick]
            \matrix[matrix anchor=base west, column sep=2\nametoenv, column 
            2/.style={anchor=base west}] at (0,0) {
            \node[text width=1cm, align=right] (ge) {global\\ env};
            & \node[code] (memoize) {memoize: ...}; \\[-8pt]
            & \node[code] (memo-fib) {memo-fib: }; \\
            };
            \coordinate (e1pos) at ($ (g.south west)!.25!(g.south east) $);
            \matrix[matrix anchor=north west, column sep=2\nametoenv,
            below=1cm of e1pos, column 2/.style={anchor=base west}] {
            \node[text width=1cm, align=right] (e1) {E1};
            & \node[code] (f) { f: ...}; \\[-11pt]
            & \node[code] (table) { table: ... }; \\
            };
            \begin{scope}[on background layer]
                \node[global env, fit=(memoize) (memo-fib),
                right=\nametoenv of ge, yshift=-1mm] (g) {};
                \node[env, minimum width=3cm, fit=(f) (table),
                right=\nametoenv of e1, yshift=1mm] (n1) {};
            \end{scope}

            \coordinate (xp1) at ($ (n1.south west)!.50!(n1.south east) $);
            \coordinate[below=of xp1] (p1);
            \procedure{p1}{c1}{c2}
            \node[code, below=of p1] (code) {
            parameters: x\\
            body: (let ...)
            };

            \draw[->] (c1) -- (c1 |- code.north);
            \draw[->] (c2) -- (c2 |- n1.south);

            \draw[->] (ge) -- (ge.west -| g.west);
            \draw[->] (e1) -- (e1.east -| n1.west);
            \draw[->] (n1.north) -- (n1.north |- g.south);

            \draw[->] (memo-fib) -- ($ (memo-fib) + (2cm, 0) $)
                |- ($ (p1) + (-2\circleradius, 0) $);

            % Evaluation of (memo-fib 3)
            \node[env, right=of n1] (n3) {n: 3};
            \node[below=of n3] {$\vdots$};
            \node[right=of n3] (e4) {E4};
            \draw[->] (e4) -- (n3);
            \draw[->] (n3) -- (n3 -| n1.east);
        \end{tikzpicture}

        \caption{Environments created during the evaluation of
        \vscm{(memo-fib 3)}.}
        \label{3.27fig}
    \end{figure}
    Figure \ref{3.27fig} shows some of the environments created during the 
    evaluation of \vscm{(memo-fib 3)}.
    The call to \vscm{memoize} creates a local environment E1 containing 
    a table, and the procedure returned by \vscm{memoize} points to this 
    environment. When \vscm{memo-fib} is called for the first time with a value, 
    it puts the computed result into the table. When it is called again with the 
    same value, it simply returns the value stored in the table instead of 
    making recursive calls. So \vscm{memo-fib} computes the $n$th Fibonacci 
    number in linear time because it never computes the value for the same 
    number twice.

    If we had defined \vscm{memo-fib} to be \vscm{(memoize fib)}, it would not 
    have worked because \vscm{fib} calls itself rather than \vscm{memo-fib} 
    recursively.
\end{exe}

\subsection{A Simulator for Digital Circuits}

\begin{exe}[3.28]
    Here is a definition of an or-gate similar to the definiton of the and-gate:
    \scm{ch3/3.28.scm}
\end{exe}

\begin{exe}[3.29]
    Using the logical equivalency between $a \lor b$ and $\lnot (\lnot a \land 
    \lnot b)$, we can build an or-gate from and-gates and inverters. The delay 
    for an or-gate built this way is the and-gate delay plus twice the inverter 
    delay.
    \scm{ch3/3.29.scm}
\end{exe}

\begin{exe}[3.30]
    The ripple-carry-adder can be defined using the following procedure:
    \scm{ch3/3.30.scm}

    Let’s use the following notations: $o = \textrm{or-delay}, 
    a = \textrm{and-delay}, i = \textrm{inverter-delay}$.
    Let’s call $R_{Ci}$ the delay to obtain $C_i$ in a ripple-carry adder, 
    $R_{Si}$ the delay to obtain $S_i$ in a ripple-carry adder, $F_s$ the delay 
    to obtain the sum bit in a full-adder, $F_c$ the delay to obtain the carry 
    bit in a full-adder, $H_s$ the delay to obtain the carry in a half-adder, 
    $H_c$ the delay to obtain the carry in a half-adder.

    For the half-adder we have $H_c = a$ and
    $H_s = \max(2a + i, o + a)$.

    For the full-adder, we have $F_s = 2 H_s = 2 \max(2a + i, o + a)$, and $F_c 
    = H_s + H_c + o = a + o + \max(2a + i, o + a)$, from where
    $F_s \geq F_c$.

    For the full-carry adder, we have:
    $R_{Ci} = (n - i) \times F_c$ and $R_{Si} = R_{Ci} + F_s$ because only the 
    carry is transmitted to the following full-adders.

    The delay to obtain the complete output from an $n$-bit ripple-carry adder 
    is $R_{C1} + \max(F_s, F_c) = R_{C1} + F_s = (n - 1) F_c + F_s$. In terms of 
    the delays for and-gates, or-gates and inverters, the delay for the 
    ripple-carry adder is
    $ (n - 1) (a + o) + (n + 1) \max(2a + i, o + a) $.
\end{exe}

\begin{exe}[3.31]
    The initialization is necessary to compute correctly all the signals with 
    the initial values of the inputs. Since there are inverters, not all signals 
    are 0 even if all inputs and outputs are 0, and if they are not initialized 
    properly, the values computed after changing the inputs could be wrong as 
    well.

    In the case of the half-adder example, without the initialization, the 
    output of the inverter would have a signal of 0 instead of 1 initially. 
    Since setting the first input to 1 does not trigger a change to the 
    inverter’s input, the sum would remain 0 after the propagation. After 
    setting the second input to 1, the sum would still remain 0 and the carry 
    would become 1.
\end{exe}

\begin{exe}[3.32]
    If the (last in, first out) order is used, the values set by the actions 
    executed last in a segment could set incorrect values because they are not 
    taking into account all the changes that occurred in that segment.

    If the inputs $a_1$ and $a_2$ are $0$ and $1$ are both changed in the same 
    segment, if the action triggered by the change of $a_2$ is executed first, 
    an action setting the output value to $0$ is scheduled, than when the change 
    to $a_1$ is taken into account, the same action is scheduled again at the 
    same time, so the order of execution of these two actions does not matter.

    However, if the change to $a_1$ is treated first, an action setting the 
    output signal to $1$ is scheduled, and when $a_2$ switches to $0$ an action 
    setting the output signal to $0$ is scheduled at the same time. If these 
    actions are executed (last in, first out), the final output value will be 
    $1$ instead of $0$ because the scheduled action executed last used stale 
    values for some signals.
\end{exe}

\subsection{Propagation of Constraints}

\begin{exe}[3.33]
    The averager can be defined in the following way, since we want 
    $ a + b = 2c$:
    \scm{ch3/3.33.scm}
\end{exe}

\begin{exe}[3.34]
    The squarer defined in this way works only in one direction because the 
    multiplier needs two of its three connectors to have a value to be able to 
    set the third connector’s value. If the value of \vscm{a} is set, the value 
    of \vscm{b} will be set correctly, but if the value of \vscm{b} is set, the 
    value of \vscm{a} won’t be set because only one connector has a value.
\end{exe}

\begin{exe}[3.35]
    The squarer can be defined in the following way:
    \scm{ch3/3.35.scm}
\end{exe}

\begin{exe}[3.36]
    \begin{figure}
        \hspace{-1cm}
        \begin{tikzpicture}[>=Stealth, thick]
            % Global environment
            \matrix[matrix anchor=base west, column sep=2\nametoenv, column 
            2/.style={anchor=base west}] at (0,0) {
                & \node[code] (fee) {for-each-except: ...}; \\[-8pt]
                \node[text width=1cm, align=right] (ge) {global\\ env};
                & \node[code] (mc) {make-connector: ...}; \\[-9pt]
                & \node[code] (sv) {set-value!: ...}; \\
                & \node[code] {inform-about-value: ...}; \\
                & \node[code] (b) {b:}; \\
                & \node[code] (a) {a:}; \\
            };

            % E1
            \coordinate (e1pos) at ($ (g.south west)!.05!(g.south east) $);
            \matrix[column sep=2\nametoenv, below=of e1pos] (env1m) {
                \node[text width=1cm, align=right] (env1n) {E1};
                & \node[env, minimum width=3cm, minimum height=1em] (env1) {}; 
            \\
            };

            % E2
            \matrix[column sep=2\nametoenv, below=of env1m, column 
                2/.style={anchor=base west}] (env2m) {
                \node[text width=1cm, align=right] (env2n) {E2};
                & \node[code] (value1) { value: 10}; \\[-11pt]
                & \node[code] { informant: 'user}; \\
                & \node[code] { constraints: '() }; \\
                & \node[code] { set-my-value: ... }; \\
                & \node[code] { forget-my-value: ... }; \\
                & \node[code] { connect: ... }; \\
                & \node[code] (me1) { me: }; \\
            };

            % E3
            \coordinate (e3pos) at ($ (g.south west)!.85!(g.south east) $);
            \matrix[column sep=2\nametoenv, below=of e3pos] (env3m) {
                \node[text width=1cm, align=right] (env3n) {E3};
                & \node[env, minimum width=3cm, minimum height=1em] (env3) {}; 
            \\
            };

            % E4
            \matrix[column sep=2\nametoenv, below=of env3m, column 
            2/.style={anchor=base west}] {
                \node[text width=1cm, align=right] (env4n) {E4};
                & \node[code] (value2) { value: false}; \\[-11pt]
                & \node[code] { informant: false}; \\
                & \node[code] { constraints: '() }; \\
                & \node[code] { set-my-value: ... }; \\
                & \node[code] { forget-my-value: ... }; \\
                & \node[code] { connect: ... }; \\
                & \node[code] (me2) { me: }; \\
            };

            % E5 (set-value! a 10 'user)
            \coordinate (e5pos) at ($ (g.north west)!.05!(g.north east) $);
            \matrix[column sep=2\nametoenv, above=6.5em of e5pos, column 
                2/.style={anchor=west}] (env5m) {
                & \node[code] (conn5) {connector: a}; \\[-9pt]
                \node (env5n) {E5};
                & \node[code] {new-value: 10}; \\
                & \node[code] (inf5) {informant: 'user}; \\
            };

            % E6 (a 'set-value)
            \node[code, right=3cm of value1, env] (env6) {request: 'set-value};
            \node[above=of env6] (env6n) {E6};

            % E7 (set-my-value 10 'user)
            \node[below=1em of env6] (env7n) {E7};
            \matrix[below=of env7n, nodes={anchor=west}] {
                \node[code] (nv7) {newval: 10}; \\
                \node[code] (setter7) {setter: 'user}; \\
            };

            % E8 (for-each-except ...)
            \matrix[column sep=2\nametoenv, yshift=1em, right=of env5m, column 
                2/.style={anchor=west}] (env8m) {
                & \node[code] (ex8) {exception: 'user}; \\
                \node[align=right] (env8n) {E8};
                & \node[code] (proc8) {procedure: inform-about-value}; \\
                & \node[code] (list8) {list: '()}; \\
                & \node[code] (loop8) {loop: ...}; \\
            };

            % Backgrounds (global environment, E2, E4, E5, E7, E8)
            \begin{scope}[on background layer]
                \node[global env, fit=(fee) (a), right=\nametoenv of ge, 
                yshift=-1.5em] (g) {};
                \node[env, minimum width=4cm, fit=(value1) (me1), below=of env1, 
                yshift=-1mm] (env2) {};
                \node[env, minimum width=4cm, fit=(value2) (me2), below=of env3, 
                yshift=-1mm] (env4) {};
                \node[env, fit=(conn5) (inf5)] (env5) {};
                \node[env, fit=(nv7) (setter7)] (env7) {};
                \node[env, fit=(ex8) (proc8) (list8) (loop8)] (env8) {};
            \end{scope}

            % E9 (loop ...)
            \node[env, code, above=of env8] (env9) {list: list}; \\
            \node[left=of env9] (env9n) {E9};
            \draw[->] (env9n) -- (env9);
            \draw[->] (env9) -- (env8);

            \coordinate[below=of env2] (p1);
            \procedure{p1}{c1}{c2}
            \node[code, below=of p1] (code) {
            parameters: request\\
            body: (cond ...)
            };

            \coordinate[below=of env4] (p2);
            \procedure{p2}{c3}{c4}

            \draw[->] (c1) -- (c1 |- code.north);
            \draw[->] (c2) -| ($ (env2.south east) - (1cm, 0) $);

            \draw[->] (ge) -- (ge.west -| g.west);
            \draw[->] (env1n) -- (env1n.east -| env1.west);
            \draw[->] (env2n) -- (env2n.east -| env2.west);
            \draw[->] (env3n) -- (env3n.east -| env3.west);
            \draw[->] (env4n) -- (env4n.east -| env4.west);
            \draw[->] (env1.north) -- (env1.north |- g.south);
            \draw[->] (me1) -| ($ (p1) + (0, \circleradius) $);
            \draw[->] (env2) -- (env1);

            \draw[->] (env5n) -- (env5n.east -| env5.west);
            \draw[->] (env5) -- (env5.south |- g.north);

            \draw[->] (env6n) -- (env6n |- env6.north);
            \draw[->] (env6) -- (env6 -| env2.east);
            \draw[->] (env7n) -- (env6n |- env7.north);
            \draw[->] (env7) -- (env7 -| env2.east);

            \draw[->] (env8n) -- (env8n -| env8.west);
            \draw[->] (env8) -- (env8 |- g.north);

            \draw[->] (c3) |- (code);
            \draw[->] (c4) -| ($ (env4.south east) - (1cm, 0) $);
            \draw[->] (env3.north) -- (env3.north |- g.south);
            \draw[->] (env4) -- (env3);
            \draw[->] (me2) -| ($ (p2) + (0, \circleradius) $);
            \draw[->] (a) -- ($ (a) - (2.3cm, 0) $)
                |- ($ (p1) + (-2\circleradius, 0) $);
            \draw[->] (b) -- ($ (b) + (8cm, 0) $)
                |- ($ (p2) + (-2\circleradius, 0) $);
        \end{tikzpicture}
        \caption{Environment structure in which the expresion
        \vscm{(for-each-except setter inform-about-value constraints)} is 
        evaluated.}
        \label{3.36fig}
    \end{figure}

    Figure \ref{3.36fig} shows the environment structure in which the expression 
    is evaluated. The environments created when defining \vscm{a} and \vscm{b} 
    are similar to those created in exercise \ref{3.10}: an environment E1 
    (resp. E3) is created for the evaluation of \vscm{(make-connector)}, then 
    since \vscm{(make-connector)} contains a \vscm{let}, a new environment E2 
    (resp. E4) pointing to E1 (resp. E3) is created. The variables \vscm{value}, 
    \vscm{informant}, \vscm{constraints} and the local procedures 
    \vscm{set-my-value}, \vscm{forget-my-value}, \vscm{connect} and \vscm{me} 
    are defined in E2 (resp. E4). The local procedure \vscm{me} pointing to E2 
    (resp. E4) is returned and bound to \vscm{a} (resp. \vscm{b}).

    The following environments are created during the evaluation of 
    \vscm{(set-value! a 10 'user)}:
    \begin{itemize}
        \item E5: for the evaluation of \vscm{(set-value! a 10 'user)}. It 
            points to the global environment since \vscm{set-value!} is defined 
            there.
        \item E6: for the evaluation of \vscm{(a 'set-value!)}. It points to E2 
            since \vscm{a} is a procedure pointing to E2. The evaluation returns 
            the \vscm{set-my-value} procedure from E2.
        \item E7: for the evaluation of \vscm{(set-my-value 10 'user)}. It 
            points to E2 since \vscm{set-my-value} is the procedure returned at 
            the previous step.
        \item E8: After \vscm{(has-value? me)} is evaluated (environments 
            omitted) the values of \vscm{value} and \vscm{informant} are changed 
            in \vscm{set-my-value}, \vscm{for-each-except} is called, which 
            leads to the creation of environment E8 pointing to the global 
            environment since that’s where \vscm{for-each-except} was defined. 
            The \vscm{loop} procedure is defined in E8, so the evaluation of 
            \vscm{(loop list)} leads to the creation of E9 pointing to E8.

            There are no constraints here so no further environments are 
            created. Otherwise, for each constraint, an environment pointing to 
            the global environment would be created, in which 
            \vscm{(inform-about-value construint)} would be evaluated. Then 
            \newline
            \vscm{(constraint 'I-have-a-value)} would be evaluated in an 
            environment pointing to the constraint’s local environment.
    \end{itemize}
\end{exe}

\begin{exe}[3.37]
    The procedures can be defined in the following way:
    \scm{ch3/3.37.scm}
\end{exe}

\section{Concurrency: Time Is of the Essence}

\subsection{The Nature of Time in Concurrent Systems}

\begin{exe}[3.38]
    \ \vspace{-20pt}
    \begin{itemize}
        \item[a.] If no interleaving is possible, the possible final values are 
            \$35, \$40, \$45 and \$50:
            \begin{itemize}
                \item Peter, Paul, Mary: \$45;
                \item Paul, Peter, Mary: \$45;
                \item Mary, Peter, Paul: \$40;
                \item Mary, Paul, Peter: \$40;
                \item Peter, Mary, Paul: \$35;
                \item Paul, Mary, Peter: \$50.
            \end{itemize}
        \item[b.] Some of the possible other values are \$110, \$80, \$55, \$90. 
            Figures \ref{3.38.1} and \ref{3.38.2} show timing diagrams 
            explaining how the value \$110 and \$90 can occur.
    \end{itemize}

    \begin{figure}
        \small
        \centering
        \begin{tikzpicture}[>=Stealth]
            \matrix[time matrix] {
                \node (Peter) {Peter}; &
                \node (Mary) {Mary}; &
                \node (Bank) {Bank}; &
                \node (Paul) {Paul}; \\

                && \node[bank] (b100) {\$100}; & \\
                \node[action] (peacc) {Access balance: \$100}; &&& \\
                & \node[action] (macc) {Access balance: \$100}; && \\
                &&& \node[action] (paacc) {Access balance: \$100}; \\
                \node[action] (penv) {new value: $100 + 10 = 110$}; &&& \\
                & \node[action] (mnv) {new value: $100 / 2 = 50$}; && \\
                &&& \node[action] (panv) {new value: $100 - 20 = 80$}; \\
                & \node[action] (mset) {set! balance to \$50}; && \\
                && \node[bank] (b50) {\$50}; &\\
                &&& \node[action] (paset) {set! balance to \$80}; \\
                && \node[bank] (b80) {\$80}; &\\
                \node[action] (peset) {set! balance to \$110}; &&& \\
                && \node[bank] (b110) {\$110}; &\\
            };

            \draw[arrow] (b100) -| (peacc);
            \draw[arrow] (peacc) -- (penv);
            \draw[arrow] (penv) -- (peset);
            \draw[arrow] (peset) |- (b110);

            \draw[arrow] (b100) -| (macc);
            \draw[arrow] (macc) -- (mnv);
            \draw[arrow] (mnv) -- (mset);
            \draw[arrow] (mset) |- (b50);

            \draw[arrow] (b100) -| (paacc);
            \draw[arrow] (paacc) -- (panv);
            \draw[arrow] (panv) -- (paset);
            \draw[arrow] (paset) |- (b80);

            \draw[<-, line width=.8mm] (current bounding box.south west) 
            --node[sloped,above]{time} (current bounding box.north west);
        \end{tikzpicture}
        \caption{A timing diagram showing how the final value can be \$110.}
        \label{3.38.1}
    \end{figure}

    \begin{figure}
        \small
        \centering
        \begin{tikzpicture}[>=Stealth]
            \matrix[time matrix] {
                \node (Peter) {Peter}; &
                \node (Mary) {Mary}; &
                \node (Bank) {Bank}; &
                \node (Paul) {Paul}; \\

                && \node[bank] (b100) {\$100}; & \\
                \node[action] (peacc) {Access balance: \$100}; &&& \\[+1.5em]
                \node[action] (penv) {new value: $100 + 10 = 110$}; &&& 
                \\[+1.5em]
                \node[action] (peset) {set! balance to \$110}; &&& \\
                && \node[bank] (b110) {\$110}; &\\
                & \node[action] (macc) {Access balance: \$110}; && \\[+1.5em]
                & \node[action] (mnv) {new value: $110 / 2 = 55$}; && \\
                &&& \node[action] (paacc) {Access balance: \$110}; \\[+1.5em]
                &&& \node[action] (panv) {new value: $110 - 20 = 90$}; \\
                & \node[action] (mset) {set! balance to \$55}; && \\
                && \node[bank] (b55) {\$55}; &\\
                &&& \node[action] (paset) {set! balance to \$90}; \\
                && \node[bank] (b90) {\$90}; &\\
            };

            \draw[arrow] (b100) -| (peacc);
            \draw[arrow] (peacc) -- (penv);
            \draw[arrow] (penv) -- (peset);
            \draw[arrow] (peset) |- (b110);

            \draw[arrow] (b100) -| (macc);
            \draw[arrow] (macc) -- (mnv);
            \draw[arrow] (mnv) -- (mset);
            \draw[arrow] (mset) |- (b55);

            \draw[arrow] (b100) -| (paacc);
            \draw[arrow] (paacc) -- (panv);
            \draw[arrow] (panv) -- (paset);
            \draw[arrow] (paset) |- (b90);

            \draw[<-, line width=.8mm] (current bounding box.south west) 
            --node[sloped,above]{time} (current bounding box.north west);
        \end{tikzpicture}
        \caption{A timing diagram showing how the final value can be \$90.}
        \label{3.38.2}
    \end{figure}
\end{exe}

\subsection{Mechanisms for Controlling Concurrency}

\begin{exe}[3.39]
    The remaining possibilities are 101, 121 and 100. The value of \vscm{x} 
    can’t change between the two times that $P_1$ accesses it to evaluate
    \vscm{(* x x)} so 110 is not possible anymore. The value of \vscm{x} can’t 
    change during the execution of $P_2$ so 11 is not possible anymore. The 
    final value can be 100 if $P_1$ accesses \vscm{x} and computes the final 
    value as 100 but $P_2$ accesses it and sets it to 11 before $P_1$ can set it 
    to 100.
\end{exe}

\begin{exe}[3.40]
    The possible values are:
    \begin{itemize}
        \item $100$: $P_1$ accesses \vscm{x} twice and $P_2$ accesses \vscm{x} 
            thrice, then \vscm{x} $P_2$ sets \vscm{x} to 1000 before $P_1$ sets 
            it to 100.
        \item $1000$: $P_1$ accesses \vscm{x} twice and $P_2$ accesses \vscm{x} 
            thrice, then \vscm{x} $P_1$ sets \vscm{x} to 100 before $P_2$ sets 
            it to 1000.
        \item $10\,000$: either $P_1$ accesses \vscm{x} once before $P_2$ sets 
            it to 1000, or $P_2$ accesses \vscm{x} twice before $P_1$ sets it to 
            100.
        \item $100\,000$: $P_2$ accesses \vscm{x} once before $P_1$ sets it to 
        100.
        \item $1\,000\,000$: the two procedures execute sequentially in any 
    order.
    \end{itemize}

    If the procedures are serialized the only possible value is $100\,000$.
\end{exe}

\begin{exe}[3.41]
    I don’t think serializing access to balance is necessary because both 
    \vscm{withdraw} and \vscm{deposit} make a single assignment to 
    \vscm{balance}, so accessing \vscm{balance} concurrently with one of these 
    procedures will reflect the state of the account either before or after the 
    withdrawal or deposit, but it will correspond to a real state of the 
    account.
\end{exe}

\begin{exe}[3.42]
    The change proposed by Ben Bitdiddle is safe to make. The two versions of 
    \vscm{make-account} allow the same concurrency.
\end{exe}

\begin{exe}[3.43]
    \begin{figure}
        \centering
        \begin{tikzpicture}[>=Stealth]
            \matrix[time matrix] {
                &
                \node (a1) {a1}; &
                \node (a2) {a2}; &
                \node (a3) {a3}; & \\

                & \node[bank] (a130) {\$30}; &
                \node[bank] (a220) {\$20}; &
                \node[bank] (a310) {\$10}; & \\
                \node[action] (p1aa1) {Access balance a1: \$30}; &&& \\
                &&&& \node[action] (p2aa2) {Access balance a2: \$20}; \\
                \node[action] (p1aa2) {Access balance a2: \$20}; &&& \\
                &&&& \node[action] (p2aa3) {Access balance a3: \$10}; \\
                \node[action] (p1diff) {Difference: \$10}; &&& \\
                &&&& \node[action] (p2diff) {Difference: \$10}; \\
                \node[action] (p1w) {\vscm{((a1 'withdraw) 10)}}; &&& \\
                & \node[bank] (a120) {\$20}; &&& \\
                &&&& \node[action] (p2w) {\vscm{((a2 'withdraw) 10)}}; \\
                && \node[bank] (a210) {\$10}; && \\
                \node[action] (p1d) {\vscm{((a2 'deposit) 10)}}; &&& \\
                && \node[bank] (a220b) {\$20}; && \\
                &&&& \node[action] (p2d) {\vscm{((a3 'deposit) 10)}}; \\
                &&& \node[bank] (a320) {\$20}; & \\
            };

            \draw[arrow] (a130) -| (p1aa1);
            \draw[arrow] (a220) |- (p1aa2);
            \draw[arrow] (p1aa1) -- (p1aa2);
            \draw[arrow] (p1aa2) -- (p1diff);
            \draw[arrow] (p1diff) -- (p1w);
            \draw[arrow] (p1w) -- (p1d);
            \draw[arrow] (p1w) |- (a120);
            \draw[arrow] (p1d) |- (a220b);

            \draw[arrow] (a220) |- (p2aa2);
            \draw[arrow] (a310) |- (p2aa3);
            \draw[arrow] (p2aa2) -- (p2aa3);
            \draw[arrow] (p2aa3) -- (p2diff);
            \draw[arrow] (p2diff) -- (p2w);
            \draw[arrow] (p2w) -- (p2d);
            \draw[arrow] (p2w) |- (a210);
            \draw[arrow] (p2d) |- (a320);

            \draw[<-, line width=.8mm] (current bounding box.south west) 
            --node[sloped,above]{time} (current bounding box.north west);
        \end{tikzpicture}
        \caption{A timing diagram showing how the account balances can all three 
        be \$20 after exchanging concurrently \vscm{a1} and \vscm{a2} on one 
        side, \vscm{a2} and \vscm{a3} on the other side.}
        \label{3.43.1}
    \end{figure}

    \begin{figure}
        \centering
        \begin{tikzpicture}[>=Stealth]
            \matrix[time matrix] {
                &
                \node (a1) {a1}; &
                \node (a2) {a2}; &
                \node (a3) {a3}; & \\

                & \node[bank] (a130) {\$30}; &
                \node[bank] (a220) {\$20}; &
                \node[bank] (a310) {\$10}; & \\
                \node[action] (p1aa1) {Access balance a1: \$30}; &&& \\
                &&&& \node[action] (p2aa2) {Access balance a2: \$20}; \\
                \node[action] (p1aa2) {Access balance a2: \$20}; &&& \\
                &&&& \node[action] (p2aa3) {Access balance a3: \$10}; \\
                \node[action] (p1diff) {Difference: \$10}; &&& \\
                &&&& \node[action] (p2diff) {Difference: \$10}; \\
                \node[action] (p1w) {\vscm{((a1 'withdraw) 10)}}; &&& \\
                & \node[bank] (a120) {\$20}; &&& \\
                &&&& \node[action] (p2w1) {Access balance a2: \$20}; \\[+1.2em]
                &&&& \node[action] (p2w2) {New value for a2: \$10}; \\
                \node[action] (p1d) {\vscm{((a2 'deposit) 10)}}; &&& \\
                && \node[bank] (a220b) {\$20}; && \\
                &&&& \node[action] (p2w3) {set! balance for a2 to \$10}; \\
                && \node[bank] (a210) {\$10}; && \\
                &&&& \node[action] (p2d) {\vscm{((a3 'deposit) 10)}}; \\
                &&& \node[bank] (a320) {\$20}; & \\
            };

            \draw[arrow] (a130) -| (p1aa1);
            \draw[arrow] (a220) |- (p1aa2);
            \draw[arrow] (p1aa1) -- (p1aa2);
            \draw[arrow] (p1aa2) -- (p1diff);
            \draw[arrow] (p1diff) -- (p1w);
            \draw[arrow] (p1w) -- (p1d);
            \draw[arrow] (p1w) |- (a120);
            \draw[arrow] (p1d) |- (a220b);

            \draw[arrow] (a220) |- (p2aa2);
            \draw[arrow] (a220) |- (p2w1);
            \draw[arrow] (a310) |- (p2aa3);
            \draw[arrow] (p2aa2) -- (p2aa3);
            \draw[arrow] (p2aa3) -- (p2diff);
            \draw[arrow] (p2diff) -- (p2w1);
            \draw[arrow] (p2w1) -- (p2w2);
            \draw[arrow] (p2w2) -- (p2w3);
            \draw[arrow] (p2w3) -- (p2d);
            \draw[arrow] (p2w3) |- (a210);
            \draw[arrow] (p2d) |- (a320);

            \draw[<-, line width=.8mm] (current bounding box.south west) 
            --node[sloped,above]{time} (current bounding box.north west);
        \end{tikzpicture}
        \caption{A timing diagram showing how the sum of the balances in the 
        accounts is not preserved if the transactions on individual accounts are 
        not serialized.}
        \label{3.43.2}
    \end{figure}

    Since each exchange run individually exchanges the balances of two of the 
    accounts, if any number of exchanges happen sequentially the balances of the 
    accounts will still be \$10, \$20 and \$30 in some order.

    Figure \ref{3.43.1} shows how the account balances can be different than 
    \$10, \$20 and \$30 with the first version of the account-exchange program. 
    The \vscm{withdraw} and \vscm{deposit} procedures from each account are 
    serialized, so each of them is an atomic operation. Each exchange operation 
    removes an amount from an account and adds the same amount to another 
    account, and these operations are serialized, so the sum of the balances is 
    preserved.

    Figure \ref{3.43.2} shows a case where the sum of the balances of the 
    accounts is not preserved after exchanges between \vscm{a1} and \vscm{a2} on 
    one side, \vscm{a2} and \vscm{a3} on the other side. Compared to the version 
    from figure \ref{3.43.1}, the only transaction for a single account that 
    does not happen as if it were serialized is \vscm{((a2 'withdraw) 10)}: the 
    first exchange procedure sets the account’s balance to \$20 before the 
    second procedure can set it to \$10, and the final sum is \$50 instead of 
    \$60.
\end{exe}

\begin{exe}[3.44]
    Louis is wrong, there is no problem with Ben Bitdiddle’s procedure: at the 
    end of the transfer, \vscm{amount} has been withdrawn from 
    \vscm{from-account} and deposited on \vscm{to-account}, it does not matter 
    whether other procedures access the accounts in-between. The fundamental 
    difference with the exchange problem is that for the transfer the amount is 
    an argument to the procedure, so there are only two atomic operations, one 
    on each account. In the case of exchange there are two atomic operations for 
    each account: an access to the balance and then either a deposit or 
    a withdrawal.
\end{exe}

\begin{exe}[3.45]
    The problem with Louis’ reasoning is that when a procedure serialized with 
    an account’s serializer attempts to withdraw or deposit to that account, it 
    calls a procedure serialized with the same serializer and it gets stuck 
    forever waiting for the mutex it holds to be available. In the case of 
    \vscm{serialized-exchange}, it’ll wait forever while trying to execute \\
    \vscm{((account1 'withdraw) difference)}.
\end{exe}

\begin{exe}[3.46]
    Figure \ref{3.46fig} shows how the implementation of \vscm{test-and-set!} 
    can fail if two procedures access the cell and find its contents at false 
    before both set its contents to true.
    \begin{figure}
        \centering
        \begin{tikzpicture}[>=Stealth]
            \matrix[time matrix] {
                & \node (cell) {mutex cell}; & \\

                & \node[bank] (cfalse) {false}; & \\
                \node[action] (p1acc) {Access cell: false}; && \\
                && \node[action] (p2acc) {Access cell: false}; \\
                \node[action] (p1set) {set! cell to true}; && \\
                & \node[bank] (ctrue1) {true}; & \\
                && \node[action] (p2set) {set! cell to true}; \\
                & \node[bank] (ctrue2) {true}; & \\
            };

            \draw[arrow] (cfalse) -| (p1acc);
            \draw[arrow] (p1acc) -- (p1set);
            \draw[arrow] (p1set) |- (ctrue1);

            \draw[arrow] (cfalse) -| (p2acc);
            \draw[arrow] (p2acc) -- (p2set);
            \draw[arrow] (p2set) |- (ctrue2);

            \draw[<-, line width=.8mm] (current bounding box.south west) 
            --node[sloped,above]{time} (current bounding box.north west);
        \end{tikzpicture}
        \caption{A timing diagram showing how the mutex implementation can allow 
        to processes to acquire the mutex at the same time.}
        \label{3.46fig}
    \end{figure}
\end{exe}

\begin{exe}[3.47]
    \ \vspace{-20pt}
    \begin{itemize}
        \item[a.] The following implementation of semaphores uses a mutex to 
            guard a variable holding the number of available accesses to the 
            semaphore:
            \scm{ch3/3.47a.scm}
        \item[b.] The following implementation uses a cell holding the number of 
            available accesses, and an atomic \vscm{test-and-set!} operation to 
            change the value of that cell:
            \scm{ch3/3.47b.scm}
    \end{itemize}
\end{exe}

\begin{exe}[3.48]
    The \vscm{serialized-exchange} and \vscm{make-account-and-serializer} 
    procedures can be modified in the following way to incorporate the numbering 
    technique. It works because if two procedures need to access the same two 
    accounts, they will both attempt to access the same account first and one of 
    them will wait until the other one is done with no deadlock. More generally, 
    if one procedure P1 already holds a lock L1 and waits to acquire a lock L2 
    held by another procedure P2, we know that P2 won’t attempt to acquire L1, 
    otherwise it would have done so before acquiring L2, so the lock held by P1 
    can’t prevent P2 from completing.
    \scm{ch3/3.48.scm}
\end{exe}

\begin{exe}[3.49]
    If the resources to lock are not all known in advance, it’s of course 
    impossible to order the locks. For instance, let’s assume that we must 
    change a field in a row of a table in a database, and then find rows in 
    other tables that contain the new value of the field to update these rows. 
    (Not sure of a concrete example, but it’s the idea…)
\end{exe}

\section{Streams}

\begin{comp}
    In order to make the code in this section work with Gambit scheme, it’s 
    necessary to define \vscm{the-empty-stream} and \vscm{stream-null?}, as well 
    as the macros \vscm{delay} and \vscm{cons-stream}. This can be done with the 
    following:
    \scm{ch3/3.50pre.gambc.scm}
\end{comp}

\subsection{Streams Are Delayed Lists}

\begin{exe}[3.50]
    The generalized version of \vscm{stream-map} is:
    \scm{ch3/3.50.scm}
\end{exe}

\begin{exe}[3.51]
    After the first expression, the interpreter prints \vscm{0}. This is the 
    result of applying \vscm{show} to the first element of the stream in 
    \vscm{stream-map}, in \vscm{(cons-stream (proc (stream-car s)) ...)}.

    After the second expression, it prints \vscm{1 2 3 4 5 5} (one value per 
    line). Each call to \vscm{stream-cdr} on the result of \vscm{stream-map} 
    causes a call to \vscm{show} as the following value is evaluated. Then the 
    value returned by \vscm{stream-ref} is returned.

    After the third expression, it prints \vscm{6 7 7} (one value per line). 
    Since \vscm{delay} uses \vscm{memo-proc}, \vscm{show} is only called for the 
    elements of the stream not visited during the previous evaluation. Then the 
    value \vscm{7} is returned.
\end{exe}

\begin{exe}[3.52]
    Since \vscm{delay} uses \vscm{memo-proc}, \vscm{accum} is called only once 
    for each element of the enumerated interval, so the elements of \vscm{seq} 
    are the sums of successive integers: $1, 3, 6, 10, 15, ...$. Furthermore, 
    \vscm{sum} is equal to the sum of the integers from 1 to $n + 1$, where $n$ 
    is the maximum of the indices of the elements already visited in the stream 
    \vscm{seq}.

    After the definition of \vscm{seq}, the value of \vscm{sum} is 1 because 
    \vscm{accum} was applied to the first element of the enumeration by 
    \vscm{stream-map} ($n = 0$).

    The definition of \vscm{y} causes \vscm{stream-cdr} to be called on 
    \vscm{seq} until the first even element is found. This happens for the third 
    element of the stream interval, and then the value of \vscm{sum} is 6 ($n 
    = 2$).

    The definition of \vscm{y} causes \vscm{stream-cdr} to be called on 
    \vscm{seq} until the first multiple of 5 is found. This is the fourth 
    element of the stream, which is 10 ($n = 3$).

    The eighth even element of \vscm{seq} is 136, so after evaluationg
    \vscm{(stream-ref y 7)}, \vscm{sum} will be equal to 136, which is also the 
    printed response to the evaluation.

    The evaluation of \vscm{display-stream} causes the stream \vscm{seq} to be 
    fully evaluated, so at the end the value of \vscm{sum} is the last value of 
    that stream: 210. The evaluation of \vscm{(display-stream z)} displays the 
    elements of \vscm{seq} that are multiples of 5, followed by \vscm{done}, one 
    element per line:
    \vscm{10 15 45 55 105 120 190 210 done}.

    \vspace{\baselineskip}

    The responses would differ if \vscm{delay} were implemented without using 
    \vscm{memo-proc}. Since the result of \vscm{accum} depends on the value of 
    a variable that changes with every call to \vscm{accum}, the stream 
    \vscm{seq} would be completely different each time it is used. It becomes 
    a lot harder to reason about what \vscm{seq}, \vscm{sum}, \vscm{y}, \vscm{z} 
    represent since they don’t correspond to something clearly definable.
\end{exe}

\subsection{Infinite Streams}

\begin{exe}[3.53]
    The elements of this stream are the powers of 2.
\end{exe}

\begin{exe}[3.54]
    The procedure \vscm{mul-streams} and the stream \vscm{factorials} can be 
    defined as:
    \scm{ch3/3.54.scm}
    We use \vscm{(stream-cdr integers)} rather than \vscm{integers}, otherwise 
    the $n$th element would be $n!$ and not $(n + 1)!$.
\end{exe}

\begin{exe}[3.55]
    The procedure \vscm{partial-sums} can be defined as:
    \scm{ch3/3.55.scm}
\end{exe}

\begin{exe}[3.56]
    The required stream can be constructed as:
    \scm{ch3/3.56.scm}
\end{exe}

\begin{exe}[3.57]
    Only $n - 1$ additions are performed to compute the $n$th Fibonacci number 
    since the values already computed are not computed again.

    If \vscm{memo-proc} is not used, the values are recomputed every time they 
    are needed, so the number of additions is exponential.
\end{exe}

\begin{exe}[3.58]
    The given stream computes the expansion of the ratio of \vscm{num} and 
    \vscm{den} in basis \vscm{radix}. More precisely, if we call $n$, $d$ and 
    $b$ the values of \vscm{num}, \vscm{den} and \vscm{radix}, $q_i$ the $i$th 
    element of the stream, and $n_i$ the value of \vscm{num} during the $i$th 
    recursive call to \vscm{expand} (with $n = n_0$), we can prove by induction 
    that for any $k \geq 0$:
    \[
    n = \sum_{i = 0}^{k} \frac{q_i d}{b^{i + 1}} + \frac{n_{k + 1}}{b^{k + 1}}
    \]
    so by taking the limit when $k$ tends to $+\infty$:
    \[
	\frac{n}{d} = \sum_{i \geq 0} \frac{q_i}{b^{i + 1}}
    \]
    The streams \vscm{(expand 1 7 10)} and \vscm{(expand 3 8 10)} give the 
    decimal expansions of $\sfrac{1}{7}$ and $\sfrac{3}{8}$ respectively: in the 
    first case the elements are 1, 4, 2, 8, 5 and 7 repeating indefinitely, in 
    the second case the elements are 3, 7 and 5 followed by zeroes.
\end{exe}

\begin{exe}[3.59]
    \ \vspace{-20pt}
    \begin{itemize}
	\item[a.] The procedure \vscm{integrate-series} can be written:
	    \scm{ch3/3.59a.scm}
	\item[b.] The series for sine and cosine can be defined with:
	    \scm{ch3/3.59b.scm}
    \end{itemize}
\end{exe}

\chapter{Metalinguistic Abstraction}

\section{The Metacircular Evaluator}

\subsection{The Core of the Evaluator}

\begin{exe}[4.1]
    We can put the value to evaluate first inside a \vscm{let} expression. Since 
    the \vscm{let} is transformed into a \vscm{lambda} internally, we know that 
    its argument will be evaluated before the body of the \vscm{let}.
    \scm{ch4/4.01.scm}
\end{exe}

\subsection{Representing Expressions}

\begin{exe}[4.2]
    \ \vspace{-20pt}
    \begin{enumerate}
        \item The test to determine whether an expression is a procedure just 
            checks whether it is a pair, so it will return true for all list 
            expressions: \vscm{if}, \vscm{cond}, \vscm{begin}, \vscm{define}, 
            \vscm{set!}, etc., and the evaluator will try to apply the procedure 
            \vscm{if}, \vscm{cond}, etc. to the rest of the arguments. It’s not 
            possible to transform the special forms into procedures because all 
            the arguments of a procedure are evaluated before evaluation.
        \item The only required change to \vscm{eval} is to put the 
            \vscm{application?} test first. Then we only need changing the 
            definition of the \vscm{application?} predicate and the associated 
            selectors so they reflect the new syntax.
	    \scm{ch4/4.02.scm}
    \end{enumerate}
\end{exe}

\begin{exe}[4.3]
    As in exercise~\ref{2.73}, a few cases can’t be assimilated to the 
    data-directed dispatch because they correspond to untagged data: here it’s 
    the case of \vscm{variable?}, \vscm{self-evaluating?} and 
    \vscm{application?}. If we had kept Louis’ idea from the previous exercise 
    to start procedure applications with \vscm{call}, \vscm{application?} could 
    have been assimilated into the dispatch. Since we don’t keep this idea, we 
    have to redefine the \vscm{application?} predicate: if we keep it as 
    \vscm{pair?} we can’t put it before the dispatch because all the special 
    forms will be considered as procedure applications, and we can’t put it 
    after the dispatch either because the procedure won’t be found in the table 
    and it will cause an error. So I consider that every expression for which no 
    procedure of the correct type is found in the table is a procedure 
    application. I used a two-dimensional table as in chapter 2, though 
    a one-dimensional table would be enough if we don’t do any other dispatch on 
    the expressions’ type.
    \scm{ch4/4.03.scm}
\end{exe}

\begin{exe}[4.4]
    \label{4.4}
    If we define specific evaluation procedures \vscm{eval-and} and 
    \vscm{eval-or}, we have to add the following two lines to the definition of 
    \vscm{eval}
    \scm{ch4/4.04a.scm}
    Then we can implement the predicates, selectors and the evaluation 
    procedures in the following way:
    \scm{ch4/4.04b.scm}
    If \vscm{and} and {or} are implemented as derived expressions, the 
    predicates and selectors don’t change, but the lines regarding \vscm{and} 
    and \vscm{or} in \vscm{eval} are replaced by:
    \scm{ch4/4.04c.scm}
    And the procedures transforming \vscm{and} and \vscm{or} expressions into 
    \vscm{if} expressions can be defined as:
    \scm{ch4/4.04d.scm}
\end{exe}

\begin{exe}[4.5]
    \label{4.5}
    We only need to add the appropriate predicate and selectors for this type of 
    clause so that the appropriate action is taken if the predicate evaluates to 
    a true value.
    \scm{ch4/4.05.scm}
\end{exe}

\begin{exe}[4.6]
    \label{4.6}
    The only necessary modification to \vscm{eval} is to add the line:
    \scm{ch4/4.06b.scm}
    Then we can define the appropriate predicate and selectors and the 
    transformation procedure \vscm{let->combination} in the following way:
    \scm{ch4/4.06.scm}
\end{exe}

\begin{exe}[4.7]
    \label{4.7}
    A \vscm{let*} expression can be rewritten as a set of nested \vscm{let} 
    expressions defining only one binding at a time, so that the second binding 
    is defined in the body of the first \vscm{let} etc. For instance, the 
    example from the book becomes:
    \scm{ch4/4.07b.scm}
    If we have already implemented \vscm{let}, it is sufficient to rewrite 
    \vscm{let*} expressions in terms of \vscm{let} expressions to handle them. 
    This can be done in the following way:
    \scm{ch4/4.07.scm}
\end{exe}

\begin{exe}[4.8]
    \label{4.8}
    To support named \vscm{let}, we modify the selectors for \vscm{let} so they 
    handle both named \vscm{let}s and ordinary \vscm{let}s, and 
    \vscm{let->combination} so the generated code creates a procedure with the 
    given name instead of a lambda.
    \scm{ch4/4.08.scm}
\end{exe}

\begin{exe}[4.9]
    \label{4.9}
    I chose to implement three iterative control structures: a \vscm{while} 
    loop, an \vscm{until} loop, and a \vscm{for} loop. The return value of all 
    the expressions described is \vscm{false}, so they are useful only for their 
    side effects. The \vscm{until} and the \vscm{for} constructs are derived 
    from \vscm{while}.
    \begin{itemize}
        \item[\emph{while}] The syntax of the \vscm{while} loop is
            \vscm{(while <predicate> <body>)}. The predicate must consist of 
            a single expression. The body can consist of several expressions. 
            The body is executed while the predicate is true. If the predicate 
            is false from the beginning, the body is never evaluated.

            For instance, the evaluation of the following code prints the 
            numbers from 0 to 5 inclusive, each on a new line:
            \begin{cscm}
                (define x 0)
                (while (<= x 5)
                    (display x)
                    (newline)
                    (set! x (+ x 1)))
            \end{cscm}
            The transformation procedure for \vscm{while} expressions turns them 
            into the definition of a recursive procedure that uses the predicate 
            to decide whether to execude the body of the \vscm{while} or not, 
            followed by a call to this procedure. The procedure name is 
            generated by \vscm{gensym}, so it does not conflict with other names 
            in the program, and if two \vscm{while} are evaluated in the same 
            environment, the generated procedures will have different names.
            \scm{ch4/4.09while.scm}
        \item[\emph{until}] The syntax of the \vscm{until} loop is
            \vscm{(until <predicate> <body>)}, and such an expression is 
            equivalent to \vscm{(while (not <predicate>) <body>)}. For instance, 
            the following code prints the numbers from 0 to 4 inclusive, each on 
            a new line:
            \begin{cscm}
                (define x 0)
                (until (= x 5)
                    (display x)
                    (newline)
                    (set! x (+ x 1)))
            \end{cscm}
            \vscm{Until} expressions are easily derived from \vscm{while} 
            expressions in the following way:
            \scm{ch4/4.09until.scm}
        \item[\emph{for}] The general syntax of the \vscm{for} loop is
            \vscm{(for (<var> <start> <end-test> [<inc-exp>]) <body>)}. The body 
            can consist of several expressions. The other parts are:
            \begin{itemize}
                \item \vscm{<var>}: the name of the variable whose value is 
                    bound by the \vscm{for} loop.
                \item \vscm{<start>}: an expression giving the initial value of 
                    the variable.
                \item \vscm{<end-test>} can be either:
                    \begin{itemize}
                        \item an expression where \vscm{<var>} appears as a free 
                            variable, for instance \vscm{(< <var> 3)}. The body 
                            of the \vscm{for} is evaluated as long as the 
                            evaluation of this expression with the current value 
                            of \vscm{<var>} returns \vscm{true}.
                        \item an expression not depending on the variable bound 
                            by the \vscm{for}, whose evaluation must return 
                            a number. This is equivalent to either
                            \vscm{(<= <var> <end-test>)} or
                            \vscm{(>= <var> <end-test>)}, depending on whether 
                            the returned value is greater or smaller than the 
                            initial value.
                    \end{itemize}
                \item \vscm{<inc-exp>} can be either:
                    \begin{itemize}
                        \item an expression where \vscm{<var>} appears as a free 
                            variable, for instance \vscm{(+ <var> 2)}. It is 
                            evaluated to update the value of \vscm{<var>} before 
                            evaluating (or not) the body of the \vscm{for} 
                            again.
                        \item an expression not depending on the variable bound 
                            by the \vscm{for}, whose evaluation must return 
                            a number. This is equivalent to specifying
                            \vscm{(+ <var> <inc-exp>)}.
                        \item empty, in which case it will be assumed to be 1.
                    \end{itemize}
            \end{itemize}
            For instance, the following expressions all print the numbers from 
            0 to 5:
            \begin{cscm}
                (for (x 1 (<= x 5) (+ x 1)) (display x) (newline))

                (for (x 1 (<= x 5) 1) (display x) (newline))

                (for (x 1 (<= x 5)) (display x) (newline))

                (for (x 1 5 (+ x 1)) (display x) (newline))

                (for (x 1 5 1) (display x) (newline))

                (for (x 1 5) (display x) (newline))
            \end{cscm}
            Using a step other than one, we can print 1, .8, .6, .4, .2 and 
            0 with:
            \begin{cscm}
                (for (x 1 -.1 -.2)
                    (display x) (newline))
            \end{cscm}
            With an update function that is not simply an addition or 
            subtraction, the following prints 16, 8, 4, 2 and 1.
            \begin{cscm}
                (for (x 16 1 (/ x 2))
                    (display x) (newline))
            \end{cscm}
            It’s possible to use nested \vscm{for}s, for instance:
            \begin{cscm}
            (for (x 0 5)
                (for (y (+ x 1) (+ x 2))
                    (display x)
                    (display ", ")
                    (display y)
                    (newline)))
            \end{cscm}
            The output is:
            \begin{cscm}
                0, 1
                0, 2
                1, 2
                1, 3
                2, 3
                2, 4
                3, 4
                3, 5
                4, 5
                4, 6
                5, 6
                5, 7
            \end{cscm}
    \end{itemize}
    The implementation is more complex than that of \vscm{while}, because 
    several cases have to be considered. I had to write a procedure that checks 
    whether a given variable appears as a free variable in a given expression. 
    Then the different cases have to be considered when constructing the 
    predicate that tests whether to interrupt the loop or not, and the procedure 
    that updates the value of the variable. The \vscm{for} expression is then 
    transformed in a \vscm{let} expression containing a \vscm{while}.
    \scm{ch4/4.09for.scm}
\end{exe}

\begin{exe}[4.10]
    It’s easy to modify the symbol used as the \vscm{car} of an expression to 
    define its type: we only need changing the predicate (and the associated 
    constructor if any was defined) for this type of expression. For instance, 
    we can replace \vscm{cond} with \vscm{case}, \vscm{and} with \vscm{&&}, and 
    \vscm{or} with \vscm{||} with the following code:
    \scm{ch4/4.10a.scm}

    We can replace \vscm{not} with \vscm{!} by replacing
    \vscm{(list 'not not)} with \vscm{(list '! not)} in the list of primitive 
    procedures and redefining the \vscm{make-not} procedure defined in 
    exercise~\ref{4.9}:
    \scm{ch4/4.10b.scm}

    I also modified the selectors for procedure application so that
    \vscm{(proc args)} becomes \vscm{(proc (args))}. This change actually makes 
    the syntax more complex since e.g. \vscm{(not (> x 3))} has to be written 
    \vscm{(! ((> (x 3))))}, it’s just for the sake of the example.
    \scm{ch4/4.10c.scm}

    I think that more dramatic changes such as replacing the parentheses with 
    brackets would require a lot of work since currently we rely on the 
    underlying Scheme for list processing.
\end{exe}

\subsection{Evaluator Data Structures}

\begin{exe}[4.11]
    If each binding is represented as a name-value pair, each frame represents 
    a table, so we can simplify the implementation by using \vscm{assoc}. We can 
    change the implementation by rewriting the following procedures:
    \scm{ch4/4.11.scm}
\end{exe}

\begin{exe}[4.12]
    \label{4.12}
    We can define a procedure that recursively traverses the environment 
    structure and executes the given action when a binding is found for the 
    given variable as shown below. By default, the procedure goes to the 
    enclosing environment if the procedure is not found in the first frame, but 
    a second procedure can be passed as an argument to specify another behavior.

    Implementation for the representation of frames as a pair of lists:
    \scm{ch4/4.12a.scm}
    Implementation for the reperesentation of frames as a list of pairs:
    \scm{ch4/4.12b.scm}
\end{exe}

\begin{exe}[4.13]
    The solutions I read on the internet unbind variables only from the first 
    frame of the environment because modifying the enclosing environment seemed 
    too risky or similar reasons. I chose to delete the first binding found even 
    if it’s not in the first frame: the interpreter implementation already 
    allows us to change bindings in the enclosing environments, e.g. if I define 
    a function
    \vscm{(define (f x) (set! + -) x)}, the sequence of interactions:
    \begin{cscm}
	(+ 2 1)
	(f 2)
	(+ 2 1)
    \end{cscm}
    produces 3, 2 and 1. This choice is thus coherent with the rest of the 
    implementation.

    To add the \vscm{make-unbound!} operation, we add to \vscm{eval} the line:
    \begin{cscm}
	((unbind? exp) (eval-unbind exp env))
    \end{cscm}
    The necessary procedures with the implementation of frames as a pair of 
    lists are:
    \scm{ch4/4.13a.scm}

    The implementation of \vscm{delete-binding-from-frame} is simpler with 
    a list of pairs than with a pair of lists because there is only one list to 
    modify, and the list is headed, so the case where the variable to unbind is 
    the first in the frame need not be handled separately:
    \scm{ch4/4.13b.scm}
\end{exe}

\subsection{Running the Evaluator as a Program}

\begin{exe}[4.14]
    Louis’ \vscm{map} fails because the \vscm{map} procedure from the underlying 
    Scheme is called with a procedure representation from the interpreter as the 
    procedure to apply with the underlying Scheme’s \vscm{apply}, and it 
    considers that representation as a list and not as a procedure.
\end{exe}

\subsection{Data as Programs}

\begin{exe}[4.15]
    Let’s assume that \vscm{(halts? try try)} returns true. Then 
    \vscm{(run-forever)} is executed so \vscm{(try try)} does not halt.

    Let’s assume that on the contrary, \vscm{(halts? try try)} returns false. 
    Then \vscm{(try try)} halts.

    Both assumptions lead to a contradiction, so such a \vscm{halts?} procedure 
    can’t exist.
\end{exe}

\subsection{Internal Definitions}

\begin{exe}[4.16]
    \label{4.16}
    \ \vspace{-20pt}
    \begin{enumerate}
        \item Here is the updated version of \vscm{lookup-variable-value} for 
            the frame representation as a list of pairs, with the abstractions 
            defined in exercise~\ref{4.12}:
            \scm{ch4/4.16a.scm}
        \item The \vscm{scan-out-defines} procedure can be defined as:
            \scm{ch4/4.16b.scm}
            No transformation must be done if no definition is found because 
            \vscm{let} expressions are transformed into procedure applications, 
            so this would lead to infinite loops. The \vscm{*unassigned*} symbol 
            has to be quoted twice so that the returned code contains a quoted 
            symbol.
        \item It is better to install \vscm{scan-out-defines} in 
            \vscm{make-procedure} than in \vscm{procedure-body}, because the 
            latter is called every time the procedure is applied, while the 
            former is called only when it is defined.
            \scm{ch4/4.16c.scm}
    \end{enumerate}
\end{exe}

\begin{exe}[4.17]
    \begin{figure}
        \begin{tikzpicture}[>=Stealth, thick,
            code/.append style={font=\ttfamily},
            label distance=5mm]
            % Sequential evaluation
            \matrix[column sep=2\nametoenv, column 2/.style={anchor=west},
            label=below:{Sequential evaluation}] (gem1) {
            \node[text width=1cm, align=right] (ge1) {global\\ env};
            & \node[env,minimum width=4cm, minimum height=1cm] (geb1) 
            {};\\[+1.5cm]

            & \node[code] (vars1) {\ <vars>: <vals>}; \\
            \node[text width=1cm, align=right] (e11) {E1};
            & \node[code] (u1) {\ u: <e1>}; \\
            & \node[code] (v1) {\ v: <e2>}; \\
            };

            \begin{scope}[on background layer]
                \node[env, minimum width=4cm, fit=(vars1) (u1) (v1), below=of 
                geb1,yshift=-2mm] (e1b) {};
            \end{scope}

            \draw[->] (ge1) -- (ge1.west -| geb1.west);
            \draw[->] (e11) -- (e11.east -| e1b.west);
            \draw[->] (e1b) -- (geb1);

            % After scanning out defines
            \matrix[column sep=2\nametoenv, column 2/.style={anchor=west},
            right=of gem1, label=below:{After scanning out}] {
            \node[text width=1cm, align=right] (ge2) {global\\ env};
            & \node[env,minimum width=4cm, minimum height=1cm] (geb2) 
            {};\\[+1.5cm]

            \node[text width=1cm, align=right] (e12) {E1};
            & \node[code] (vars2) {\ <vars>: <vals>}; \\[+1.5cm]

            \node[text width=1cm, align=right] (e2) {E2};
            & \node[code] (u2) {\ u: <e1>}; \\
            & \node[code] (v2) {\ v: <e2>}; \\
            };

            \begin{scope}[on background layer]
                \node[env, minimum width=4cm, fit=(vars2), below=of 
                geb2,yshift=-2.5mm] (e12b) {};
                \node[env, minimum width=4cm, fit=(u2) (v2), below=of 
                e12b,yshift=-1mm] (e2b) {};
            \end{scope}

            \draw[->] (ge2) -- (ge2.west -| geb2.west);
            \draw[->] (e12) -- (e12.east -| e12b.west);
            \draw[->] (e12b) -- (geb2);
            \draw[->] (e2) -- (e2.east -| e2b.west);
            \draw[->] (e2b) -- (e12b);
        \end{tikzpicture}
        \caption{Environment structures in which \vscm{<e3>} is evaluated when 
        the definitions are interpreted sequentially and after scanning out.}
        \label{4.17fig}
    \end{figure}
    Figure~\ref{4.17fig} shows the environment structures in which \vscm{<e3>} 
    is evaluated when the definitions are interpreted sequentially and when they 
    are scanned out. In the latter case, there is an extra frame because 
    \vscm{let} corresponds to a \vscm{lambda} application.

    This difference in environment structure makes no difference in a correct 
    program’s behavior because the same bindings are accessible when the body of 
    the procedure is evaluated, and since the inner \vscm{let} contains the 
    whole body of the outer lambda, both frames in the latter case always go 
    together: if the outer lambda returns a procedure, its environment will be 
    E2, it can never be E1, so there can be no lost bindings.

    A possible solution would be to move all the inner definitions to the top of 
    the procedure body. This will work only if the values of the defined 
    variables don’t use the value of a variable defined later. This can be 
    implemented in the following way:
    \scm{ch4/4.17.scm}
\end{exe}

\begin{exe}[4.18]
    This procedure won’t work if internal definitions are scanned out as in this 
    exercise because the value of \vscm{y} is needed to evaluate the value of 
    \vscm{dy}, and in the exercise the value has been computed but not yet 
    assigned to \vscm{y}.

    It works if internal definitions are scanned out as shown in the text.
\end{exe}

\begin{exe}[4.19]
    Inner definitions should be simultaneous, so Eva is right, but if it’s too 
    difficult to implement internal definitions so they behave that way, it’s 
    better to signal an error than to use an incorrect value as Ben suggests.

    To implement Eva’s idea, we would have to sort the definitions so that if 
    the value of a defined variable is needed to compute another’s value, it 
    should come first. However, it won’t always be possible, definitions such as
    \begin{cscm}
	(define (f x)
	    (define a (+ b 5))
	    (define b (+ a 1))
	    <exps>)
    \end{cscm}
    should result in an error. But there are cases of mutual recursion that 
    work, so it’s not enough to scan the defined symbols’ names in other defined 
    symbols’ values to order the definitions: mutually recursive procedure 
    definitions are not problematic as long as neither procedure is called 
    before both are defined. Mutual recursion is not problematic either in cases 
    where evaluation is delayed, as in the \vscm{solve} example. A symbol could 
    also appear at a place where it will never be evaluated, for instance:
    \begin{cscm}
	(define (f x)
	    (define a (if (> 0 1) (* b 2) 3))
	    (define b 5)
	    <exps>)
    \end{cscm}
    So it seems difficult to implement a general solution that does what Eva 
    prefers.

    \medskip

    Another possible solution is to make definitions and assignments lazy by 
    automatically delaying their values’ evaluation, and forcing them only when 
    a variable’s value is looked up. This is a big change to Scheme’s evaluation 
    order, but since we have already seen how to use \vscm{delay} and 
    \vscm{force} in \secref{3.5} this is not difficult to implement. We need to 
    redefine \vscm{eval-definition} and \vscm{eval-assignment} so they delay the 
    values’ evaluation:
    \scm{ch4/4.19a.scm}
    Then we must call \vscm{force} when we look up a variable’s value, for 
    instance by modifying the \vscm{variable?} case in \vscm{eval}:
    \scm{ch4/4.19b.scm}
\end{exe}

\begin{exe}[4.20]
    \label{4.20}
    \ \vspace{-20pt}
    \begin{figure}
        \begin{tikzpicture}[>=Stealth, thick,
            code/.append style={font=\ttfamily},
	    label distance=5mm,
	    env/.append style={minimum width=3cm, minimum height=1cm}]
	    % With letrec
	    \matrix[column sep=2\nametoenv, column 2/.style={anchor=west}, 
	    label={[xshift=1cm]above:{With letrec}}] (gem1) {
            \node[text width=1cm, align=right] (ge1) {global\\ env};
	    & \node[code] (gec1) {\ f: ...};\\[+1.5cm]

	    \node[text width=1cm, align=right] (e111) {E1};
	    & \node[code] (x1) {\ x: 5}; \\[+1.5cm]

	    \node[text width=1cm, align=right] (e11) {E2};
	    & \node[code] (even1) {\ even?:}; \\
	    & \node[code] (odd1) {\ odd?:}; \\
	    };

            \begin{scope}[on background layer]
		\node[env, fit=(gec1), xshift=2\nametoenv] (geb1) {};
		\node[env, fit=(x1), below=of geb1, yshift=-2mm] (e21b) {};
		\node[env, fit=(even1) (odd1), below=of e21b] (e1b) {};
            \end{scope}

	    \coordinate (odd1px) at ($ (e1b.south west)!.5!(e1b.south) $);
	    \coordinate[below=of odd1px] (odd1p);
	    \procedure{odd1p}{oc1}{oe1}
	    \draw[->] (oe1) -- (oe1 |- e1b.south);
	    \draw[->] (odd1.south -| odd1p) -- ($ (odd1p) + (0, \circleradius) 
	    $);

	    \coordinate (even1px) at ($ (e1b.south)!.5!(e1b.south east) $);
	    \coordinate[below=of even1px] (even1p);
	    \procedure{even1p}{ec1}{ee1}
	    \draw[->] (ee1) -- (ee1 |- e1b.south);
	    \draw[->] (even1.east) -| ($ (even1p) + (0, \circleradius) $);

            \draw[->] (ge1) -- (ge1.west -| geb1.west);
            \draw[->] (e11) -- (e11.east -| e1b.west);
	    \draw[->] (e111) -- (e111.east -| e21b.west);
	    \draw[->] (e21b) -- (geb1);
	    \draw[->] (e1b) -- (e21b);

	    % With let
	    \matrix[right=3cm of gem1, column sep=2\nametoenv, column 
	    2/.style={anchor=west},
	    label={[xshift=1cm]above:{With let}}] {
	    \node[text width=1cm, align=right] (ge2) {global\\ env};
	    & \node[code] (gec2) {\ f: ...};\\[+1.5cm]

	    \node[text width=1cm, align=right] (e112) {E1};
	    & \node[code] (x2) {\ x: 5}; \\[+1.5cm]

	    \node[text width=1cm, align=right] (e12) {E2};
	    & \node[code] (even2) {\ even?:}; \\
	    & \node[code] (odd2) {\ odd?:}; \\
	    };

            \begin{scope}[on background layer]
		\node[env, fit=(gec2), xshift=2\nametoenv] (geb2) {};
		\node[env, fit=(x2), below=of geb2, yshift=-2mm] (e22b) {};
		\node[env, fit=(even2) (odd2), below=of e22b] (e2b) {};
            \end{scope}

	    \coordinate (odd2px) at ($ (e2b.south west)!.5!(e2b.south) $);
	    \coordinate[below=of odd2px] (odd2p);
	    \procedure{odd2p}{oc2}{oe2}
	    \draw[->] (oe2) -- ($ (oe2) + (0,-.8cm) $) -- ++(3cm,0) |-
	    ($ (e22b.east)!.4!(e22b.north east) $);
	    \draw[->] (odd2.south -| odd2p) -- ($ (odd2p) + (0, \circleradius) 
	    $);

	    \coordinate (even2px) at ($ (e2b.south)!.5!(e2b.south east) $);
	    \coordinate[below=of even2px] (even2p);
	    \procedure{even2p}{ec2}{ee2}
	    \draw[->] (ee2) -- ($ (ee2) + (1cm,0) $) |-
	    ($ (e22b.east)!.4!(e22b.south east) $);
	    \draw[->] (even2.east) -| ($ (even2p) + (0, \circleradius) $);

            \draw[->] (ge2) -- (ge2.west -| geb2.west);
            \draw[->] (e12) -- (e12.east -| e2b.west);
	    \draw[->] (e112) -- (e112.east -| e22b.west);
	    \draw[->] (e22b) -- (geb2);
	    \draw[->] (e2b) -- (e22b);
	\end{tikzpicture}
	\caption{Environment structures in which \vscm{<rest of body of f>} is 
	evaluated during evaluation of the expression \vscm{(f 5)}.}
	\label{4.20fig}
    \end{figure}
    \begin{enumerate}
        \item \vscm{Letrec} can be implemented as shown below. This is very 
            similar to exercise~\ref{4.16}, so the code could be mutualized:
	    \scm{ch4/4.20.scm}
        \item Figure~\ref{4.20fig} shows the environment structures in which the 
            \vscm{<rest of body of f>} is evaluated during evaluation of the 
            expression \vscm{(f 5)}, first with \vscm{letrec}, then with 
            \vscm{let}. With \vscm{letrec}, the values of \vscm{odd?} and 
            \vscm{even?} are evaluated in an environment where the bindings 
            already exist, so the procedures created point to an environment 
            where the other is defined. With \vscm{let}, the values of 
            \vscm{even?} and \vscm{odd?} are evaluated before the bindings 
            exist, so they point to an environment where they are undefined and 
            the recursive calls won’t work.
    \end{enumerate}
\end{exe}

\begin{exe}[4.21]
    \ \vspace{-20pt}
    \begin{enumerate}
        \item The trick used here is to modify the recursive procedure 
            (\vscm{(lambda (ft k) ...)} in the exercise) so it takes as an 
            additional parameter the procedure to call where a recursive call 
            would be used otherwise, and the argument used is the modified 
            procedure itself. The first inner \vscm{lambda}
	    (\vscm{(lambda (fact) ...)}) is used to do the initial call to the 
	    modified (derecursified?) procedure.

	    The Fibonacci numbers can be computed similarly:
	    \scm{ch4/4.21a.scm}

        \item This time, since there are two mutually recursive procedures, two 
            procedure parameters are needed to pass around the procedures to 
            call when we are not in the terminal case.
	    \scm{ch4/4.21b.scm}
    \end{enumerate}
\end{exe}

\subsection{Separating Syntactic Analysis from Execution}

\begin{exe}[4.22]
    Since \vscm{let} is a derived form, all that’s needed to support it is to 
    add the following line to \vscm{analyze}:
    \scm{ch4/4.22a.scm}
    \begin{comp}
        We can just as easily add support for all the derived forms implemented 
        in the previous exercises:
        \scm{ch4/4.22b.scm}
        We need to implement \vscm{analyze} procedures to support the versions of 
        \vscm{or} and \vscm{and} defined with special evaluation functions, as 
        well as \vscm{make-unbound!}:
        \scm{ch4/4.22c.scm}
        Lastly, we can implement scanning out of internal definitions as in 
        exercise~\ref{4.16} by calling \vscm{scan-out-defines} in 
        \vscm{analyze-lambda}:
        \scm{ch4/4.22d.scm}
    \end{comp}
\end{exe}

\begin{exe}[4.23]
    If the sequence has just one expression, the procedure produced by Alyssa’s 
    program tests the \vscm{cdr} of \vscm{procs}, finds that it’s null and then 
    calls the first procedure. The program in the text does the test during 
    analysis and returns the result of analyzing the expression directly, so the 
    only work done during evaluation is to call it.

    For a sequence with two expressions, Alyssa’s procedure will loop through 
    the procedures each time the sequence is evaluated, while the procedure from 
    the text loops through them only once during analysis.
\end{exe}

\begin{exe}[4.24]
    I wrote a small procedure to interpret code with the interpreter without 
    using the REPL (I should probably have done that some time ago…):
    \scm{ch4/4.24a.scm}
    I then ran the following two tests: one with a recursive procedure and the 
    other with a non-recursive procedure:
    \scm{ch4/4.24b.scm}
    With the evaluator in this section, the first test evaluates in 7877~ms, the 
    second one in 36~ms.

    With the original version, the first test evaluates in 16024~ms, the second 
    one in 79~ms.

    Separating analysis from execution speeds up execution by a factor of more 
    than two, so we can estimate that about half the time was spent in analysis 
    with the original version of the evaluator.
\end{exe}

\section{Variations on a Scheme―Lazy Evaluation}

\subsection{Normal Order and Applicative Order}

\begin{exe}[4.25]
    If we attempt to evaluate \vscm{(factorial 5)}, we will get an infinite loop 
    since Scheme attempts to evaluate the second argument to \vscm{unless} 
    recursively.

    It would work in a normal-order language.
\end{exe}

\begin{exe}[4.26]
    Ben is right that it’s possible to define \vscm{unless} as a derived 
    expression:
    \scm{ch4/4.26a.scm}
    Then we just have to have the following line to \vscm{eval}:
    \scm{ch4/4.26b.scm}
    or the following line to \vscm{analyze} for the evaluator from 
    \secref{4.1.7}:
    \scm{ch4/4.26c.scm}

    It would be useful to have \vscm{unless} available as a procedure rather 
    than a special form to use it as a parameter to higher-order procedures, for 
    instance in \vscm{(map unless bools list1 list2)}, which returns a list 
    where the element of index $i$ is:
    \begin{itemize}
	\item the element of index $i$ of \vscm{list2} if the element of index 
	    $i$ in the list \vscm{bools} is true;
	\item the element of index $i$ of \vscm{list1} otherwise.
    \end{itemize}
\end{exe}

\subsection{An Interpreter with Lazy Evaluation}

\begin{exe}[4.27]
    The value of \vscm{count} the first time is 1 because the body of the outer 
    \vscm{id} in the definition of \vscm{w} was evaluated. Then the value of 
    \vscm{w} is 10 as expected, and once \vscm{w} has been forced the value of 
    \vscm{count} is 2 because the inner \vscm{id} in the definition of \vscm{w} 
    was evaluated as well.
\end{exe}

\begin{exe}[4.28]
    This forcing is needed if the operator has been passed as an argument to 
    a higher-order procedure, for instance in \vscm{map}.
\end{exe}

\begin{exe}[4.29]
    Recursive procedures such as \vscm{factorial}, defined as usual as:
    \scm{ch4/4.29.scm}
    run much more slowly without memoization. Let’s compare what happens when we 
    evaluate, for intance \vscm{(factorial 10)} with and without memoization:
    \begin{itemize}
	\item With memoization, the argument is evaluated when the value of 
	    \vscm{(= n 0)} is needed in the \vscm{if} expression. Then 
	    \vscm{(factorial (- n 1))} is evaluated, and the argument passed to 
	    factorial is a thunk containing the expression \vscm{(- n 1)}, where 
	    \vscm{n} is an evaluated thunk with value 10. This new thunk is 
	    transformed into an evaluated thunk with value 9 when testing the 
	    predicate in the \vscm{if}, and this evaluated thunk’s value is 
	    immediately accessible in the evaluation of the argument of the next 
	    call to \vscm{factorial}. And so on until 0 is reached. The time 
	    complexity of the computation is linear, the same as with 
	    applicative-order.
	\item Without memoization, the thunk with value 10 is evaluated in the 
	    \vscm{if} predicate, but it is not transformed into an evaluated 
	    thunk. So the recursive call to \vscm{factorial} has as its argument 
	    a thunk containing \vscm{(- n 1)}, where \vscm{n} is an unevaluated 
	    thunk. Two evaluations are needed to find that the value of this 
	    thunk is 9. Then, for the next recursive call, 3 evaluations are 
	    needed to find that the value of the argument is 8. During the 
	    terminal call, 11 thunk evaluations are needed each time the 
	    argument’s value is needed. As a consequence, the time complexity of 
	    \vscm{factorial} with a lazy evaluator that does not memoize is 
	    quadratic instead of linear.
    \end{itemize}
    Of course, the same happens with any recursive procedure running in linear 
    time, such as \vscm{length} to compute a list’s length.

    \medskip

    The value of \vscm{(square (id 10))} is 100 both when the evaluator memoizes 
    and when it does not. When the evaluator memoizes, the value of \vscm{count} 
    after evaluating \vscm{(square (id 10))} is 1 because \vscm{(id 10)} has 
    been evaluated only once. When the evaluator does not memoize, the value of 
    \vscm{count} is 2 because \vscm{(id 10)} has been evaluated twice.
\end{exe}

\begin{exe}[4.30]
    \ \vspace{-20pt}
    \begin{enumerate}
        \item Ben is right about the behavior of \vscm{for-each} because each 
            expression in the body of the \vscm{begin} expression is evaluated 
            with \vscm{eval}, which causes each expression in the body of 
            \vscm{proc} to be evaluated with each item in turn, so the 
            side-effects they cause do take place.

	    The only case where a side effect could not take place is when the 
	    expression defining it is delayed and then never forced, which can 
	    happen only if that expression was passed as an argument to 
	    a compound procedure, as in b. below.
        \item With the original \vscm{eval-sequence}, the value of
	    \vscm{(p1 1)} is \vscm{(1 2)} because the \vscm{set!} expression in 
	    the body of \vscm{p1} was evaluated. The value of \vscm{(p2 1)} is 
	    \vscm{1} because during the evaluation of \vscm{e} in the body of 
	    \vscm{p}, \vscm{e} is a variable whose value is a thunk containing 
	    an expression defining a side effect, but this expression is never 
	    actually evaluated because \vscm{e}’s value is not used.

	    With Cy’s proposed change to \vscm{eval-sequence}, the values would 
	    be \vscm{(1 2)} for both \vscm{(p1 1)} and \vscm{(p2 1)}.
        \item The proposed change does not affect the behavior of the example in 
            part a. because if the result of \vscm{eval} is not a thunk, 
            applying \vscm{force-it} to it does nothing.
        \item I prefer the approach in the text. I don’t think the change 
            proposed by Cy is necessary because, as noticed in part a., the only 
            case when a side effect could not take place is when the expression 
            causing it―not a procedure containing that expression, unless said 
            procedure is not called of course—is passed as an argument to 
            a compound procedure\footnote{It could also happen if an argument is 
            the result of a procedure application that produces side effects and 
            returns a value. If this argument’s value is not needed in the body 
            of the procedure, the side effects won’t happen, but this is to be 
            expected with normal-order evaluation and this is not directly 
            related to the evaluation of sequences.}, which I don’t think should 
            be done anyway: an expression passed as an argument should be there 
            for its value, not for its side effects, and it does not make much 
            sense to use an expression defining only side effects where a value 
            should be used. For the example of \vscm{p2} in the exercise, if we 
            want to use a parameter of \vscm{p} to define a side-effect before 
            returning \vscm{x}, it is better to use a procedure, and it works 
            without changing the evaluator’s behavior:
	    \scm{ch4/4.30.scm}
	    When reading the definition above, we can tell that the parameter 
	    \vscm{e} should be a procedure, and that it is expected to cause 
	    side-effects because its return value is not used. From reading the 
	    definition of \vscm{p2}, it is not clear at all what the evaluation 
	    of \vscm{e} is expected to do. It would be even less clear with 
	    applicative-order since \vscm{e} would already have been evaluated 
	    before applying \vscm{p}, so evaluating it again in the body of 
	    \vscm{p} would do absolutely nothing.

	    To sum it up, it can indeed happen that some side effects do not 
	    take place with the approach taken in the text, but I think that the 
	    programs where this happens can be modified in a straightforward way 
	    to fix the problem, and furthermore the modification is likely to 
	    make them clearer.
    \end{enumerate}
\end{exe}

\begin{exe}[4.31]
    In order to have both memoized and non memoized thunks, we define a new type 
    of thunk with an associated delay procedure. I chose to define a new type 
    for non memoized thunks. Then we redefine \vscm{force-it} so it handles the 
    three types of thunks―unevaluated non memoized, unevaluated to memoize, and 
    already evaluated—instead of two. And lastly we modify \vscm{apply} so it 
    delays each argument or not as appropriate.
    \scm{ch4/4.31.scm}
\end{exe}

\subsection{Streams as Lazy Lists}

\begin{exe}[4.32]
    With the streams of \secref{3.5}, there were places where we had to define 
    the first element of a stream separately to avoid infinite loops. With the 
    lazy evaluator, we can simplify some of the definitions. For instance, the 
    following definition of \vscm{pairs} taken from exercise~\ref{3.68}, where 
    it caused an infinite loop, works for lazy lists:
    \scm{ch4/4.32a.scm}
    Similar simplifications can be applied to the answers to 
    exercises~\ref{3.69} and~\ref{3.70}.

    \medskip

    As noticed in the text, the values of the elements of the lazy list won’t 
    actually be computed until they are absolutely needed, so that for instance, 
    if we define the lazy lists:
    \scm{ch4/4.32b.scm}
    we get no error, while this would cause an error with streams.

    \medskip

    And if we define a \vscm{length} procedure:
    \scm{ch4/4.32c.scm}
    we can compute \vscm{l2}’s length without trouble since the values of the 
    elements are not used.

    \medskip

    As noted in the book’s footnote, lazy lists allow us to define arbitrary 
    lazy data structures which can’t be defined with the streams of 
    \secref{3.5}, such as trees where all branches could potentially be 
    infinite.
\end{exe}

\begin{exe}[4.33]
    We need to convert the lists from the underlying Scheme to lazy lists using 
    a series of \vscm{cons}. We now need to call \vscm{eval} when evaluating 
    quotations, so the statement for quotations in \vscm{eval} is replaced with:
    \scm{ch4/4.33a.scm}
    and the \vscm{eval-quotation} procedure can be defined as:
    \scm{ch4/4.33b.scm}
\end{exe}

\begin{exe}[4.34]
    My initial idea was to implement \vscm{cons}, \vscm{car} and \vscm{cdr} as 
    special forms. In the end, I decided to find a solution where they are 
    ordinary procedures so that they can be used in higher-order procedures.

    I redefined \vscm{cons} so that it produces a pair with the tag 
    \vscm{lazy-pair}, the actual pair being represented by the same procedure as 
    in the text. But the tagged pair must be produced with the \vscm{cons} 
    procedure from the underlying Scheme, because it needs to be recognized by 
    the implementation language. So we first save the values of \vscm{cons}, 
    \vscm{car} and \vscm{cdr} as initial procedures before redefining them in 
    the interpreter as shown below:
    \scm{ch4/4.34e.scm}
    I also defined a \vscm{lazy-struct->pairs} procedure in the implemented 
    language to transform a structure built from lazy pairs into a structure 
    built from ordinary pairs. It uses the value of the global variables 
    \vscm{*max-depth*} and \vscm{*max-breadth*} to limit the number of items 
    included in the result. The advantage of defining the transformation in the 
    implemented language rather than in the evaluator is that the values of 
    \vscm{*max-depth*} and \vscm{*max-breadth*} can be changed from the driver 
    loop.
    \scm{ch4/4.34f.scm}
    Then we need to define \vscm{lazy-pair?} in the underlying language so that
    the evaluator can identify them:
    \scm{ch4/4.34b.scm}
    Then we can rewrite \vscm{user-print} so it applies 
    \vscm{lazy-struct->pairs} to lazy pairs before printing the result:
    \scm{ch4/4.34d.scm}
    The last step is to modify the evaluator so it treats lazy pairs as 
    self-evaluating values, for instance by adding:
    \scm{ch4/4.34c.scm}
\end{exe}

\section{Variations on a Scheme―Nondeterministic Computing}

\subsection{Amb and Search}

\begin{exe}[4.35]
    \label{4.35}
    The procedure \vscm{an-integer-between} can be defined as:
    \scm{ch4/4.35tex.scm}
\end{exe}

\begin{exe}[4.36]
    Replacing \vscm{an-integer-between} by \vscm{an-integer-starting-from} in 
    the procedure in exercise~\ref{4.35} wouldn’t work because the interpreter 
    would pick the lowest possible value for $i$ and $j$ and would then try all 
    the possible values for $k$ without ever finding a successful triple.

    We can generate all Pythagorean triples by first picking $j$, and then 
    picking $i \leq j$. We can then define $k$ as $\sqrt{i^2 + j^2}$ and check 
    if it’s an integer. To define a procedure more similar to the one in 
    exercise~\ref{4.35}, we could choose $k$ between finite bounds, for instance 
    $j$ and $2j$. Either way, once $j$ has been picked, there are only a finite 
    number of possibilities to test for $i$ and $k$, and these possibilities 
    contain all the Pythagorean triples for the given value of $j$, so all 
    Pythagorean triples could in principle be generated by typing 
    \vscm{try-again}.
    \scm{ch4/4.36tex.scm}
\end{exe}

\begin{exe}[4.37]
    Ben is correct. The procedure in exercise~\ref{4.35} systematically searches 
    all possible triples $(i, j, k)$ with $low \leq i \leq j \leq k \leq high$. 
    Ben’s version eliminates the pairs $(i, j)$ for which $i^2 + j^2 > high^2$ 
    since there is no possible value of $k$ within the bounds in this case, and 
    then instead of trying all possible values for $k$, it tests only whether 
    $\sqrt{i^2 + j^2}$ is an integer, since this is necessarily the value of $k$ 
    if $(i, j, k)$ is a Pythagorean triple.
\end{exe}

\subsection{Examples of Nondeterministic Programs}

\subsubsection{Logic Puzzles}

\begin{exe}[4.38]
    The only modification needed is to remove the line:
    \begin{cscm}
        (require (not (= (abs (- smith fletcher)) 1)))
    \end{cscm}
    Without this requirement, there are five possible solutions:
    \begin{cscm}
        ((baker 1) (cooper 2) (fletcher 4) (miller 3) (smith 5))
        ((baker 1) (cooper 2) (fletcher 4) (miller 5) (smith 3))
        ((baker 1) (cooper 4) (fletcher 2) (miller 5) (smith 3))
        ((baker 3) (cooper 2) (fletcher 4) (miller 5) (smith 1))
        ((baker 3) (cooper 4) (fletcher 2) (miller 5) (smith 1))
    \end{cscm}
\end{exe}

\begin{exe}[4.39]
    The order of the restrictions does not affect the answer. It does affect the 
    time to find an answer but in a limited way: the original procedure takes 
    about 370~ms on my computer, and I can’t do better than about 270~ms with 
    the following order:
    \scm{ch4/4.39tex.scm}

    The only influence of the order of restrictions is that a possibility that 
    leads to a dead end can be rejected after less computation time depending on 
    that order. However, the real reason why the procedure is slow is that a lot 
    of possibilities are explored though they could be ruled out from the start 
    (as shown in the following exercise). For instance, \vscm{cooper} must not 
    be equal to 1, but for a given value of \vscm{baker}, the procedure will 
    explore and eliminate all the $5^3 = 75$ possibilities where \vscm{cooper} 
    is 1 before setting \vscm{cooper} to 2.
\end{exe}

\begin{exe}[4.40]
    Before the requirement that floor assignments be distinct, there are $5^5 
    = 3125$ sets of assignments of people to floors. After that requirement, 
    there are $5! = 120$ such sets.

    The following procedure finds the answer in about 40~ms:
    \scm{ch4/4.40a.scm}
    We can make it even faster by breaking up the
    \vscm{(require (distinct? ...))} requirement into a series of
    \vscm{(require (not (= ...)))} so that any case where two values are not 
    distinct is ruled out as soon as possible. The following procedure solves 
    the problem in about 20~ms:
    \scm{ch4/4.40b.scm}
    It would also be more efficient to modify the list of possibilities in 
    \vscm{amb} for the people with floor restrictions, for instance using 
    \vscm{(amb 2 3 4)} for \vscm{fletcher}. But this could be considered as 
    solving part of the problem for the evaluator instead of stating the 
    solution’s requirements.
\end{exe}

\begin{exe}[4.41]
    Each assignment of people to floors where each floor has a unique person 
    corresponds to a permutation of the set \vscm{(1 2 3 4 5)}, where we can 
    consider that the first number is Baker’s floor, the second number is 
    Cooper’s floor and so on. We can solve the puzzle by generating all the 
    permutations and then filtering the set to keep only those that verify the 
    puzzle’s requirements.

    We already defined a \vscm{permutations} procedure in 
    \secref[Nested-Mappings]{2.2.3}, but I rewrote a different procedure to 
    generate them before remembering that, and then I noticed that my version 
    was faster so I kept it. To generate the permutations of a set $S$, 
    I generate all the permutations of the set minus the first element, and then 
    for each permutation I insert the removed element at each possible position 
    in the returned list. The version in \secref[Nested-Mappings]{2.2.3} 
    generated all permutations of $S - x$ for \emph{each} element $x$ of $S$ and 
    adjoined $x$ at the front of each permutation, whereas I generate all 
    permutations of $S - x$ only for one $x$ and the adjoin $x$ at all possible 
    positions.
    \scm{ch4/4.41.scm}
\end{exe}

\begin{exe}[4.42]
    We can solve the puzzle by defining a \vscm{liars} procedure in the amb 
    evaluator as shown below. It uses the helper procedure \vscm{xor} to 
    simplify the expression of the requirements.
    \scm{ch4/4.42tex.scm}
    The procedure’s output is:
    \begin{cscm}
        ((Betty 3) (Ethel 5) (Joan 2) (Kitty 1) (Mary 4))
    \end{cscm}
    so the order in which the girls were placed was: Kitty, Joan, Betty, Mary, 
    and Ethel.
\end{exe}

\begin{exe}[4.43]
    The puzzle can be solved by the following procedure, where we define lists
    containing the daughter’s name and the boat’s name for each father.
    \scm{ch4/4.43tex.scm}
    Some of the requirements are redundant, for instance since we know the name 
    of Barnacle’s boat and of his daughter, it’s not strictly necessary to check 
    that they are distinct. Keeping these requirements makes it easier to modify 
    the procedure if some constraints are removed.

    The procedure returns a single possible solution, which is
    \begin{cscm}
        ((Moore (Mary-Ann Lorna))
         (Downing (Lorna Melissa))
         (Hall (Gabrielle Rosalind))
         (Barnacle (Melissa Gabrielle))
         (Parker (Rosalind Mary-Ann)))
    \end{cscm}
    so Lorna’s father is Colonel Downing.

    If we are not told that Mary Ann’s father is Moore, the problem has two 
    solutions. The first one is the same as above, the second one is:
    \begin{cscm}
        ((Moore (Gabrielle Lorna))
         (Downing (Rosalind Melissa))
         (Hall (Mary-Ann Rosalind))
         (Barnacle (Melissa Gabrielle))
         (Parker (Lorna Mary-Ann)))
    \end{cscm}
\end{exe}

\begin{exe}[4.44]
    We use the same idea as in exercise~\ref{2.42}: we place a queen in each 
    column successively. Once we have placed $k - 1$ queens in the first $k - 1$ 
    columns, we place a queen in the $k$th column and require that it is safe 
    with respect to the others. The \vscm{safe?} procedure is the same as in 
    exercise~\ref{2.42}.
    \scm{ch4/4.44tex.scm}
\end{exe}

\subsubsection{Parsing natural language}

\begin{exe}[4.45]
    The five ways in which the sentence can be parsed are:
    \begin{itemize}
        \item The professor lectures with the cat, in the class, to the student.
            \scm{ch4/4.45a.scm}
        \item The professor lectures to the student, in the class that has the 
            cat.
            \scm{ch4/4.45b.scm}
        \item The professor lectures with the cat, to the student who is in the 
            class.
            \scm{ch4/4.45c.scm}
        \item The professor lectures to the student in the class, the student 
            has the cat.
            \scm{ch4/4.45d.scm}
        \item The professor lectures to the student, who is in the class that 
            has the cat.
            \scm{ch4/4.45e.scm}
    \end{itemize}
\end{exe}

\begin{exe}[4.46]
    The \vscm{parse-word} procedure looks for the required part of speech at the 
    beginning of the contents of \vscm{*unparsed*}. So with 
    \vscm{parse-sentence} defined as:
    \begin{cscm}
        (define (parse-sentence)
          (list 'sentence
                (parse-noun-phrase)
                (parse-word verbs)))
    \end{cscm}
    when we try to parse \vscm{(the cat eats)}, if \vscm{(parse-word verbs)} is 
    evaluated first, it will fail since \vscm{the} is not in the list of verbs, 
    and there are no alternatives to try.
\end{exe}

\begin{exe}[4.47]
    The change suggested by Louis does not work because the second branch of the 
    \vscm{amb} contains an infinite loop. For instance, if we parse the sentence 
    “The cat eats.”, we get the correct result, but if we then type 
    \vscm{try-again}, there is an infinite loop as the evaluator tries the 
    second branch of the \vscm{amb} in \vscm{parse-verb-phrase}: the recursive 
    call to \vscm{parse-verb-phrase} succeeds as it finds the verb \vscm{eats}, 
    then the call to \vscm{parse-prepositional-phrase} fails since there is 
    nothing left to parse, which causes the second branch of \vscm{amb} to be 
    explored in the recursive call, and so on to infinity.

    If we interchange the order of the expressions in the \vscm{amb}, the 
    infinite loop is immediately apparent.
\end{exe}

\begin{exe}[4.48]
    \label{4.48}
    I decided to allow an arbitrary number of adjectives in front of a noun, and 
    an adverb after a verb. This is done by running the following code in the 
    interpreter:
    \scm{ch4/4.48tex.scm}
    We can then parse sentences sentences such as “The big black cat eats fast.” 
    with as result:
    \scm{ch4/4.48s.scm}
\end{exe}

\begin{exe}[4.49]
    We make \vscm{parse-word} return a random element of the given list by 
    redefining it as:
    \scm{ch4/4.49tex.scm}
    The first sentences I get using \vscm{(parse '())} (the input is ignored) 
    are:
    \begin{itemize}
        \item The students eats.
        \item A class sleeps.
        \item A student eats.
        \item The cat studies.
        \item A class studies.
        \item A cat eats.
    \end{itemize}
    Every time, the evaluator selects the first choice in \vscm{amb} to generate 
    the sentence, so all the sentences have the same structure.

    The result is not much better if I type \vscm{(parse '())} and then type 
    \vscm{try-again} several times:
    \begin{itemize}
        \item A professor studies.
        \item A professor studies by the student.
        \item A professor studies by the student by a student.
        \item A professor studies by the student by a student by the professor.
        \item A professor studies by the student by a student by the professor 
            in the cat.
        \item A professor studies by the student by a student by the professor 
            in the cat for the professor.
    \end{itemize}
    A sentence consists of a noun phrase followed by a verb phrase, and a verb 
    phrase is a verb optionally followed by one or more prepositional phrases. 
    Each use of \vscm{try-again} causes the evaluator to go back to the choice 
    point in the \vscm{amb} in \vscm{parse-verb-phrase}, which adds a new 
    prepositional phrase at the end of the sentence.
\end{exe}

\subsection{Implementing the \vscm{Amb} Evaluator}

\begin{exe}[4.50]
    We can define \vscm{ramb} by adding the following dispatch clause to 
    \vscm{analyze}:
    \scm{ch4/4.50a.scm}
    and defining the following procedures:
    \scm{ch4/4.50b.scm}
    Alyssa can replace \vscm{amb} with \vscm{ramb} in \vscm{parse-verb-phrase} 
    and \vscm{parse-noun-phrase} so that her generator will use a random 
    sentence structure. The sentences generated by \vscm{(parse '())} exhibit 
    various structures:
    \begin{itemize}
        \item A professor with the student for the class for a student in 
            a student in a student to the student by the class to a class to the 
            cat with a class for a student with the class in a professor with 
            a student by a class to the cat by the class by a cat in a cat for 
            the class with a class by a cat for a professor by the professor for 
            a cat by the student for the professor in a student for the class 
            with a class with a professor to the student in the class by the 
            class for the professor in the cat in a professor to the cat by 
            a class with a student for a class in a professor to a student to 
            the professor for the professor for a student by a cat with the 
            student by the student in the cat by a class by the class to a cat 
            by a student for a class by a student to the professor with 
            a student with the professor with the cat to the student eats.
        \item The cat to the class lectures.
        \item The cat sleeps in a class by the cat by a student for the student 
            by a professor with a student for a class.
        \item The class to the student by the student eats.
        \item A student eats to a class in a student.
        \item A professor sleeps.
    \end{itemize}
    Some generated sentences if we also include the adjectives and adverbs added 
    in exercise~\ref{4.48}:
    \begin{itemize}
        \item A class studies.
        \item The student for the cat to a beautiful student by the student with 
            a class for a class eats by the professor to the cat in the big 
            student in the class for the student in a professor for the class in 
            a lazy professor by the lazy student in a lazy black clever lazy 
            clever class in a lazy class by the student to the cat in a class in 
            the clever cat by the lazy cat in the lazy professor by the clever 
            professor for the big beautiful big cat to the big clever lazy big 
            cat with the student with a class by the class by the beautiful 
            beautiful beautiful class with the student with the beautiful big 
            black cat by the clever clever professor for a clever clever 
            professor in a black big clever student for the cat for the 
            professor with a big student to the student to the cat in the 
            professor with a black student to the cat for a professor.
        \item A cat lectures fast to a professor to a professor.
        \item The black professor eats well by the beautiful beautiful 
            professor.
    \end{itemize}
\end{exe}

\begin{exe}[4.51]
    The \vscm{permanent-set!} assignement can be defined similarly to 
    \vscm{set!}, except that it simply passes along the failure continuation 
    instead of intercepting it to undo the change in case of failure:
    \scm{ch4/4.51.scm}
    If we had used \vscm{set!} rather than \vscm{permanent-set!}, the displayed 
    values would have been \vscm{(a b 1)}, since the increment of \vscm{count} 
    done during the first trial would have been undone before the second trial.
\end{exe}

\begin{exe}[4.52]
    The \vscm{if-fail} construct can be defined in the following way, after 
    adding the appropriate clause to \vscm{analyze} as usual:
    \scm{ch4/4.52.scm}
\end{exe}

\begin{exe}[4.53]
    The result is \vscm{((8 35) (3 110) (3 20))}. When \vscm{(amb)} is 
    evaluated, it causes the interpreter to go back to the previous choice 
    point, in \vscm{prime-sum-pair}, and the \vscm{let} expression fails only 
    when \vscm{prime-sum-pairs} has no more alternative to explore, after the 
    three pairs whose sum is prime have been permanently added to \vscm{pairs}.
\end{exe}

\begin{exe}[4.54]
    The \vscm{analyze-require} procedure could have been defined as:
    \scm{ch4/4.54.scm}
\end{exe}

\section{Logic Programming}

\subsection{Deductive Information Retrieval}

\subsubsection{Simple queries}

\begin{exe}[4.55]
    The queries that retrieve the required information are:
    \begin{enumerate}
        \item \scm{ch4/4.55a.scm}
        \item \scm{ch4/4.55b.scm}
        \item \scm{ch4/4.55c.scm}
    \end{enumerate}
\end{exe}

\subsubsection{Compound queries}

\begin{exe}[4.56]
    The queries that retrieve the required information are:
    \begin{enumerate}
        \item \scm{ch4/4.56a.scm}
        \item \scm{ch4/4.56b.scm}
        \item \scm{ch4/4.56c.scm}
    \end{enumerate}
\end{exe}

\subsubsection{Rules}

\begin{exe}[4.57]
    The rule \vscm{can-replace} can be defined as:
    \scm{ch4/4.57tex.scm}
    \begin{enumerate}
        \item The people who can replace Cy D. Fect can be found with the query:
            \scm{ch4/4.57a.scm}
        \item The people who can replace someone who is being paid more than 
            they are can be found thanks to the query:
            \scm{ch4/4.57b.scm}
    \end{enumerate}
\end{exe}

\begin{exe}[4.58]
    The rule can be defined as:
    \scm{ch4/4.58tex.scm}
\end{exe}

\begin{exe}[4.59]
    \ \vspace{-20pt}
    \begin{enumerate}
        \item Ben should use the query:
            \scm{ch4/4.59a.scm}
        \item Alyssa’s rule can be defined as:
            \scm{ch4/4.59b.scm}
        \item Alyssa should run the query:
            \scm{ch4/4.59c.scm}
    \end{enumerate}
\end{exe}

\begin{exe}[4.60]
    Each pair appears twice because the rule’s body is symmetric in the 
    variables \vscm{?person-1} and \vscm{?person-2}, so that if none of them is 
    bound by the query, if a frame where \vscm{?person-1} is bound to person 
    A and \vscm{?person-2} is bound to person B appears in the result, the frame 
    where \vscm{?person-1} is bound to person B and \vscm{?person-2} is bound to 
    person A appears in the result too.

    The obvious idea is to define an order on the variables’ values and modify 
    the rule so that, for instance, the value of \vscm{?person-1} is smaller 
    than that of \vscm{?person-2}. The trouble is that it would not work with 
    queries where one of the variables \vscm{?person-1} or \vscm{?person-2} is 
    already bound: the expected behavior is that the two queries:
    \begin{cscm}
        (lives-near (Hacker Alyssa P) ?x)
        (lives-near ?x (Hacker Alyssa P))
    \end{cscm}
    lead to the same set of possible values for \vscm{?x}, which won’t be the 
    case if \vscm{lives-near} is not symmetric, so the idea of sorting the 
    variables’ values does not work. The solution would involve defining a rule 
    that behaves differently depending on whether one of the variables 
    \vscm{?person-1} and \vscm{?person-2} has a value imposed by the query, 
    which does not seem possible, or at least not easily.
\end{exe}

\subsubsection{Logic as programs}

\begin{exe}[4.61]
    The response to the query \vscm{(?x next-to ?y in (1 (2 3) 4))} is:
    \begin{cscm}
        ((2 3) next-to 4 in (1 (2 3) 4))
        (1 next-to (2 3) in (1 (2 3) 4))
    \end{cscm}
    The response to the query \vscm{(?x next-to 1 in (2 1 3 1)} is:
    \begin{cscm}
        (3 next-to 1 in (2 1 3 1))
        (2 next-to 1 in (2 1 3 1))
    \end{cscm}
\end{exe}

\begin{exe}[4.62]
    \label{4.62}
    The operation can be implemented with the following rules:
    \scm{ch4/4.62.scm}
    They give the expected results with queries such as
    \vscm{(last-pair (3) ?x)},\linebreak
    \vscm{(last-pair (1 2 3) ?x)} or
    \vscm{(last-pair (2 ?x) (3))}.

    With queries such as \vscm{(last-pair ?x (3))}, the behavior depends on the 
    rules’order because of the evaluator’s implementation: if they are entered 
    as written above, the rule corresponding to lists with one element is 
    checked first (since new rules are added at the beginning of the stream of 
    rules and the stream is checked sequentially), and the evaluator displays 
    the elements of an infinite stream of results with unbound variables:
    \begin{cscm}
        (last-pair (3) (3))
        (last-pair (?u-2 3) (3))
        (last-pair (?u-2 ?u-6 3) (3))
        (last-pair (?u-2 ?u-6 ?u-10 3) (3))
        (last-pair (?u-2 ?u-6 ?u-10 ?u-14 3) (3))
        (last-pair (?u-2 ?u-6 ?u-10 ?u-14 ?u-18 3) (3))
        (last-pair (?u-2 ?u-6 ?u-10 ?u-14 ?u-18 ?u-22 3) (3))
        (last-pair (?u-2 ?u-6 ?u-10 ?u-14 ?u-18 ?u-22 ?u-26 3) (3))
        (last-pair (?u-2 ?u-6 ?u-10 ?u-14 ?u-18 ?u-22 ?u-26 ?u-30 3) (3))
        (last-pair (?u-2 ?u-6 ?u-10 ?u-14 ?u-18 ?u-22 ?u-26 ?u-30 ?u-34 3) (3))
        (last-pair (?u-2 ?u-6 ?u-10 ?u-14 ?u-18 ?u-22 ?u-26 ?u-30 ?u-34 ?u-38 3) 
                (3))
        (last-pair (?u-2 ?u-6 ?u-10 ?u-14 ?u-18 ?u-22 ?u-26 ?u-30 ?u-34 ?u-38 
                ?u-42 3) (3))
        (last-pair (?u-2 ?u-6 ?u-10 ?u-14 ?u-18 ?u-22 ?u-26 ?u-30 ?u-34 ?u-38 
                ?u-42 ?u-46 3) (3))
        [...]
    \end{cscm}
    If the rules are entered in reverse order, the rule corresponding to lists 
    of several elements is checked first and the evaluator enters an infinite 
    loop as it tries to build an infinite list ending in 3, so no result is 
    displayed.
\end{exe}

\begin{exe}[4.63]
    The rules can be defined as:
    \scm{ch4/4.63.scm}
\end{exe}

\subsection{How the Query System Works}

This subsection contains no exercises.

\subsection{Is Logic Programming Mathematical Logic?}

\begin{exe}[4.64]
    \label{4.64}
    The evaluator unifies the query with the conclusion of the rule for 
    \vscm{outranked-by} by binding the variable \vscm{?staff-person} to 
    \vscm{(Bitdiddle Ben)} and \vscm{?boss} to \vscm{?who}. It then evaluates 
    the rule’s body and finds the assertion in the database corresponding to 
    Ben’s supervisor. This produces the first frame of the result stream, and an 
    answer corresponding to this frame’s binding is displayed. The interpreter 
    then processes the first part of the \vscm{and} query:
    \vscm{(outranked-by ?middle-manager ?boss)}, with both variables unbound 
    (more precisely, \vscm{?boss} is bound to an unbound variable). The 
    recursive rule evaluation leads again to a recursive evaluation of an 
    \vscm{outranked-by} rule, which leads to an infinite loop.

    With the initial version of \vscm{outranked-by}, when evaluating the 
    compound \vscm{and} query, the query
    \vscm{(supervisor ?staff-person ?boss)}, with \vscm{?staff-person} bound to 
    a constant, would have been evaluated first, and it would have produced 
    either an empty stream, which would have ended the proccessing for the given 
    frame, or a stream containing a single frame with a binding for 
    \vscm{?middle-manager}, which would have caused the processing of \linebreak
    \vscm{(outranked-by ?staff-person ?boss)} with \vscm{?staff-person} bound to 
    the name of someone higher in the hierarchy than in the previous processing 
    of the rule. Since the tower of hierarchy is not infinite, the evaluation of 
    the rule’s body terminates eventually.
\end{exe}

\begin{exe}[4.65]
    Oliver Warbucks is listed four times because the interpreter found four sets 
    of bindings for the variables appearing in the body of the rule for 
    \vscm{wheel} with \vscm{?who} bound to \vscm{(Warbucks Oliver)}. We can 
    understand the result better if we evaluate the rule’s body at the driver 
    loop:
    \begin{cscm}
        (and (supervisor (Scrooge Eben) (Warbucks Oliver))
             (supervisor (Cratchet Robert) (Scrooge Eben)))
        (and (supervisor (Hacker Alyssa P) (Bitdiddle Ben))
             (supervisor (Reasoner Louis) (Hacker Alyssa P)))
        (and (supervisor (Bitdiddle Ben) (Warbucks Oliver))
             (supervisor (Tweakit Lem E) (Bitdiddle Ben)))
        (and (supervisor (Bitdiddle Ben) (Warbucks Oliver))
             (supervisor (Fect Cy D) (Bitdiddle Ben)))
        (and (supervisor (Bitdiddle Ben) (Warbucks Oliver))
             (supervisor (Hacker Alyssa P) (Bitdiddle Ben)))
    \end{cscm}
\end{exe}

\begin{exe}[4.66]
    Ben has just realized that some values could appear multiple times in the 
    result stream. For instance, if we try to sum all the wheels’ salaries 
    using:
    \begin{cscm}
        (sum ?amount
             (and (wheel ?w)
                  (salary ?w ?amount)))
    \end{cscm}
    the result will be the sum of Ben’s salary and four times Oliver Warbucks’ 
    salary instead of being the sum of Ben’s and Oliver Warbucks’ salaries.

    Ben could filter the frames so that each possible set of values for the 
    variables appearing in the query appears only once in the filtered stream 
    before extracting the value of the designated variable. With the example 
    above, the unfiltered stream contains five frames. In one of them, \vscm{?w} 
    is bound to \vscm{(Bitdiddle Ben)} and \vscm{?amount} is bound to 
    \vscm{60000}, and in four of them \vscm{?w} is bound to
    \vscm{(Warbucks Oliver)} and \vscm{?amount} is bound to \vscm{150000}, but 
    the values of some other variables bound in these frames differ. The 
    filtered stream would contain only one frame for Ben Bitdiddle and one for 
    Oliver Warbucks.

    Keeping unique frames would not work because in the example above, the five 
    frames in the result stream are distinct if we take all the bindings they 
    define into account. Keeping unique values of the extracted variable would 
    not work either because there can be legitimate duplicates: several 
    employees can have the same salary for instance.
\end{exe}

\begin{exe}[4.67]
    A possible way to detect loops it to define a global history that is reset 
    at the beginning of \vscm{query-driver-loop}. This history contains pairs 
    with a query and a frame containing the bindings defined before the given 
    query was processed. We modify \vscm{simple-query} to check whether the 
    given query has already been processed before going forward. If a loop is 
    detected, \vscm{simple-query} returns the empty stream. Otherwise it adds an 
    entry to the history before computing the result.

    To check whether a query has already been processed, we check whether the 
    instantiation of the query in the current frame (with the numbers in unbound 
    variables removed) is equal to the instantiation of the stored query in the 
    associated stored frame (with the numbers in unbound variables removed as 
    well). If that’s the case, we check whether the stored frame is reachable 
    from the current frame: if that’s not the case, the stored history entry 
    does not belong to the current chain of deductions so we would detect a loop 
    where there is none and lose potential results.

    The problem with this implementation is that, though it does prevent 
    infinite loops, it also cuts off some computations that would have 
    terminated. For instance, with the version of \vscm{outranked-by} from 
    exercise~\ref{4.64}, when evaluating
    \vscm{(outranked-by (Reasoner Louis) ?x)}, an answer with Alyssa P.~Hacker 
    is found from the first branch of the \vscm{or}. Then the recursive 
    evaluation of \vscm{outranked-by} (which is not eliminated because both 
    variables are unbound whereas one of them was bound in the query) allows the 
    interpreter to find that Louis is outranked by Ben Bitdiddle. But the second 
    recursion on \vscm{outranked-by} with both variables unbound is eliminated, 
    so the third answer with Oliver Warbucks is not found. The evaluation of 
    \vscm{(outranked-by ?x ?y)} is even worse since the second branch is 
    completely eliminated so we only get the supervisors. With \vscm{last-pair} 
    from exercise~\ref{4.62}, where we got an infinite stream of results by 
    evaluating \vscm{(last-pair ?x (3))}, we only get the first two results 
    because the rest of the computation is eliminated.

    I spent some time trying to find a better solution without much success: 
    either the system cut too much or not enough. After a look at some research 
    articles on the subject, it seems that it is really difficult to find 
    a system that both eliminates all infinite loops and does not eliminate 
    valid computations. From what I understood (which is not much), it is 
    certainly not easily feasible, so I decided to stick with this version.

    The modified implementation of \vscm{simple-query} and the other procedures 
    needed can be defined as:
    \scm{ch4/4.67.scm}

    It’s also necessary to add \vscm{(reset-history!)} at the beginning of 
    \vscm{query-driver-loop}.
\end{exe}

\begin{exe}[4.68]
    My initial rules were:
    \scm{ch4/4.68atex.scm}
    These rules can answer \vscm{(reverse (1 2 3) ?x)} but not
    \vscm{(reverse ?x (1 2 3))} because in this case
    \vscm{(reverse ?rest ?reverse-rest)} is evaluated with both variables 
    unbound and it leads to an infinite loop.

    The second query can be solved by the rule:
    \scm{ch4/4.68btex.scm}
    but then it’s the first one that leads to an infinite loop.

    If we add the infinite loop detector from the previous exercise and use all 
    three rules above, both queries are solved, and also queries like
    \vscm{(reverse (1 2 ?x) (3 . ?y))} and
    \vscm{(reverse (1 2  . ?x) (3 . ?y))}. However,
    \vscm{(reverse (1 2 . ?x) (4 3 . ?y))} returns no answer because the 
    computation is interrupted by the loop detector.
\end{exe}

\begin{exe}[4.69]
    The rules can be defined in the following way:
    \scm{ch4/4.69.scm}
    Without the loop detector, queries such as \vscm{(?rel Adam Irad)} cause an 
    infinite loop. With the loop detector, there are no infinite loops but 
    queries such as \vscm{((great great grandson) ?x ?y)} return no result 
    (though \vscm{((great grandson) ?x ?y)} works).
\end{exe}

\subsection{Implementing the Query System}

\begin{exe}[4.70]
    It’s necessary to use \vscm{let} because the second argument of 
    \vscm{cons-stream} is evaluated lazily, which implies that after evaluating
    \vscm{(set! THE-ASSERTIONS (cons-stream assertion THE-ASSERTIONS))}, 
    \vscm{THE-ASSERTIONS} is an infinite stream and all its elements are 
    \vscm{assertion}.
\end{exe}

\begin{exe}[4.71]
    The use of explicit delays postpones the apparition of some infinite loops 
    caused by the rules’ evaluation and allows the interpreter to display some 
    answers. For instance, with explicit delays, the query
    \vscm{(married Minnie ?who)} displays an infinite stream of answers, but 
    with the simpler version of \vscm{simple-query} no answer is displayed 
    because the infinite loop appears during the evaluation of the second 
    argument of \vscm{stream-append}. In the same way, with Louis’ version of 
    \vscm{outranked-by} from exercise~\ref{4.64}, with explicit delays the query
    \vscm{(outranked-by (Bitdiddle Ben) ?x)} displays an answer before going 
    into a loop. With the simple version of \vscm{disjoin} no answer is 
    displayed because the loop happens during the evaluation of the second 
    argument of \vscm{interleave}.
\end{exe}

\begin{exe}[4.72]
    If one of the streams resulting from the evaluation of a disjunct or from 
    the application of the mapped function to the stream of frames is infinite, 
    with append no element of any ulterior stream would appear in the result. 
    With \vscm{interleave} we are guaranteed that any element from a partial 
    result stream will appear in the global merged stream eventually.
\end{exe}

\begin{exe}[4.73]
    Without \vscm{delay}, \vscm{flatten-stream} is called on the input stream’s 
    \vscm{cdr} before any element of the flattened stream can be displayed, so 
    if the value of \vscm{(flatten-stream (stream-cdr stream))} is infinite 
    (either because \vscm{stream} is infinite or because one of its elements is 
    an infinite stream), \vscm{flatten-stream} will never return.
\end{exe}

\begin{exe}[4.74]
    \ \vspace{-20pt}
    \begin{enumerate}
        \item The program can be completed simply by filtering out the empty 
            streams and taking the first and only element of non-empty streams:
            \scm{ch4/4.74a.scm}
        \item The query system’s behavior does not change.
    \end{enumerate}
\end{exe}

\begin{exe}[4.75]
    The \vscm{uniquely-asserted} procedure can be defined as:
    \scm{ch4/4.75.scm}
    The query that lists all people who supervise precisely one person is 
    similar to the query that lists all jobs that are filled by only one person:
    \scm{ch4/4.75b.scm}
\end{exe}

\begin{exe}[4.76]
    We can define a \vscm{merge-frames} procedure that merges two frames as 
    indicated in the text, a \vscm{merge-frame-streams} procedure that applies 
    \vscm{merge-frames} to each possible pair of frames from two input streams, 
    and rewrite \vscm{conjoin} using these two procedures.
    \scm{ch4/4.76.scm}
    There are problems caused by the separate evaluation of the clauses of the 
    \vscm{and} however: we have seen previously that in some cases, the 
    evaluation of the second clause works correctly only because some bindings 
    have been provided by the evaluation of the previous clause, such as in the 
    \vscm{outranked-by} rule, or when using \vscm{not} or \vscm{lisp-value}. 
    This new implementation won’t give the expected results in such cases, it 
    works only when the clauses can be evaluated in any order.
\end{exe}

\begin{exe}[4.77]
    First, let’s define a procedure to check whether a query contains an unbound 
    variable in a given frame:
    \scm{ch4/4.77a.scm}
    Then I modified the procedures \vscm{negate} and \vscm{lisp-value} so that 
    they check if the query contains unbound variables. If so, they add 
    a promise to filter to the frame. If not, they do the filtering directly. 
    A promise is a pair consisting of a query and a predicate applying to 
    a frame that returns true if the frame must be kept and false if it must be 
    filtered out. It is added to the frame as a special binding. Since 
    \vscm{not} and \vscm{lisp-value} queries are necessarily part of an 
    \vscm{and} expression, I then modified \vscm{conjoin} so that it filters out 
    the stream of frames produced by the evaluation of the first subquery if any 
    promises are found. The disadvantage of this method is that if the frame is 
    kept, the same promise will be checked multiple times.
    \scm{ch4/4.77b.scm}
\end{exe}

\begin{exe}[4.78]
    To implement the query language as a nondeterministic program, we first 
    replace the driver loop with two procedures: \vscm{assert!} to add 
    assertions and rules to the database and \vscm{request} to run queries.
    \scm{ch4/4.78b.scm}
    All the procedures on frame streams now operate on a single frame. 
    \vscm{Qeval} is unchanged except that the \vscm{frame-stream} argument will 
    now be a single frame, but \vscm{simple-query}, \vscm{conjoin}, 
    \vscm{disjoin}, \vscm{negate} and \vscm{lisp-value} have to be modified:
    \scm{ch4/4.78c.scm}
    For \vscm{negate}, we need to succeed if the evaluation of the negated query 
    fails without producing any result, and I found no other way to do that than 
    to define a new special form \vscm{require-fail} that succeeds (with the 
    value \vscm{true}) if its argument fails:
    \scm{ch4/4.78a.scm}
    The procedures used for pattern matching and unification must be adapted as 
    well. By using \vscm{amb} instead of returning the \vscm{failed} symbol in 
    \vscm{pattern-match} and \vscm{unify-match}, we can simplify the procedures 
    a little since it’s not necessary to check for the \vscm{failed} symbol 
    anymore.
    \scm{ch4/4.78d.scm}
    I also rewrote the code for maintaining the data base so that it uses lists 
    instead of streams (not included here).

    \medskip

    There are differences in the answers’ order due to my use of \vscm{ramb} in 
    \vscm{disjoin} where the stream-based version used 
    \vscm{interleave-delayed}.

    The main differences regard infinite loops. Because I used \vscm{ramb} in 
    \vscm{disjoin}, elements from all the disjuncts will appear eventually but 
    it’s impossible to predict when. If the computation of one element causes 
    an infinite loop, it may appear at the start or after some answers have 
    been displayed. For instance, with Louis’ version of \vscm{outranked-by} 
    from exercise~\ref{4.64}, sometimes the query
    \vscm{(outranked-by (Bitdiddle Ben) ?x)} loops without displaying anything, 
    sometimes it answers first. But most of the time
    \vscm{(outranked-by ?x ?y)} displays several answers before going into 
    a loop, whereas the stream-based version always displayed only the first 
    answer.

    With the query \vscm{(last-pair ?x (3))} from exercise~\ref{4.62}, the 
    successive results displayed show an increment of one only in the ids of the 
    unbound variables, where the increment was of 4 in the stream version. The 
    first results are:
    \begin{cscm}
	(last-pair (3) (3))
	(last-pair (?u-1 3) (3))
	(last-pair (?u-1 ?u-2 3) (3))
	(last-pair (?u-1 ?u-2 ?u-3 3) (3))
	(last-pair (?u-1 ?u-2 ?u-3 ?u-4 3) (3))
    \end{cscm}
    The reason is that the rule-counter is decreased when the evaluator 
    backtracks after a rule application fails (there are 4 rules in the test 
    database).
\end{exe}

\chapter{Computing with Register Machines}

\section{Designing Register Machines}

\begin{exe}[5.1]
    \label{5.1}
    The data-path and the controller diagrams for the iterative factorial 
    machine are shown on figure \ref{5.1fig}.

    \begin{figure}
        \centering
        \begin{tikzpicture}[>=Stealth]
            % Data-path diagram.
            \matrix[data matrix] (dp) {
                & \node[reg] (n) {n}; &
                \node[test] (>) {>}; \\

                & \node[const] (c1) {1}; & \\

                \node[reg] (p) {product}; &&
                \node[reg] (c) {counter}; \\

                & \node[op] (+) {+}; & \\

                & \node[op] (*) {*}; & \\
            };

            \draw[arg] (n) -- (>);
            \draw[arg] (c) -- (>);
            \draw[button=0.82] (c1) -| node[near start, above] {p<-1} (p);
            \draw[button=0.82] (c1) -| node[near start, above] {c<-1}
                ($ (c.north) + (-1em, 0) $);
            \draw[arg] (c1) -- (+);
            \draw[button=0.8] ($ (+.north) + (1em, 0) $) |-
                node[near end, above] {c++} (c);
            \draw[arg] (c) |- (+);
            \draw[arg] ($ (p.south) + (-1em, 0) $) |-
                ($ (*.south) + (0, -1.5em) $) -- (*);
            \draw[arg] (c) |- (*);
            \draw[button=0.8] (*) -| node[near end, right]{p<-*}
                ($ (p.south) + (1em, 0) $);

            % Controller diagram
            \matrix[controller matrix, right=5em of dp] {
                \node[cio] (s) {start}; \\
                \node[cbutton] (pi) {p<-1}; \\
                \node[cbutton] (ci) {c<-1}; \\
                \node[ctest] (ct) {>}; & \node[cio] (cd) {done}; \\[+1em]
                \node[cbutton] (cp) {p<-*}; \\
                \node[cbutton] (cc) {c++}; \\
            };

            \draw[flow] (s) -- (pi);
            \draw[flow] (pi) -- (ci);
            \draw[flow] (ci) -- (ct);
            \draw[flow] (ct) --node[right] {no} (cp);
            \draw[flow] (ct) --node[above] {yes} (cd);
            \draw[flow] (cp) -- (cc);
            \draw[flow] (cc) -| ($ (ct.west) - (2.5em, 0) $) -- (ct);
        \end{tikzpicture}
        \caption{The data-path and controller diagrams for the iterative 
        factorial machine.}
        \label{5.1fig}
    \end{figure}
\end{exe}

\subsection{A Language for Describing Register Machines}

\begin{exe}[5.2]
    Anticipating on the next section to use the register-machine simulator, we 
    can define the iterative factorial machine of exercise \ref{5.1} as:
    \scm{ch5/5.02.scm}
\end{exe}

\subsection{Abstraction in Machine Design}

\begin{exe}[5.3]
    Using the simulator again, the first version of the register machines can be 
    defined as:
    \scm{ch5/5.03a.scm}
    and the second version as:
    \scm{ch5/5.03b.scm}
    The data-path diagrams are shown on figures \ref{5.03afig} and 
    \ref{5.03bfig} respectively.

    \begin{figure}
        \centering
        \begin{tikzpicture}[>=Stealth]
            \matrix[data matrix] {
                \node[const] (c1) {1};
                &[+1em] \node[reg] (g) {guess};
                & \node[test] (ge) {g-e?};
                & \node[reg] (x) {x}; \\

                && \node[op] (i) {improve}; \\
            };

            \draw[button=.5] (c1) -- (g);
            \draw[arg] (g) -- (ge);
            \draw[arg] (x) -- (ge);
            \draw[arg] ($ (g.south) + (1em, 0) $) |- (i);
            \draw[button=.7] (i.south) -- ($ (i.south) - (0, 1em) $) -|
                ($ (g.south) - (1em, 0) $);
            \draw[arg] (x) |- (i);
        \end{tikzpicture}
        \caption{The data-path diagram for the square root machine using complex 
        primitive operations.}
        \label{5.03afig}
    \end{figure}

    \begin{figure}
        \centering
        \begin{tikzpicture}[>=Stealth]
            \matrix[data matrix, matrix of nodes, nodes in empty cells]
            (table) {%
                && \node[const] (c1) {1};
                & \node[op] (div2) {/};
                & \node[const] (c2) {2}; \\

                &&& \node[minimum height=1.5\regheight] {guess}; &&& \\

                && \node[op] (divx) {/};
                & \node[op] (*) {*};
                & \node[op] (+) {+}; \\

                & \node[reg] (x) {x};
                && \node {tmp};
                && \node[test] (<eps) {<};
                & \node[const, inner xsep = -.2em] (eps) {0.001}; \\

                && \node[op] (mx) {-};
                & \node[op] (mtmp) {-};
                & \node[test] (>0) {<};
                & \node[const] (c0) {0}; \\
            };
            \begin{scope}[on background layer]
                \node[reg, fit=(table-2-3)(table-2-5)] (guess) {};
                \node[reg, fit=(table-4-3)(table-4-5)] (tmp) {};
            \end{scope}

            \draw[button=.6] (c1) -- (guess.north -| c1);
            \draw[arg] ($ (guess.north) - (3em, 0) $) |- (div2);
            \draw[arg] (c2) -- (div2);
            \draw[button=.6] (div2) -- (guess);
            \draw[arg] (x) |- (divx);
            \draw[arg] (guess.south -| divx) -- (divx);
            \draw[button=.6] (divx) -- (tmp.north -| divx.south);
            \draw[arg] ($ (guess.south) - (1em, 0) $) --
                ($ (*.north) - (1em, 0) $);
            \draw[arg] ($ (guess.south) + (1em, 0) $) --
                ($ (*.north) + (1em, 0) $);
            \draw[button=.6] (*) -- (tmp);
            \coordinate (pc) at (guess.south -| +.north);
            \draw[arg] ($ (pc) - (1em, 0) $) -- ($ (+.north) - (1em, 0) $);
            \draw[button=.6] ($ (+.north) + (1em, 0) $) -- ($ (pc) + (1em, 0) $);
            \draw[arg] (tmp.north -| +) -- (+);
            \draw[arg] (tmp) -- (<eps);
            \draw[arg] (eps) -- (<eps);
            \draw[arg] (x) |- (mx);
            \coordinate (mc) at (tmp.south -| mx.north);
            \draw[arg] ($ (mc) - (1em, 0) $) -- ($ (mx.north) - (1em, 0) $);
            \draw[button=.6] ($ (mx.north) + (1em, 0) $) --
                ($ (mc) + (1em, 0) $);
            \draw[arg] ($ (tmp.south) - (1em, 0) $) --
                ($ (mtmp.north) - (1em, 0) $);
            \draw[button=.6] ($ (mtmp.north) + (1em, 0) $) --
                ($ (tmp.south) + (1em, 0) $);
            \draw[arg] (tmp.south -| >0) -- (>0);
            \draw[arg] (c0) -- (>0);
        \end{tikzpicture}
        \caption{The data-path diagram for the square root machine using only 
        basic primitive operations.}
        \label{5.03bfig}
    \end{figure}
\end{exe}

\subsection{Subroutines}

This subsection contains no exercises.

\subsection{Using a Stack to Implement Recursion}

\begin{exe}[5.4]
    \label{5.04}
    \ \vspace{-20pt}
    \begin{itemize}
        \item[a.] The recursive exponentiation machine can be defined as 
            follows. The corresponding data-path diagram is shown on figure 
            \ref{5.04afig}.
            \scm{ch5/5.04a.scm}
        \item[b.] The iterative exponentiation machine can be defined as 
            follows. The corresponding data-path diagram is shown on figure 
            \ref{5.04bfig}.
            \scm{ch5/5.04b.scm}
    \end{itemize}

    \begin{figure}
        \centering
        \begin{tikzpicture}[>=Stealth]
            \matrix[data matrix] {
                \node[const] (c0) {0};
                & \node[test] (=) {=};
                & \node[reg] (n) {n};
                &[+1em] &[-4.5em] \node[reg] (stack) {stack}; \\

                \node[const] (c1) {1};
                & \node[op] (-) {-};
                &&& \node[reg] (continue) {continue};
                &[-4.5em]& \node (controller) {controller}; \\

                \node[reg] (val) {val};
                & \node[op] (*) {*};
                & \node[reg] (b) {b};
                & \node[const, inner xsep=-.2em] (ed) {e-d};
                && \node[const, inner xsep=-.2em] (ae) {a-e}; \\
            };

            \draw[arg] (c0) -- (=);
            \draw[arg] (n) -- (=);
            \draw[button=.5] ($ (n.east) + (0, .7em) $) --
                ($ (stack.west) + (0, .7em) $);
            \draw[button=.5] ($ (stack.west) - (0, .7em) $) --
                ($ (n.east) - (0, .7em) $);
            \draw[arg] (c1) -- (-);
            \draw[arg] ($ (n.south) - (.7em, 0) $) |- (-);
            \draw[button=.8] (-) -- ($ (-.south) - (0, 1em) $)
                -| ($ (n.south) + (.7em, 0) $);
            \draw[button=.5] ($ (stack.south) - (.7em, 0) $) --
                ($ (continue.north) - (.7em, 0) $);
            \draw[button=.5] ($ (continue.north) + (.7em, 0) $) --
                ($ (stack.south) + (.7em, 0) $);
            \draw[arg] (continue) -- (controller);
            \draw[arg] (c1) -- (val);
            \draw[arg] (val) -- (*);
            \draw[button=.7] (*) -- ($ (*) - (0, 3em) $) -| (val);
            \draw[arg] (b) -- (*);
            \draw[button=.5] (ed) -- (ed |- continue.south);
            \draw[button=.5] (ae) -- (ae |- continue.south);
        \end{tikzpicture}
        \caption{The data-path diagram for the recursive exponentiation 
        machine.}
        \label{5.04afig}
    \end{figure}

    \begin{figure}
        \centering
        \begin{tikzpicture}[>=Stealth]
            \matrix[data matrix] {
                \node[reg] (b) {b};
                & \node[const] (c0) {0};
                &[+1em] \node[op] (=) {=}; \\

                \node[op] (*) {*};
                & \node[reg] (n) {n};
                & \node[reg] (c) {counter}; \\

                \node[reg] (p) {product};
                & \node[const] (c1) {1};
                & \node[op] (-) {-}; \\
            };

            \draw[arg] (c0) -- (=);
            \draw[arg] (c) -- (=);
            \draw[arg] (b) -- (*);
            \draw[arg] ($ (p.north) - (.7em, 0) $) --
                ($ (*.south) - (.7em, 0) $);
            \draw[button=.5] ($ (*.south) + (.7em, 0) $) --
                ($ (p.north) + (.7em, 0) $);
            \draw[button=.5] (n) -- (c);
            \draw[arg] (c1) -- (p);
            \draw[arg] (c1) -- (-);
            \draw[arg] ($ (c.south) - (1em, 0) $) --
                ($ (-.north) - (1em, 0) $);
            \draw[button=.5] ($ (-.north) + (1em, 0) $) --
                ($ (c.south) + (1em, 0) $);
        \end{tikzpicture}
        \caption{The data-path diagram for the iterative exponentiation 
        machine.}
        \label{5.04bfig}
    \end{figure}
\end{exe}

\begin{exe}[5.5]
    The following table lists the instructions evaluated during the simulation 
    of factorial 3, with their effect on the values of the registers \vscm{n}, 
    \vscm{val}, and \vscm{continue} and on the stack.
    \begin{longtable}{|l|c|c|c|c|}
        \hline
        \bfseries Instruction & \bfseries n & \bfseries val & \bfseries continue 
        & \bfseries stack \\\hline
        \endhead
        (assign cont (label fact-done)) & 3 & *unassigned* & fact-done & () 
        \\\hline
        (test (op =) ...) -> false &&&& \\\hline
        (branch (label base-case)) &&&& \\\hline
        (save continue) &&&& (fact-done) \\\hline
        (save n) &&&& (3 fact-done) \\\hline
        (assign n (op -) ...) & 2 &&& \\\hline
        (assign cont (label after-fact)) &&& after-fact & \\\hline
        (goto (label fact-loop)) &&&& \\\hline
        (test (op =) ...) -> false &&&& \\\hline
        (branch (label base-case)) &&&& \\\hline
        (save cont) &&&& (after-fact 3 fact-done) \\\hline
        (save n) &&&& (2 after-fact 3 fact-done) \\\hline
        (assign n (op -) ...) & 1 &&& \\\hline
        (assign cont (label after-fact)) &&& after-fact & \\\hline
        (goto (label fact-loop)) &&&& \\\hline
        (test (op =) ...) -> true &&&& \\\hline
        (branch (label base-case)) &&&& \\\hline
        (assign val (const 1)) && 1 && \\\hline
        (goto (reg cont)) -> after-fact &&&& \\\hline
        (restore n) & 2 &&& (after-fact 3 fact-done) \\\hline
        (restore cont) &&& after-fact & (3 fact-done) \\\hline
        (assign val (op *) ...) && 2 && \\\hline
        (goto (reg cont)) -> after-fact &&&& \\\hline
        (restore n) & 3 &&& (fact-done) \\\hline
        (restore cont) &&& fact-done & () \\\hline
        (assign val (op *) ...) && 6 && \\\hline
        (goto (reg cont)) -> fact-done &&&& \\\hline
    \end{longtable}

    The following table lists the instructions evaluated during the simulation 
    of fibonacci 3, with their effect on the values of the registers \vscm{n}, 
    \vscm{val}, and \vscm{continue} and on the stack.
    \begin{longtable}{|l|c|c|c|c|}
        \hline
        \bfseries Instruction & \bfseries n & \bfseries val & \bfseries continue 
        & \bfseries stack \\\hline
        \endhead
        (assign cont (label fib-done) & 3 & *unassigned* & fib-done & () 
        \\\hline
        (test (op <) ...) -> false &&&& \\\hline
        (branch (label imm-answer)) &&&& \\\hline
        (save cont) &&&& (fib-done) \\\hline
        (assign cont (label after-fib-n-1)) &&& after-fib-n-1 & \\\hline
        (save n) &&&& (3 fib-done) \\\hline
        (assign n (op -) ...1) & 2 &&& \\\hline
        (goto (label fib-loop)) &&&& \\\hline
        (test (op <) ...) -> false &&&& \\\hline
        (branch (label imm-answer)) &&&& \\\hline
        (save cont) &&&& (after-fib-n-1 3 fib-done) \\\hline
        (assign cont (label after-fib-n-1)) &&& after-fib-n-1 & \\\hline
        (save n) &&&& (2 after-fib-n-1 3 fib-done) \\\hline
        (assign n (op -) ...1) & 1 &&& \\\hline
        (goto (label fib-loop)) &&&& \\\hline
        (test (op <) ...) -> true &&&& \\\hline
        (branch (label imm-answer)) &&&& \\\hline
        (assign val (reg n)) && 1 && \\\hline
        (goto (reg cont)) -> after-fib-n-1 &&&& \\\hline
        (restore n) & 2 &&& (after-fib-n-1 3 fib-done) \\\hline
        (restore cont) &&& after-fib-n-1 & (3 fib-done) \\\hline
        (assign n (op -) ...2) & 0 &&& \\\hline
        (save cont) &&&& (after-fib-n-1 3 fib-done) \\\hline
        (assign cont (label after-fib-n-2)) &&& after-fib-n-2 & \\\hline
        (save val) &&&& (1 after-fib-n-1 3 fib-done) \\\hline
        (goto (label fib-loop)) &&&& \\\hline
        (test (op <) ...) -> true &&&& \\\hline
        (branch (label imm-answer)) &&&& \\\hline
        (assign val (reg n)) && 0 && \\\hline
        (goto (reg cont)) -> after-fib-n-2 &&&& \\\hline
        (assign n (reg val)) & 0 &&& \\\hline
        (restore val) && 1 && (after-fib-n-1 3 fib-done) \\\hline
        (restore cont) &&& after-fib-n-1 & (3 fib-done) \\\hline
        (assign val (op +) ...) && 1 && \\\hline
        (goto (reg cont)) -> after-fib-n-1 &&&& \\\hline
        (restore n) & 3 &&& (fib-done) \\\hline
        (restore cont) &&& fib-done & () \\\hline
        (assign n (op -) ... 2) & 1 &&& \\\hline
        (save cont) &&&& (fib-done) \\\hline
        (assign cont (label after-fib-n-2)) &&& after-fib-n-2 & \\\hline
        (save val) &&&& (1 fib-done) \\\hline
        (goto (label fib-loop)) &&&& \\\hline
        (test (op <) ...) -> true &&&& \\\hline
        (branch (label imm-answer)) &&&& \\\hline
        (assign val (reg n)) && 1 && \\\hline
        (goto (reg cont)) -> after-fib-n-2 &&&& \\\hline
        (assign n (reg val)) & 1 &&& \\\hline
        (restore val) && 1 && (fib-done) \\\hline
        (restore cont) &&& fib-done & () \\\hline
        (assign val (op +) ...) && 2 && \\\hline
        (goto (reg cont)) -> fib-done &&&& \\\hline
    \end{longtable}
\end{exe}

\begin{exe}[5.6]
    In \vscm{after-fib-n-1}, the instructions \vscm{(restore continue)} and 
    \vscm{(save continue)} can be removed because no change is done to the 
    \vscm{continue} register or to the stack between them.
\end{exe}


\end{document}
