\documentclass{scrreprt}

% Language-specific settings
\usepackage{polyglossia}
\setdefaultlanguage{american}

% Fonts
\usepackage{fontspec,microtype}
\setmainfont{Linux Libertine O}
\setsansfont{Linux Libertine O}
\setmonofont[Scale=MatchLowercase]{FantasqueSansMono-Regular.ttf}

% Hyperlinks
\usepackage{xcolor}
\definecolor{darkblue}{rgb}{0,0,0.5}
\definecolor{darkgreen}{rgb}{0,0.3,0}
\usepackage[unicode,
pdftitle={Solutions to Structure and Interpretation of Computer Programs},
pdfauthor={Fabienne Ducroquet},
urlcolor=darkblue,
linkcolor=darkgreen,
breaklinks,
colorlinks]{hyperref}

% Headers
\usepackage{fancyhdr}
\pagestyle{fancy}
\fancyhead[RO,LE]{\thepage}
\fancyhead[LO]{\rightmark}
\fancyhead[RE]{\leftmark}

% Source code highlighting
\usepackage{minted}
\newminted[cscm]{scheme}{autogobble,style=lovelace}
\newmintedfile[scm]{scheme}{style=lovelace}
\newmintinline[vscm]{scheme}{style=bw}

% Show subsubsections in the table of contents.
\setcounter{tocdepth}{4}

% Counter needed to link to the right pages in the online version of the book on 
% the MIT’s website.
\newcounter{mitpage}
\setcounter{mitpage}{6}
\let\oldsection\section
\renewcommand\section{\stepcounter{mitpage}\oldsection}
\let\oldchapter\chapter
\renewcommand\chapter{\stepcounter{mitpage}\oldchapter}

% Theorem environments
\usepackage{amssymb,amsmath,amsthm}
\newtheoremstyle{example}{}{}{}{}{\itshape}{:}{ }
{\thmname{#1}\thmnumber{ #2}\thmnote{ (#3)}}
\theoremstyle{example}
\newtheorem*{example}{Example}

\newtheoremstyle{remark}{}{}{}{}{\bfseries}{.}{ }
{\thmname{#1}\thmnumber{ #2}\thmnote{ (#3)}}
\theoremstyle{remark}
\newtheorem*{remark}{Remark}

\newtheoremstyle{comp}{}{}{\small}{}{\itshape}{:}{\newline}
{\thmname{#1}\thmnumber{ #2}\thmnote{ (#3)}}
\theoremstyle{comp}
\newtheorem*{comp}{Complement}

\newtheoremstyle{sicp-exe}{}{}{}{-12pt}{\bfseries}{}{\newline}
{\href{https://mitpress.mit.edu/sicp/full-text/book/book-Z-H-\themitpage.html\thmnumber{\#\%_thm_#2}}{\thmname{#1}\thmnumber{ #2}\thmnote{ (#3)}}}
\theoremstyle{sicp-exe}
\newtheorem{exeaux}{Exercise}[chapter]

% Include the exercises in the table of contents.
\NewDocumentEnvironment{exe}{o}
{\IfNoValueTF{#1}
    {\exeaux\addcontentsline{toc}{paragraph}%
        {Exercise}}
    {\exeaux\addcontentsline{toc}{paragraph}%
        {Exercise #1}}%
        \ignorespaces}
{\endexeaux}

% For inline fractions
\usepackage{xfrac}

% For Fibonacci numbers.
\DeclareMathOperator{\Fib}{Fib}

% For tables.
\usepackage{tabls, array, longtable}

% For exercise 2.24, and some exercises in chapter 3
\usepackage{tikz}
\usetikzlibrary{positioning, arrows.meta, calc, fit, matrix, backgrounds, 
decorations.markings, shapes.misc, shapes.geometric}
\newcommand\var[2]{%
    \newlength{#1}
    \setlength{#1}{#2}
}

% Definitions for the box-and-pointer diagrams.
\var{\boxsize}{2em}
\var{\round}{3pt}

\tikzset{
    box/.style={
        minimum size = \boxsize,
        rectangle,
        rounded corners=\round,
        draw,
        fill=brown!15,
    },
    struct name/.style={
        minimum height=\boxsize,
    },
    car/.style={
        minimum size = \boxsize,
        append after command={
        \pgfextra
        \draw[sharp corners,fill=brown!15]
            (\tikzlastnode.south)
            -| (\tikzlastnode.east)
            |- (\tikzlastnode.north)
            [rounded corners=\round]
            -| (\tikzlastnode.west)
            |- (\tikzlastnode.south east);
        \endpgfextra},
    },
    cdr/.style={
        minimum size = \boxsize,
        append after command={
        \pgfextra
            \draw[sharp corners,fill=brown!15] (\tikzlastnode.south)
                -| (\tikzlastnode.west) |- (\tikzlastnode.north)
                [rounded corners=\round] -| (\tikzlastnode.east)
                |- (\tikzlastnode.south west);
        \endpgfextra},
    },
    cell matrix/.style={
        row sep = 1.2\boxsize,
        column sep = {\boxsize,between origins}
    },
    box and pointer/.style={
        very thick,
        font = \ttfamily,
        decoration = {markings,
            mark = at position 0 with { \fill circle (3pt); }
        },
    },
    pointer/.style = {
        -Stealth
    },
    box pointer/.style = {
        pointer,
        postaction = decorate
    },
    nil/.style = {
        Triangle Cap[]-,
        shorten >=.8pt,
    }
}

\newcommand\nil[1]{
    \draw[nil] (#1.south west) -- (#1.north east);
}
\newcommand\link[2]{
    \draw[box pointer] (#1.base) -- (#2);
}
% Box-and-pointer end.

% Tikz styles and commands for the environments of section 3.2.
\var{\nametoenv}{4mm}
\tikzset{
    env/.style={
        rectangle,
        rounded corners=2pt,
        % border
        very thick,
        draw=teal!80!black,
        % filling
        fill=teal!10,
        inner sep = 2mm,
        font=\ttfamily\small,
        align=left,
    },
    global env/.style={
        env,
        align=left,
        inner xsep=1ex,
        text width=13cm-2*\pgfkeysvalueof{/pgf/inner xsep},
        minimum width=13cm,
        minimum height=1.5cm
    },
    code/.style={
        align=left,
        font=\ttfamily\small,
    }
}

\var{\circleradius}{3mm}
\var{\smallcircleradius}{.8mm}
\newcommand\pointer[1]{
    \draw (#1) circle (\circleradius);
    \filldraw (#1) circle (\smallcircleradius);
}
\newcommand\procedure[3]{
    \coordinate[left=\circleradius of #1]  (#2);
    \coordinate[right=\circleradius of #1]  (#3);
    \pointer{#2}
    \pointer{#3}
}
% Environments end.

% Settings for timing diagrams in section 3.4.
\tikzset{
    bank/.style={
        circle,
        draw,
        minimum size=1.2cm,
        fill=orange!10,
    },
    action/.style={
        rounded rectangle,
        draw,
        minimum height=1.5em,
        fill=blue!5,
    },
    time matrix/.style={
        column sep=1em,
    },
    arrow/.style={
        rounded corners=10pt,
        thick,
        ->,
    },
}

% Settings for the data-path and controller diagrams in section 5.1.
\var{\regheight}{1cm}
\var{\push}{1.5ex}
\tikzset{
    % Node matrix
    data matrix/.style={
        column sep=1.5em,
        row sep=2.5em,
        nodes = { anchor = center, align = center },
    },
    controller matrix/.style={
        data matrix,
        column sep = 3em,
    },
    % Nodes
    data/.style={
        draw,
        thick,
        rounded corners=2pt,
        minimum height=\regheight,
    },
    reg/.style={
        data,
        rectangle,
        inner xsep=.5cm,
        fill=gray!12!blue!5,
    },
    const/.style={
        data,
        regular polygon,
        regular polygon sides=3,
        fill=gray!12!green!5,
        minimum height=1.7\regheight,
    },
    op/.style={
        data,
        trapezium,
        trapezium left angle=120,
        trapezium right angle=120,
        fill=pink!22!yellow!20
    },
    test/.style={
        data,
        circle,
        fill=gray!12!pink!15,
    },
    % Controller diagrams
    cio/.style={
        thick,
        rounded corners=2pt,
    },
    ctest/.style={
        cio,
        draw,
        inner xsep=.4cm,
        diamond,
        fill=gray!12!pink!15,
    },
    cbutton/.style={
        cio,
        draw,
        inner xsep=.5cm,
        rectangle,
        fill=gray!12!blue!5,
        minimum height=\regheight,
    },
    % Paths
    path/.style={
        thick,
        rounded corners=3pt,
    },
    button/.style={
        path,
        decoration = {markings,
            mark = at position #1 with {
                \fill[white] circle (.7071\push);
                \draw[-] ++(-.5\push, -.5\push) -- ++(\push, \push);
                \draw[-] ++(-.5\push, .5\push) -- ++(\push, -\push);
                \clip[draw] circle (.7071\push); }
        },
        postaction = decorate,
        ->,
        font=\ttfamily,
    },
    arg/.style={
        path,
        ->,
    },
    flow/.style={
        arg,
    },
}

\title{Solutions to Structure and Interpretation of Computer Programs}
\author{Fabienne \textsc{Ducroquet}}

\begin{document}

\maketitle

\tableofcontents

\chapter*{Introduction}

Solutions to most of the exercises of \emph{Structure and Interpretation of 
Computer Programs}, second edition, by Harold \textsc{Abelson} and Gerald Jay 
\textsc{Sussman} with Julie \textsc{Sussman}.

The answers to these exercises have been tested with the Scheme interpreter from 
\href{http://gambitscheme.org}{Gambit Scheme}, at the exception of the exercises 
from section \ref{2.2.4} about the picture language, which have been written 
with \href{http://racket-lang.org}{Racket}, using the \vscm{graphics.ss} 
library.

\chapter{Building Abstractions with Procedures}

\section{The Elements of Programming}

\subsection{Expressions}

This subsection contains no exercises.

\subsection{Naming and the Environment}

This subsection contains no exercises.

\subsection{Evaluationg Combinations}

This subsection contains no exercises.

\subsection{Compound Procedures}

This subsection contains no exercises.

\subsection{The Substitution Model for Procedure Application}

This subsection contains no exercises.

\subsection{Conditional Expressions and Predicates}

\begin{exe}[1.1]
    Check with a scheme interpreter.
\end{exe}

\begin{exe}[1.2]
    A minimally indented version:
    \scm{ch1/1.02a.scm}
    A heavily indented version:
    \scm{ch1/1.02b.scm}
\end{exe}

\begin{exe}[1.3]
    \ \vspace{-20pt}
    \scm{ch1/1.03.scm}
\end{exe}

\begin{exe}[1.4]
    If $b > 0$, \vscm{(a-plus-abs-b a b)} returns $a + b$; otherwise it returns 
    $a - b$. In other words, \vscm{(a-plus-abs-b a b)} returns $a + |b|$.
\end{exe}

\begin{exe}[1.5]
    With applicative-order evaluation, the interpreter tries to evaluate 
    \vscm{(p)}, which results in an infinite loop, so the interpreter never 
    returns (or returns an error).

    With normal-order evaluation, the interpreter doesn't try to evaluate 
    \vscm{(p)} until it's really needed, but that never happens since
    \vscm{(= x 0)} returns true, so the call returns 0.
\end{exe}

\subsection{Example: Square Roots by Newton’s Method}
\label{1.1.7}

\begin{exe}[1.6]
    When Alyssa attempts to use this to compute square roots, the program never 
    returns.

    Explication: \vscm{new-if} is an ordinary procedure, so each time it is 
    called, the evaluator tries to evaluate all of its arguments. In particular, 
    each call to \vscm{sqrt-iter} will cause one more call to \vscm{sqrt-iter}, 
    whether \vscm{(good-enough? guess x)} returns true or not, so the evaluator 
    ends up in an infinite loop.
\end{exe}

\begin{exe}[1.7]
    \label{1.7}
    A possible solution:
    \scm{ch1/1.07.scm}

    Let's call $x$ the number whose root we want to compute.

    With the initial \vscm{good-enough?} test:
    \begin{itemize}
        \item If $x$ is very small, the difference between the guess and $x$ 
        becomes smaller than $0.001$ (or any number we would replace $0.001$ 
        with, for small enough numbers) while the guess is still several times 
        larger than $\sqrt{x}$, or even orders of magnitude away from it.
        \begin{example}
            \vscm{(sqrt 0.0001)} returns $0.03230844833048122$ instead of $0.01$ 
            because\linebreak
            \vscm{(abs (- (square 0.03230844833048122) 0.0001))} 
            returns\linebreak
            $9.438358335233747e-4$.
        \end{example}
    \item If $x$ is very large, the difference between the guess and $x$ will 
        always be found to be larger than $0.001$ (or any number we would 
        replace $0.001$ with, for large enough numbers) because
        ($x - \text{any number}$) can not be expressed to the precision required 
        to compare it to $0.001$, so the call never returns.
        \begin{example}
            \vscm{(sqrt 1e+129)} does not return, while \vscm{(sqrt 1e+128)} 
            returns a correct answer almost instantly\footnote{These values are 
            implementation-dependent.}.
        \end{example}
    \end{itemize}

    With the modified versions of \vscm{good-enough?} and \vscm{sqrt-iter}, the 
    above examples work.
\end{exe}

\begin{exe}[1.8]
    Here is a solution based on the solution of exercise \ref{1.7}.
    \scm{ch1/1.08.scm}
\end{exe}

\subsection{Procedures as Black-Box Abstractions}

This subsection contains no exercises.

\section{Procedures and the Processes They Generate}

\subsection{Linear Recursion and Iteration}

\begin{exe}[1.9]
    With the first procedure:
    \scm{ch1/1.09a.scm}

    With the second procedure:
    \scm{ch1/1.09b.scm}

    The first process is recursive, the second is iterative.
\end{exe}

\begin{exe}[1.10]
    Using the interpreter, we obtain $1024 = 2^{10}$ for \vscm{(A 1 10)}, and 
    $65536 = 2^{16}$ for \vscm{(A 2 4)} and \vscm{(A 3 3)}.

    By definition of the Ackermann function, \vscm{(A 0 n)}, i.e.
    \vscm{(f n)} computes $2n$.

    \medskip

    If $n > 0$, \vscm{(g n)} computes $2^n$.

    \begin{proof}
        By definition, \vscm{(g 1)} equals $2$, and for $n > 1$, \vscm{(A 1 n)} 
        equals \vscm{(A 0 (A 1 (- n 1)))}. Since \vscm{(A 0 n)} computes $2n$,
        the result follows by mathematical induction.
    \end{proof}

    \medskip

    If $n > 0$, \vscm{(h n)} computes $2 \uparrow \uparrow n$, that is, 
    $2^{2^{2^{…}}}$ with $n$ copies of $2$.

    \begin{proof}
        This is true for $n = 1$ by definition. For $n > 1$, \vscm{(A 2 n)} is 
        equal to\linebreak
        \vscm{(A 1 (A 2 (- n 1)))}. The result follows by mathematical induction 
        using the previous result.
    \end{proof}

    \medskip

    \begin{remark}
        \vscm{(A 3 3)} returns $2^{16} = 2^{2^{2^2}}$ as well. The recursion 
        beginning with \vscm{(A 3 n)} with\linebreak
        $n > 1$ gives \vscm{(A 2 (A 3 (- n 1)))}, so according to the previous 
        result, \vscm{(A 3 n)} is obtained from \vscm{(A 3 (- n 1))} by 
        computing $2^{2^{2^{…}}}$ with a tower of
        \vscm{(A 3 (- n 1))} 2s, so since \vscm{(A 3 1)} is $2$, \vscm{(A 3 2)} 
        is $2^2 = 4$, and \vscm{(A 3 3)} is $2^{2^{2^2}}$. The general value of 
        \vscm{(A 3 n)} can be noted $2 \uparrow \uparrow \uparrow n$, or $2 
        \uparrow^3 n$.

        This notation can be extended for all $m$s, so \vscm{(A m n)} computes 
        $2 \uparrow^m n$.
    \end{remark}
\end{exe}

\subsection{Tree Recursion}

\subsubsection{Example: Counting change}

\begin{exe}[1.11]
    Procedure computing $f$ by means of a recursive process :
    \scm{ch1/1.11a.scm}
    Procedure computing $f$ by means of a iterative process :
    \scm{ch1/1.11b.scm}
\end{exe}

\begin{exe}[1.12]
    Here is an example of a solution :
    \scm{ch1/1.12.scm}
    The argument $n$ is the line number from the top starting from $0$, and 
    $k$ is the column number from the left starting from $0$.
\end{exe}

\begin{exe}[1.13]
    \label{1.13}
    Let’s prove that for any $n \geq 0$, $\Fib(n) = \sfrac{\left( \phi^n 
    - \psi^n \right)}{\sqrt{5}}$, where $\phi = \sfrac{\left( 1 + \sqrt{5} 
    \right)}{2}$ and $\psi = \sfrac{\left( 1 - \sqrt{5} \right)}{2}$.

    It’s true for $n = 0$ and $n = 1$.

    Let’s assume that it’s true for any $k < n$. We have :

    \begin{align*}
        \Fib(n) &= \Fib(n - 1) + \Fib(n - 2) \\
        &= \frac{\phi^{n - 1} - \psi^{n - 1}}{\sqrt{5}} + \frac{\phi^{n - 2} 
        - \psi^{n - 2}}{\sqrt{5}} \\
        &= \frac{1}{\sqrt{5}}\left(\phi^{n - 2}\left(\phi + 1\right) - \psi^{n 
        - 2}\left(\psi + 1\right) \right) \\
    \end{align*}

    But $\phi$ and $\psi$ are the roots of the equation $x^2 - x - 1 = 0$, in 
    other words, $\phi^2 = \phi + 1$ and $\psi^2 = \psi + 1$, hence $\Fib(n) 
    = \sfrac{\left(\phi^n - \psi^n \right)}{\sqrt{5}}$.

    Furthermore, $ \lvert 1 - \sqrt{5} \rvert < 2 $, so for any $n \geq 0$, 
    $ \lvert 1 - \sqrt{5} \rvert^n < 2^n $, so dividing by $2^n$, we get 
    $ \lvert \psi^n \rvert < 1 $, and by dividing by $\sqrt{5}$, $ \lvert 
    \sfrac{\psi^n}{\sqrt{5}} \rvert < \sfrac{1}{\sqrt{5}} $. Since 
    $ \sfrac{1}{\sqrt{5}} < \sfrac{1}{2} $, we have $ \lvert 
    \sfrac{\psi^n}{\sqrt{5}} \rvert < \sfrac{1}{2} $ for any $n \geq 0$, which 
    means that $\Fib(n)$ is the closest integer to $\sfrac{\phi^n}{5}$.

\end{exe}

\subsection{Orders of Growth}

\begin{exe}[1.14]
    The space required is proportional to the maximum depth of the tree, so it 
    grows as $\Theta(n)$.

    For the time complexity, let’s use the mathematical notation $\text{cc}(n, 
    k)$ rather than \vscm{(cc n k)}.

    The time complexity for $\text{cc}(n, 1)$ grows as $\Theta(n)$.

    If we note $v$ the denomination of the $k$-th coin, we have:
    \begin{align*}
        \text{cc}(n, k) &= \text{cc}(n - v, k) + \text{cc}(n, k - 1) \\
                        &= \text{cc}(n - 2v, k) + 2\, \text{cc}(n, k - 1) \\
                        &= … \\
                        &= \text{cc}(n - \left\lceil \frac{n}{v} \right\rceil v, 
                        k) + \left\lceil \frac{n}{v} \right\rceil \text{cc}(n, 
                        k - 1)
    \end{align*}

    Since $n - \left\lceil \frac{n}{v} \right\rceil v \leq 0$, the time 
    complexity of $\text{cc}(n, k)$ is proportional to $n$ times the time 
    complexity of $\text{cc}(n, k - 1)$. As a consequence, the time complexity 
    for $5$ kinds of coins grows as $\Theta(n^5)$.
\end{exe}

\begin{exe}[1.15]
    \ \vspace{-20pt}
    \begin{enumerate}
        \item If the argument is greater than $0.1$, \vscm{p} is called once, 
            and the argument is divided by three. So the number of steps 
            required is the smallest integer $n$ such that $\sfrac{12.15}{3^n} 
            < 0.1$, or equivalently $121.5 < 3^n$. The smallest such $n$ is 5, 
            so \vscm{p} is called 5 times when \vscm{(sine 12.15)} is evaluated.
        \item By the same calculation as above, if $a > 0.1$, the number of 
            steps is the smallest $n$ such that $10 \, a < 3^n$. By taking the 
            logarithm, we get $\log(10) + \log(a) < n \log(3)$, so $n 
            = \left\lceil \sfrac{(\log(10) + \log(a))}{\log(3)} \right\rceil$.

            Therefore, the number of steps has order of growth 
            $\Theta(\log(n))$. The space required is proportional to the number 
            of steps, so its order of growth is the same.
    \end{enumerate}
\end{exe}

\subsection{Exponentiation}

\begin{exe}[1.16]
    A possible solution to compute exponentials in a logarithmic number of steps 
    iteratively:
    \scm{ch1/1.16.scm}
\end{exe}

\begin{exe}[1.17]
    A recursive process that multiplies two non-negative integers using 
    a logarithmic number of steps.
    \scm{ch1/1.17.scm}
\end{exe}

\begin{exe}[1.18]
    An iterative process that multiplies two non-negative integers using 
    a logarithmic number of steps.

    We keep a state variable $c$ such that $ab + c$ is constant at each call 
    of the inner function.
    \scm{ch1/1.18.scm}
\end{exe}

\begin{exe}[1.19]
    By calculation, we get $p' = p^2 + q^2$ and $q' = q^2 + 2pq$, so the 
    procedure becomes:
    \scm{ch1/1.19.scm}
\end{exe}

\subsection{Greatest Common Divisors}

\begin{exe}[1.20]
    With normal-order evaluation, \vscm{(gcd 206 40)} expands to
    \vscm{(gcd 40 (remainder 206 40))}, a \vscm{remainder} operation is 
    performed to test whether the remainder is null, then the expression expands 
    to \vscm{(gcd (remainder 206 40) (remainder 40 (remainder 206 40)))}. Two 
    \vscm{remainder} operations are performed to test whether the second 
    argument is null, and the expression expands to
    \begin{cscm}
        (gcd (remainder 40 (remainder 206 40))
             (remainder (remainder 206 40) (remainder 40 (remainder 206 40))))
    \end{cscm}
    Four new executions of \vscm{remainder} are necessary to determine that the 
    second argument is not null, and the expression becomes:
    \begin{cscm}
        (gcd (remainder (remainder 206 40) (remainder 40 (remainder 206 40)))
             (remainder (remainder 40 (remainder 206 40))
                        (remainder (remainder 206 40)
                                   (remainder 40 (remainder 206 40)))))
    \end{cscm}
    Seven executions of \vscm{remainder} are necessary to determine that the 
    second argument is null, and the GCD is computed with four executions of 
    \vscm{remainder}. In total, 18 \vscm{remainder} operations are performed in 
    the normal-order evaluation.

    \bigskip

    With applicative-order evaluation, \vscm{(gcd 206 40)} expands to
    \vscm{(gcd 40 6)}, then to \vscm{(gcd 6 4)}, \vscm{(gcd 4 2)},
    \vscm{(gcd 2 0)} and to 2. One \vscm{remainder} operation is performed each 
    time \vscm{b} is not null, so four such operations are performed.
\end{exe}

\subsection{Example: Testing for Primality}

\begin{exe}[1.21]
    The smallest divisors of 199, 1999 and 19999 are 199, 1999 and 
    7 respectively.
\end{exe}

\begin{exe}[1.22]
    Example solution:
    \scm{ch1/1.22.scm}

    Nowadays, it’s necessary to use numbers much larger than those suggested in 
    the book to test the prediction about the timing, but the data support the 
    $\sqrt{n}$ prediction.

    The result is compatible with the notion that programs run in time 
    proportional to the number of steps required for the computation.
\end{exe}

\begin{exe}[1.23]
    The next procedure is:
    \scm{ch1/1.23a.scm}
    and \vscm{smallest-divisor} becomes:
    \scm{ch1/1.23b.scm}

    The modified version does not run twice as fast, but only about 1.7 times 
    as fast as the original version. This is because time is necessary to 
    apply the \vscm{next} procedure at each step.
\end{exe}

\begin{exe}[1.24]
    The only change needed is to replace \vscm{prime?} with \vscm{fast-prime?} 
    using an arbitrary number of tests (100 here) in \vscm{start-prime-test}.
    \scm{ch1/1.24.scm}

    When the number of digits is doubled, the time needed should be doubled as 
    well since the Fermat test has logarithmic growth. This is what is found 
    experimentally, though again, it’s necessary to use numbers much larger 
    than 1000 and 1,000,000 to get significant results.
\end{exe}

\begin{exe}[1.25]
    Alyssa’s procedure computes the correct result but it is much slower 
    because it deals with huge numbers, whereas by taking the remainder at 
    each recursion step, the numbers remain smaller than the tested number.
\end{exe}

\begin{exe}[1.26]
    If we use an explicit multiplication rather than calling square, 
    \vscm{(expmod base (/ exp 2) m)} is computed twice rather than once at 
    each recursive call with \vscm{exp} even, so that the process becomes 
    linear again.
\end{exe}

\begin{exe}[1.27]
    Here is an example of a procedure that tells whether $a^n$ is congruent to 
    $a$ modulo $n$ for every $a < n$:
    \scm{ch1/1.27.scm}

    It returns true for the given Carmichael numbers.
\end{exe}

\begin{exe}[1.28]
    A possible way to implement the Miller-Rabin test is:
    \scm{ch1/1.28.scm}
    It returns false on the Carmichael numbers listed in footnote 47.
\end{exe}

\section{Formulating Abstractions with Higher-Order Procedures}

\subsection{Procedures as Arguments}

\begin{exe}[1.29]
    Here is a solution:
    \scm{ch1/1.29.scm}
\end{exe}

\begin{exe}[1.30]
    A sum procedure generating an iterative process:
    \scm{ch1/1.30.scm}
\end{exe}

\begin{exe}[1.31]
    \ \vspace{-20pt}
    \begin{enumerate}
        \item Here is a procedure analogous to \vscm{sum} that computes 
            a product, generating a recursive process, and examples of its use 
            to define \vscm{factorial} and to compute approximations of $\pi$. 
            In the latter case, we use $\sfrac{(i - 1)(i + 1)}{i^2} = \sfrac{i^2 
            - 1}{i^2}$ as the general term.
            \scm{ch1/1.31a.scm}
        \item A \vscm{product} procedure generating an iterative process.
            \scm{ch1/1.31b.scm}
    \end{enumerate}
\end{exe}

\begin{exe}[1.32]
    \ \vspace{-20pt}
    \begin{enumerate}
        \item A recursive \vscm{accumulate} procedure and the definition of 
            \vscm{sum} and \vscm{product} using that procedure:
            \scm{ch1/1.32a.scm}
        \item An iterative version of \vscm{accumulate}:
            \scm{ch1/1.32b.scm}
    \end{enumerate}
\end{exe}

\begin{exe}[1.33]
    A \vscm{filtered-accumulate} procedure generating a recursive process:
    \scm{ch1/1.33.1.scm}
    A \vscm{filtered-accumulate} procedure generating an iterative process:
    \scm{ch1/1.33.2.scm}
    \begin{enumerate}
        \item Assuming \vscm{prime?} is already written, the sum of the squares 
            of the prime numbers in the interval $a$ to $b$ can be computed 
            with:
            \scm{ch1/1.33a.scm}
        \item The product of all positive integers less than $n$ that are 
            relatively prime to $n$ can be computed with:
            \scm{ch1/1.33b.scm}
    \end{enumerate}
\end{exe}

\subsection{Construction Procedures Using \vscm{Lambda}}

\begin{exe}[1.34]
    If we try to evaluate \vscm{(f f)}, we get an error saying that the operator 
    is not a procedure. The reason is that \vscm{(f f)} evaluates to
    \vscm{(f 2)}, which itself evaluates to \vscm{(2 2)}, and this operation is 
    impossible since 2 is not a procedure.
\end{exe}

\subsection{Procedures as General Methods}
\label{1.3.3}

\begin{exe}[1.35]
    We already noticed in exercise \ref{1.13} that $\phi^2 = \phi + 1$. By 
    dividing this equation by $\phi$, we get $\phi = 1 + \sfrac{1}{\phi}$.

    We can then compute $\phi$ with the command:
    \scm{ch1/1.35.scm}
\end{exe}

\begin{exe}[1.36]
    Modified version of \vscm{fixed-points}:
    \scm{ch1/1.36a.scm}
    We can find a solution to $x^x = 1000$ with, for instance, without average 
    damping:
    \scm{ch1/1.36b.scm}
    And with average damping:
    \scm{ch1/1.36c.scm}

    The former takes 35 steps while the latter takes 9 steps, so average damping 
    makes the search much faster here.
\end{exe}

\begin{exe}[1.37]
    \ \vspace{-20pt}
    \begin{enumerate}
        \item A procedure \vscm{cont-frac} generating an iterative process, 
            doing the computation starting from \vscm{k}:
            \scm{ch1/1.37a.scm}

            11 steps are necessary to get an approximation that is accurate to 
            4 decimal places.

        \item A procedure \vscm{cont-frac} generating a recursive process, doing 
            the computation starting from 1:
            \scm{ch1/1.37b.scm}
    \end{enumerate}
\end{exe}

\begin{exe}[1.38]
    The following procedure computes an approximation of $e$ using a $k$-term 
    finite continued fraction.
    \scm{ch1/1.38.scm}
\end{exe}

\begin{exe}[1.39]
    A possible solution for \vscm{(tan-cf x k)}:
    \scm{ch1/1.39.scm}
\end{exe}

\subsection{Procedures as Returned Values}

\begin{exe}[1.40]
    The procedure \vscm{cubic} is:
    \scm{ch1/1.40.scm}
\end{exe}

\begin{exe}[1.41]
    The procedure \vscm{double}:
    \scm{ch1/1.41.scm}

    \vscm{(double double)} is a procedure that takes a procedure of one argument 
    as argument and returns a procedure that applies the original procedure four 
    times.

    \vscm{((double (double double)) f)} evaluates to
    \vscm{((double double) ((double double) f))}, so it returns a procedures 
    that applies \vscm{f} $4 \times 4 = 16$ times.

    So the value returned by \vscm{(((double (double double)) inc) 5)} is 21.
\end{exe}

\begin{exe}[1.42]
    Here is a procedure \vscm{compose}:
    \scm{ch1/1.42.scm}
\end{exe}

\begin{exe}[1.43]
    \label{1.43}
    A solution generationg a recursive process:
    \scm{ch1/1.43a.scm}

    A solution generationg an iterative process:
    \scm{ch1/1.43b.scm}
\end{exe}

\begin{exe}[1.44]
    A possible solution is:
    \scm{ch1/1.44.scm}

    The $n$-fold smoothed function of a function \vscm{f} can be obtained with
    \vscm{((repeated smooth n) f)}.
\end{exe}

\begin{exe}[1.45]
    Experimentally, we find that the number of average dampings necessary to 
    compute $n$th roots in this way is $\lfloor \log n \rfloor$, so the 
    procedure to compute $n$th roots is:
    \scm{ch1/1.45.scm}
\end{exe}

\begin{exe}[1.46]
    A solution for \vscm{iterative-improve}:
    \scm{ch1/1.46a.scm}

    The \vscm{sqrt} procedure of section \ref{1.1.7} becomes:
    \scm{ch1/1.46b.scm}

    The \vscm{fixed-point} procedure of section \ref{1.3.3} becomes:
    \scm{ch1/1.46c.scm}
\end{exe}

% vim:filetype=tex:set expandtab

\chapter{Building Abstractions with Data}

\section{Introduction to Data Abstraction}

\subsection{Example: Arithmetic Operations for Rational Numbers}

\begin{exe}[2.1]
    A possibility for a \vscm{make-rat} handling both positive and negative 
    arguments:
    \scm{ch2/2.01.scm}
\end{exe}

\begin{exe}[2.2]
    Exemple implementation for the representation of segments in a plane:
    \scm{ch2/2.02.scm}
\end{exe}

\begin{exe}[2.3]
    In the following implementation, a rectangle is represented by its two 
    opposite sides, which must have the same orientation. The code makes use of 
    auxiliary procedures defined below.

    I added selectors to access each of the vertices of the rectangle to be 
    able to print rectangles in a uniform format.
    \scm{ch2/2.03.1.scm}

    The procedures that compute the perimeter and area of a rectangle, and the 
    procedure that prints a rectangle, are defined thus:
    \scm{ch2/2.03.perim-area.scm}

    Another possibility is to represent a rectangle by its four vertices:
    \scm{ch2/2.03.2.scm}

    Yet another possibility is to represent a rectangle by two perpendicular 
    segments with the same origin:
    \scm{ch2/2.03.3.scm}

    The procedures \vscm{perim-rect}, \vscm{area-rect} and \vscm{print-rect} 
    work in all three cases.

    \begin{comp}
        The code above makes use of the following auxiliary procedures to check 
        that the input is correct, compute the length of a segment, and print 
        points inline for use in \vscm{print-rect}:
        \scm{ch2/2.03.aux.scm}
    \end{comp}
\end{exe}

\chapter{Modularity, Objects, and State}

\section{Assignment and Local State}

\subsection{Local State Variables}

\begin{exe}[3.1]
    The procedure \vscm{make-accumulator} can be written:
    \scm{ch3/3.01.scm}
\end{exe}

\begin{exe}[3.2]
    The \vscm{make-monitored} procedure can be written:
    \scm{ch3/3.02.scm}
\end{exe}

\begin{exe}[3.3]
    The \vscm{make-account} procedure can be modified in the following way:
    \scm{ch3/3.03.scm}
\end{exe}

\begin{exe}[3.4]
    The procedure can be rewritten as:
    \scm{ch3/3.04.scm}
\end{exe}

\subsection{The Benefits of Introducing Assignment}

\begin{exe}[3.5]
    Using Gambit Scheme’s \vscm{random-real} procedure, that generates a random 
    real number between 0 and 1, \vscm{random-in-range} and the other procedures 
    can be written:
    \scm{ch3/3.05.scm}
\end{exe}

\begin{exe}[3.6]
    The \vscm{rand} procedure can be rewritten as:
    \scm{ch3/3.06.scm}
\end{exe}

\subsection{The Costs of Introducing Assignment}

\begin{exe}[3.7]
    I simply added a \vscm{join} action to the account returned by 
    \vscm{make-account} that creates an access with another password. I also 
    make \vscm{incorrect-password} throw an error instead of simply returning 
    a string, otherwise a call such as
    \vscm{(define new-acc (make-join account curr-pass new-pass))} with an 
    incorrect current password will affect a string value to \vscm{new-acc} 
    without reporting an error, and subsequent uses of the account will throw 
    errors because \vscm{"Incorrect password"} is not a procedure.
    \scm{ch3/3.07.scm}
\end{exe}

\begin{exe}[3.8]
    The procedure \vscm{f} returns:
    \begin{itemize}
        \item 0 if it is the first time it is called;
        \item the previous argument it was called with otherwise.
    \end{itemize}
    Thus, if we evaluate \vscm{(f 0)}, then \vscm{(f 1)}, we get 0 both times, 
    but if we evaluate \vscm{(f 1)}, then \vscm{(f 0)}, we get 0 the first 
    time and 1 the second.
    \scm{ch3/3.08.scm}
\end{exe}

\chapter{Metalinguistic Abstraction}

\section{The Metacircular Evaluator}

\subsection{The Core of the Evaluator}

\begin{exe}[4.1]
    We can put the value to evaluate first inside a \vscm{let} expression. Since 
    the \vscm{let} is transformed into a \vscm{lambda} internally, we know that 
    its argument will be evaluated before the body of the \vscm{let}.
    \scm{ch4/4.01.scm}
\end{exe}

\subsection{Representing Expressions}

\begin{exe}[4.2]
    \ \vspace{-20pt}
    \begin{itemize}
	\item[a.] The test to determine whether an expression is a procedure 
	    just checks whether it is a pair, so it will return true for all 
	    list expressions: \vscm{if}, \vscm{cond}, \vscm{begin}, 
	    \vscm{define}, \vscm{set!}, etc., and the evaluator will try to 
	    apply the procedure \vscm{if}, \vscm{cond}, etc. to the rest of the 
	    arguments. It’s not possible to transform the special forms into 
	    procedures because all the arguments of a procedure are evaluated 
	    before evaluation.
	\item[b.] The only required change to \vscm{eval} is to put the 
	    \vscm{application?} test first. Then we only need changing the 
	    definition of the \vscm{application?} predicate and the associated 
	    selectors so they reflect the new syntax.
	    \scm{ch4/4.02.scm}
    \end{itemize}
\end{exe}

\chapter{Computing with Register Machines}

\section{Designing Register Machines}

\begin{exe}[5.1]
    \label{5.1}
    The data-path and the controller diagrams for the iterative factorial 
    machine are shown on figure~\ref{5.1fig}.

    \begin{figure}
        \centering
        \begin{tikzpicture}[>=Stealth]
            % Data-path diagram.
            \matrix[data matrix] (dp) {
                & \node[reg] (n) {n}; &
                \node[test] (>) {>}; \\

                & \node[const] (c1) {1}; & \\

                \node[reg] (p) {product}; &&
                \node[reg] (c) {counter}; \\

                & \node[op] (+) {+}; & \\

                & \node[op] (*) {*}; & \\
            };

            \draw[arg] (n) -- (>);
            \draw[arg] (c) -- (>);
            \draw[button=0.82] (c1) -| node[near start, above] {p<-1} (p);
            \draw[button=0.82] (c1) -| node[near start, above] {c<-1}
                ($ (c.north) + (-1em, 0) $);
            \draw[arg] (c1) -- (+);
            \draw[button=0.8] ($ (+.north) + (1em, 0) $) |-
                node[near end, above] {c++} (c);
            \draw[arg] (c) |- (+);
            \draw[arg] ($ (p.south) + (-1em, 0) $) |-
                ($ (*.south) + (0, -1.5em) $) -- (*);
            \draw[arg] (c) |- (*);
            \draw[button=0.8] (*) -| node[near end, right]{p<-*}
                ($ (p.south) + (1em, 0) $);

            % Controller diagram
            \matrix[controller matrix, right=5em of dp] {
                \node[cio] (s) {start}; \\
                \node[cbutton] (pi) {p<-1}; \\
                \node[cbutton] (ci) {c<-1}; \\
                \node[ctest] (ct) {>}; & \node[cio] (cd) {done}; \\[+1em]
                \node[cbutton] (cp) {p<-*}; \\
                \node[cbutton] (cc) {c++}; \\
            };

            \draw[flow] (s) -- (pi);
            \draw[flow] (pi) -- (ci);
            \draw[flow] (ci) -- (ct);
            \draw[flow] (ct) --node[right] {no} (cp);
            \draw[flow] (ct) --node[above] {yes} (cd);
            \draw[flow] (cp) -- (cc);
            \draw[flow] (cc) -| ($ (ct.west) - (2.5em, 0) $) -- (ct);
        \end{tikzpicture}
        \caption{The data-path and controller diagrams for the iterative 
        factorial machine.}
        \label{5.1fig}
    \end{figure}
\end{exe}

\subsection{A Language for Describing Register Machines}

\begin{exe}[5.2]
    Anticipating on the next section to use the register-machine simulator, we 
    can define the iterative factorial machine of exercise~\ref{5.1} as:
    \scm{ch5/5.02.scm}
\end{exe}

\subsection{Abstraction in Machine Design}

\begin{exe}[5.3]
    Using the simulator again, the first version of the register machines can be 
    defined as:
    \scm{ch5/5.03a.scm}
    and the second version as:
    \scm{ch5/5.03b.scm}
    The data-path diagrams are shown on figures~\ref{5.03afig} 
    and~\ref{5.03bfig} respectively.

    \begin{figure}
        \centering
        \begin{tikzpicture}[>=Stealth]
            \matrix[data matrix] {
                \node[const] (c1) {1};
                &[+1em] \node[reg] (g) {guess};
                & \node[test] (ge) {g-e?};
                & \node[reg] (x) {x}; \\

                && \node[op] (i) {improve}; \\
            };

            \draw[button=.5] (c1) -- (g);
            \draw[arg] (g) -- (ge);
            \draw[arg] (x) -- (ge);
            \draw[arg] ($ (g.south) + (1em, 0) $) |- (i);
            \draw[button=.7] (i.south) -- ($ (i.south) - (0, 1em) $) -|
                ($ (g.south) - (1em, 0) $);
            \draw[arg] (x) |- (i);
        \end{tikzpicture}
        \caption{The data-path diagram for the square root machine using complex 
        primitive operations.}
        \label{5.03afig}
    \end{figure}

    \begin{figure}
        \centering
        \begin{tikzpicture}[>=Stealth]
            \matrix[data matrix, matrix of nodes, nodes in empty cells]
            (table) {%
                && \node[const] (c1) {1};
                & \node[op] (div2) {/};
                & \node[const] (c2) {2}; \\

                &&& \node[minimum height=1.5\regheight] {guess}; &&& \\

                && \node[op] (divx) {/};
                & \node[op] (*) {*};
                & \node[op] (+) {+}; \\

                & \node[reg] (x) {x};
                && \node {tmp};
                && \node[test] (<eps) {<};
                & \node[const, inner xsep = -.2em] (eps) {0.001}; \\

                && \node[op] (mx) {-};
                & \node[op] (mtmp) {-};
                & \node[test] (>0) {<};
                & \node[const] (c0) {0}; \\
            };
            \begin{scope}[on background layer]
                \node[reg, fit=(table-2-3)(table-2-5)] (guess) {};
                \node[reg, fit=(table-4-3)(table-4-5)] (tmp) {};
            \end{scope}

            \draw[button=.6] (c1) -- (guess.north -| c1);
            \draw[arg] ($ (guess.north) - (3em, 0) $) |- (div2);
            \draw[arg] (c2) -- (div2);
            \draw[button=.6] (div2) -- (guess);
            \draw[arg] (x) |- (divx);
            \draw[arg] (guess.south -| divx) -- (divx);
            \draw[button=.6] (divx) -- (tmp.north -| divx.south);
            \draw[arg] ($ (guess.south) - (1em, 0) $) --
                ($ (*.north) - (1em, 0) $);
            \draw[arg] ($ (guess.south) + (1em, 0) $) --
                ($ (*.north) + (1em, 0) $);
            \draw[button=.6] (*) -- (tmp);
            \coordinate (pc) at (guess.south -| +.north);
            \draw[arg] ($ (pc) - (1em, 0) $) -- ($ (+.north) - (1em, 0) $);
            \draw[button=.6] ($ (+.north) + (1em, 0) $) -- ($ (pc) + (1em, 0) $);
            \draw[arg] (tmp.north -| +) -- (+);
            \draw[arg] (tmp) -- (<eps);
            \draw[arg] (eps) -- (<eps);
            \draw[arg] (x) |- (mx);
            \coordinate (mc) at (tmp.south -| mx.north);
            \draw[arg] ($ (mc) - (1em, 0) $) -- ($ (mx.north) - (1em, 0) $);
            \draw[button=.6] ($ (mx.north) + (1em, 0) $) --
                ($ (mc) + (1em, 0) $);
            \draw[arg] ($ (tmp.south) - (1em, 0) $) --
                ($ (mtmp.north) - (1em, 0) $);
            \draw[button=.6] ($ (mtmp.north) + (1em, 0) $) --
                ($ (tmp.south) + (1em, 0) $);
            \draw[arg] (tmp.south -| >0) -- (>0);
            \draw[arg] (c0) -- (>0);
        \end{tikzpicture}
        \caption{The data-path diagram for the square root machine using only 
        basic primitive operations.}
        \label{5.03bfig}
    \end{figure}
\end{exe}

\subsection{Subroutines}

This subsection contains no exercises.

\subsection{Using a Stack to Implement Recursion}

\begin{exe}[5.4]
    \label{5.04}
    \ \vspace{-20pt}
    \begin{enumerate}
	\item The recursive exponentiation machine can be defined as follows. 
	    The corresponding data-path diagram is shown on figure~\ref{5.04afig}.
            \scm{ch5/5.04a.scm}
	\item The iterative exponentiation machine can be defined as follows. 
	    The corresponding data-path diagram is shown on 
	    figure~\ref{5.04bfig}.
            \scm{ch5/5.04b.scm}
    \end{enumerate}

    \begin{figure}
        \centering
        \begin{tikzpicture}[>=Stealth]
            \matrix[data matrix] {
                \node[const] (c0) {0};
                & \node[test] (=) {=};
                & \node[reg] (n) {n};
                &[+1em] &[-4.5em] \node[reg] (stack) {stack}; \\

                \node[const] (c1) {1};
                & \node[op] (-) {-};
                &&& \node[reg] (continue) {continue};
                &[-4.5em]& \node (controller) {controller}; \\

                \node[reg] (val) {val};
                & \node[op] (*) {*};
                & \node[reg] (b) {b};
                & \node[const, inner xsep=-.2em] (ed) {e-d};
                && \node[const, inner xsep=-.2em] (ae) {a-e}; \\
            };

            \draw[arg] (c0) -- (=);
            \draw[arg] (n) -- (=);
            \draw[button=.5] ($ (n.east) + (0, .7em) $) --
                ($ (stack.west) + (0, .7em) $);
            \draw[button=.5] ($ (stack.west) - (0, .7em) $) --
                ($ (n.east) - (0, .7em) $);
            \draw[arg] (c1) -- (-);
            \draw[arg] ($ (n.south) - (.7em, 0) $) |- (-);
            \draw[button=.8] (-) -- ($ (-.south) - (0, 1em) $)
                -| ($ (n.south) + (.7em, 0) $);
            \draw[button=.5] ($ (stack.south) - (.7em, 0) $) --
                ($ (continue.north) - (.7em, 0) $);
            \draw[button=.5] ($ (continue.north) + (.7em, 0) $) --
                ($ (stack.south) + (.7em, 0) $);
            \draw[arg] (continue) -- (controller);
            \draw[arg] (c1) -- (val);
            \draw[arg] (val) -- (*);
            \draw[button=.7] (*) -- ($ (*) - (0, 3em) $) -| (val);
            \draw[arg] (b) -- (*);
            \draw[button=.5] (ed) -- (ed |- continue.south);
            \draw[button=.5] (ae) -- (ae |- continue.south);
        \end{tikzpicture}
        \caption{The data-path diagram for the recursive exponentiation 
        machine.}
        \label{5.04afig}
    \end{figure}

    \begin{figure}
        \centering
        \begin{tikzpicture}[>=Stealth]
            \matrix[data matrix] {
                \node[reg] (b) {b};
                & \node[const] (c0) {0};
                &[+1em] \node[op] (=) {=}; \\

                \node[op] (*) {*};
                & \node[reg] (n) {n};
                & \node[reg] (c) {counter}; \\

                \node[reg] (p) {product};
                & \node[const] (c1) {1};
                & \node[op] (-) {-}; \\
            };

            \draw[arg] (c0) -- (=);
            \draw[arg] (c) -- (=);
            \draw[arg] (b) -- (*);
            \draw[arg] ($ (p.north) - (.7em, 0) $) --
                ($ (*.south) - (.7em, 0) $);
            \draw[button=.5] ($ (*.south) + (.7em, 0) $) --
                ($ (p.north) + (.7em, 0) $);
            \draw[button=.5] (n) -- (c);
            \draw[arg] (c1) -- (p);
            \draw[arg] (c1) -- (-);
            \draw[arg] ($ (c.south) - (1em, 0) $) --
                ($ (-.north) - (1em, 0) $);
            \draw[button=.5] ($ (-.north) + (1em, 0) $) --
                ($ (c.south) + (1em, 0) $);
        \end{tikzpicture}
        \caption{The data-path diagram for the iterative exponentiation 
        machine.}
        \label{5.04bfig}
    \end{figure}
\end{exe}

\begin{exe}[5.5]
    The following table lists the instructions evaluated during the simulation 
    of factorial 3, with their effect on the values of the registers \vscm{n}, 
    \vscm{val}, and \vscm{continue} and on the stack.
    \begin{longtable}{|l|c|c|c|c|}
        \hline
        \bfseries Instruction & \bfseries n & \bfseries val & \bfseries continue 
        & \bfseries stack \\\hline
        \endhead
        (assign cont (label fact-done)) & 3 & *unassigned* & fact-done & () 
        \\\hline
        (test (op =) ...) -> false &&&& \\\hline
        (branch (label base-case)) &&&& \\\hline
        (save continue) &&&& (fact-done) \\\hline
        (save n) &&&& (3 fact-done) \\\hline
        (assign n (op -) ...) & 2 &&& \\\hline
        (assign cont (label after-fact)) &&& after-fact & \\\hline
        (goto (label fact-loop)) &&&& \\\hline
        (test (op =) ...) -> false &&&& \\\hline
        (branch (label base-case)) &&&& \\\hline
        (save cont) &&&& (after-fact 3 fact-done) \\\hline
        (save n) &&&& (2 after-fact 3 fact-done) \\\hline
        (assign n (op -) ...) & 1 &&& \\\hline
        (assign cont (label after-fact)) &&& after-fact & \\\hline
        (goto (label fact-loop)) &&&& \\\hline
        (test (op =) ...) -> true &&&& \\\hline
        (branch (label base-case)) &&&& \\\hline
        (assign val (const 1)) && 1 && \\\hline
        (goto (reg cont)) -> after-fact &&&& \\\hline
        (restore n) & 2 &&& (after-fact 3 fact-done) \\\hline
        (restore cont) &&& after-fact & (3 fact-done) \\\hline
        (assign val (op *) ...) && 2 && \\\hline
        (goto (reg cont)) -> after-fact &&&& \\\hline
        (restore n) & 3 &&& (fact-done) \\\hline
        (restore cont) &&& fact-done & () \\\hline
        (assign val (op *) ...) && 6 && \\\hline
        (goto (reg cont)) -> fact-done &&&& \\\hline
    \end{longtable}

    The following table lists the instructions evaluated during the simulation 
    of fibonacci 3, with their effect on the values of the registers \vscm{n}, 
    \vscm{val}, and \vscm{continue} and on the stack.
    \begin{longtable}{|l|c|c|c|c|}
        \hline
        \bfseries Instruction & \bfseries n & \bfseries val & \bfseries continue 
        & \bfseries stack \\\hline
        \endhead
        (assign cont (label fib-done) & 3 & *unassigned* & fib-done & () 
        \\\hline
        (test (op <) ...) -> false &&&& \\\hline
        (branch (label imm-answer)) &&&& \\\hline
        (save cont) &&&& (fib-done) \\\hline
        (assign cont (label after-fib-n-1)) &&& after-fib-n-1 & \\\hline
        (save n) &&&& (3 fib-done) \\\hline
        (assign n (op -) ...1) & 2 &&& \\\hline
        (goto (label fib-loop)) &&&& \\\hline
        (test (op <) ...) -> false &&&& \\\hline
        (branch (label imm-answer)) &&&& \\\hline
        (save cont) &&&& (after-fib-n-1 3 fib-done) \\\hline
        (assign cont (label after-fib-n-1)) &&& after-fib-n-1 & \\\hline
        (save n) &&&& (2 after-fib-n-1 3 fib-done) \\\hline
        (assign n (op -) ...1) & 1 &&& \\\hline
        (goto (label fib-loop)) &&&& \\\hline
        (test (op <) ...) -> true &&&& \\\hline
        (branch (label imm-answer)) &&&& \\\hline
        (assign val (reg n)) && 1 && \\\hline
        (goto (reg cont)) -> after-fib-n-1 &&&& \\\hline
        (restore n) & 2 &&& (after-fib-n-1 3 fib-done) \\\hline
        (restore cont) &&& after-fib-n-1 & (3 fib-done) \\\hline
        (assign n (op -) ...2) & 0 &&& \\\hline
        (save cont) &&&& (after-fib-n-1 3 fib-done) \\\hline
        (assign cont (label after-fib-n-2)) &&& after-fib-n-2 & \\\hline
        (save val) &&&& (1 after-fib-n-1 3 fib-done) \\\hline
        (goto (label fib-loop)) &&&& \\\hline
        (test (op <) ...) -> true &&&& \\\hline
        (branch (label imm-answer)) &&&& \\\hline
        (assign val (reg n)) && 0 && \\\hline
        (goto (reg cont)) -> after-fib-n-2 &&&& \\\hline
        (assign n (reg val)) & 0 &&& \\\hline
        (restore val) && 1 && (after-fib-n-1 3 fib-done) \\\hline
        (restore cont) &&& after-fib-n-1 & (3 fib-done) \\\hline
        (assign val (op +) ...) && 1 && \\\hline
        (goto (reg cont)) -> after-fib-n-1 &&&& \\\hline
        (restore n) & 3 &&& (fib-done) \\\hline
        (restore cont) &&& fib-done & () \\\hline
        (assign n (op -) ... 2) & 1 &&& \\\hline
        (save cont) &&&& (fib-done) \\\hline
        (assign cont (label after-fib-n-2)) &&& after-fib-n-2 & \\\hline
        (save val) &&&& (1 fib-done) \\\hline
        (goto (label fib-loop)) &&&& \\\hline
        (test (op <) ...) -> true &&&& \\\hline
        (branch (label imm-answer)) &&&& \\\hline
        (assign val (reg n)) && 1 && \\\hline
        (goto (reg cont)) -> after-fib-n-2 &&&& \\\hline
        (assign n (reg val)) & 1 &&& \\\hline
        (restore val) && 1 && (fib-done) \\\hline
        (restore cont) &&& fib-done & () \\\hline
        (assign val (op +) ...) && 2 && \\\hline
        (goto (reg cont)) -> fib-done &&&& \\\hline
    \end{longtable}
\end{exe}

\begin{exe}[5.6]
    In \vscm{after-fib-n-1}, the instructions \vscm{(restore continue)} and 
    \vscm{(save continue)} can be removed because no change is done to the 
    \vscm{continue} register or to the stack between them.
\end{exe}

\subsection{Instruction Summary}

This subsection contains no exercises.

\section{A Register-Machine Simulator}

\begin{exe}[5.7]
    Already done while doing exercise~\ref{5.04}.
\end{exe}

\subsection{The Machine Model}

This subsection contains no exercises.

\subsection{The Assembler}

\begin{exe}[5.8]
    With the simulator as written, the contents of register \vscm{a} will be 3: 
    two labels with the name \vscm{here} are present in the list of labels, but 
    the one corresponding to the first location comes first and is the one 
    returned by \vscm{lookup-label}.

    We can return an error if the same label name is used to indicate two 
    different locations by modifying \vscm{extract-labels} as follows:
    \scm{ch5/5.08.scm}
\end{exe}

\subsection{Generating Execution Procedures for Instructions}

\begin{exe}[5.9]
    We can forbid labels as arguments to operations by adding a test to the 
    \vscm{make-operation-exp} procedure:
    \scm{ch5/5.09.scm}
\end{exe}

\begin{exe}[5.10]
    I only modified the syntax of \vscm{op}, for instance
    \vscm{(assign <reg-name> (op <op-name>) <args>)} becomes
    \vscm{(assign <reg-name> op <op-name> <args>)}:
    \scm{ch5/5.10.scm}
\end{exe}

\begin{exe}[5.11]
    \ \vspace{-20pt}
    \begin{enumerate}
	\item After the label \vscm{after-fib-n-2}, the last value placed on the 
	    stack is $\Fib(n - 1)$, and the \vscm{val} register contains $\Fib(n 
	    - 2)$. The Fibonacci machine places $\Fib(n - 2)$ in the \vscm{n} 
	    register before restoring the last saved value in the \vscm{val} 
	    register. We can instead restore the last saved value in the 
	    \vscm{n} register and remove an assignment operation. So the lines:
            \begin{cscm}
                (assign n (reg val))
                (restore val)
            \end{cscm}
            become:
            \begin{cscm}
                (restore n)
            \end{cscm}
	\item We need to change \vscm{make-save} to put the register name on the 
	    stack, and \vscm{make-restore} to check that the original register 
	    corresponds to the target register:
	    \scm{ch5/5.11b.scm}
	\item I chose to modify \vscm{make-register} to directly add a stack to 
	    each register:
	    \scm{ch5/5.11c.scm}
	    Some other procedures must be modified as well: no stack is needed 
	    anymore in
	    \linebreak
	    \vscm{make-new-machine}, and the \vscm{stack} parameter can be 
	    removed from
	    \linebreak
	    \vscm{make-execution-procedure}. As indicated in the text, the 
	    \vscm{initialize-stack} operation should initialize all the register 
	    stacks, which can be done by replacing the original definition of 
	    \vscm{the-ops} in \vscm{make-new-machine} with:
	    \scm{ch5/5.11c-init.scm}
    \end{enumerate}
\end{exe}

\begin{exe}[5.12]
    \label{5.12}
    Since we need ordered sets, we first reuse a slightly modified version of 
    the representation of sets as ordered lists defined in section~\ref{2.3.3}:
    \scm{ch5/5.12set.scm}
    To store the list of all instructions, we change \vscm{assemble} and 
    \vscm{extract-label} to build the list gradually as the instructions are 
    processed and then add it to the machine:
    \scm{ch5/5.12a1.scm}
    We also modify \vscm{make-new-machine} to define \vscm{instructions-list} as 
    the empty list at first and add new cases to the \vscm{dispatch} procedure:
    \scm{ch5/5.12a2.scm}
    To store the lists of the registers used to hold entry points and of the 
    registers that are \vscm{saved} or \vscm{restored}, we change 
    \vscm{make-new-machine} to define those lists, as well as procedures to add 
    an element to them and new messages to access them:
    \scm{ch5/5.12b2.scm}
    We then modify \vscm{make-goto}, \vscm{make-save} and \vscm{make-restore} to 
    add elements to the lists:
    \scm{ch5/5.12b1.scm}
    To store the sources from which each register is assigned, we change 
    \vscm{make-register} to store this information in each register, and 
    \vscm{make-assign} to add the sources to the target register. We also define 
    a \vscm{get-sources} procedure to access this information more easily:
    \scm{ch5/5.12c.scm}
\end{exe}

\begin{exe}[5.13]
    We change \vscm{make-machine} so it does not take a list of registers as an 
    argument:
    \scm{ch5/5.13a.scm}
    Then, we change the \vscm{lookup-register} procedure in 
    \vscm{make-new-machine} so that it allocates a new register if no register 
    with the given name exists. I also modified \vscm{allocate-register} to 
    return the newly allocated register to simplify the procedures. The dispatch 
    clause for \vscm{allocate-register} can be removed.
    \scm{ch5/5.13b.scm}
\end{exe}

\subsection{Monitoring Machine Performance}

\begin{exe}[5.14]
    The recursive factorial machine can be modified as shown below to initialize 
    the stack and print the statistics.
    \scm{ch5/5.14.scm}
    The total number of push operations and the maximum stack depth used in 
    computing $n!$ are both equal to $2 (n - 1)$: the \vscm{n} and 
    \vscm{continue} registers are each saved $n - 1$ times, then all the 
    elements of the stack are popped.
\end{exe}

\begin{exe}[5.15]
    We need to change \vscm{make-new-machine} at three places:
    \begin{enumerate}[label=\arabic*.]
	\item Add an \vscm{instruction-count} in the variables defined by the 
	    \vscm{let} at the beginning:
	    \begin{cscm}
		(instruction-count 0)
	    \end{cscm}

	\item Modify \vscm{execute} to increase the count before executing the 
	    procedure:
	    \scm{ch5/5.15a.scm}

	\item Add a message to the \vscm{dispatch} procedure to print and reset 
	    the count:
	    \scm{ch5/5.15b.scm}
    \end{enumerate}
\end{exe}

\begin{exe}[5.16]
    As in the previous exercise, the \vscm{make-new-machine} procedure must be 
    modified in three places:
    \begin{enumerate}[label=\arabic*.]
	\item Add a \vscm{trace} variable (initialized to \vscm{false}).
	\item Modify \vscm{execute} to display the instruction:
	    \scm{ch5/5.16a.scm}
	\item Add the messages \vscm{trace-on} and \vscm{trace-off} to the 
	    \vscm{dispatch} procedure:
	    \scm{ch5/5.16b.scm}
    \end{enumerate}
\end{exe}

\begin{exe}[5.17]
    \label{5.17}
    First, we change the representation of instructions from a pair to a list of 
    three elements: the instruction text, the label immediately preceding the 
    instruction, and the execution procedure.
    \scm{ch5/5.17a.scm}
    Then, we change \vscm{make-label-entry} so that it sets the label name of 
    the instruction following the label, if any:
    \scm{ch5/5.17b.scm}
    Lastly, we define a procedure to print an instruction and call it in 
    \vscm{execute}:
    \scm{ch5/5.17c.scm}
\end{exe}

\begin{exe}[5.18]
    We can modify the \vscm{make-register} procedure and define procedures to 
    turn tracing on and off for a given register as shown below. This extends 
    the definition of \vscm{make-register} from exercise~\ref{5.12}.
    \scm{ch5/5.18.scm}
\end{exe}

\begin{exe}[5.19]
    The procedures described in the text are just syntactic sugar:
    \scm{ch5/5.19a.scm}
    We will need the labels in order to find the instruction where to put the 
    breakpoint, so we add a message to set them to the machine and set them in 
    \vscm{assemble}:
    \scm{ch5/5.19b.scm}
    We add the appropriate messages to the \vscm{dispatch} procedure in 
    \vscm{make-new-machine}:
    \scm{ch5/5.19c.scm}
    A breakpoint is defined as a pair containing the label and offset, and the 
    representation of instructions from exercise~\ref{5.17} is extended to 
    include the breakpoint. We define procedures to check if an instruction has 
    a breakpoint and to remove the breakpoint from an instruction as well.
    \scm{ch5/5.19d.scm}
    The only missing part is the code actually handling the messages:
    \begin{itemize}
	\item To set or cancel a breakpoint, we look up the instruction 
	    corresponding to the given label and offset and add or remove the 
	    breakpoint to the instruction.
	\item To cancel all breakpoints, we remove any breakpoint from each 
	    instruction in the instruction sequence.
	\item The \vscm{execute} procedure is modified to check if the next 
	    instruction has a breakpoint. If so, it prints the information 
	    regarding the breakpoint and stops executing instructions.
	\item To continue execution, we just call \vscm{execute}, but we need to 
	    tell this procedure that it should not check again whether the next 
	    instruction is a breakpoint (we know it is and we already stopped), 
	    which we do with an additional parameter to \vscm{execute}, which is 
	    \vscm{false} only when proceeding from a breakpoint.
    \end{itemize}
    \scm{ch5/5.19e.scm}
\end{exe}

\section{Storage Allocation and Garbage Collection}

\subsection{Memory as Vectors}

\begin{exe}[5.20]
    The box-and-pointer representation and the memory-vector representation of 
    the given list structure are shown in figure~\ref{5.20y}. The final value of 
    \vscm{free} is \vscm{p4}. The values of \vscm{x} and \vscm{y} are 
    represented by the pointers \vscm{p1} and \vscm{p3} respectively.
    \begin{figure}
        \centering
	\begin{tikzpicture}[box and pointer]
	    % Box-and-pointer representation.
	    \matrix[cell matrix] (box-pointer) {
		\node[struct name] (y) {y}; &[+2\boxsize]
		\node[car, label=225:{3}] (c11) {}; & \node[cdr] (c12) {}; 
		&[+\boxsize]
		\node[car, label=225:{2}] (c13) {}; & \node[cdr] (c14) {}; \\

		\node[struct name] (x) {x}; &[+2\boxsize]
		\node[car, label=225:{1}] (c21) {}; & \node[cdr] (c22) {}; &
		\node[box] (n2) {2}; \\

		&
		\node[box] (n1) {1}; \\
	    };

	    \draw[pointer] (y) -- (c11);
	    \draw[pointer] (x) -- (c21);
            \link{c11}{c21}
            \link{c12}{c13}
	    \draw[box pointer] (c13.base) -- ++(0, -1cm) -| (c22);
	    \nil{c14}
	    \link{c21}{n1}
	    \link{c22}{n2}

	    % Memory-vector representation.
	    \node[right=of box-pointer] {
	    \renewcommand\arraystretch{1.5}
	    \newcommand\mc[1]{\multicolumn{1}{|c}{#1}}
	    \newcommand\emc{\mc{\hphantom{n1}}}
	    \setlength{\arrayrulewidth}{1pt}
	    \begin{tabular}{rcccccc}
		Index & 0 & 1 & 2 & 3 & 4 & ... \\\cline{2-7}
		the-cars & \emc & \mc{n1} & \mc{p1} & \mc{p1} & \emc & \mc{...} 
		\\\cline{2-7}
		the-cdrs & \emc & \mc{n2} & \mc{e0} & \mc{p2} & \emc & \mc{...} 
		\\\cline{2-7}
	    \end{tabular}
	    };
	\end{tikzpicture}
	\caption{The list structure produced by \vscm{(define x (cons 1 2))} 
	followed by \vscm{(define y (list x x))}.}
	\label{5.20y}
    \end{figure}
\end{exe}

\begin{exe}[5.21]
    \begin{enumerate}
	\item We can define a register machine to count the leaves of a tree as 
	    follows:
	    \scm{ch5/5.21a.scm}
	\item The register machine with an explicit counter can be defined as 
	    follows. This version is simpler because the second recursion can be 
	    transformed into a loop.
	    \scm{ch5/5.21b.scm}
    \end{enumerate}
\end{exe}

\begin{exe}[5.22]
    We can define the \vscm{append-machine} and \vscm{append!-machine} as 
    follows. The second version is simpler because there is a simple loop while 
    the first version uses recursion.
    \scm{ch5/5.22a.scm}
    \scm{ch5/5.22b.scm}
\end{exe}

\subsection{Maintaining the Illusion of Infinite Memory}

This subsection contains no exercises.

\section{The Explicit-Control Evaluator}

\subsection{The Core of the Explicit-Control Evaluator}

This subsection contains no exercises.

\subsection{Sequence Evaluation and Tail Recursion}

This subsection contains no exercises.

\subsection{Conditionals, Assignments, and Definitions}

\begin{exe}[5.23]
    I added the expressions \vscm{cond} (including the alternative syntax from 
    exercise~\ref{4.5}), \vscm{and} and \vscm{or} from exercise~\ref{4.4}, 
    \vscm{let} (\ref{4.6} and~\ref{4.8}), \vscm{let*} (\ref{4.7}), \vscm{letrec} 
    (\ref{4.20}), and \vscm{while}, \vscm{until} and \vscm{for} from 
    exercise~\ref{4.9}.

    The changes are similar for each expression type, so I am including them 
    here only for \vscm{cond} expressions. Besides loading the necessary syntax 
    predicates, selectors and transformers, we add a new test to the 
    \vscm{eval-dispatch} loop:
    \begin{cscm}
	(test (op cond?) (reg exp))
	(branch (label ev-cond))
    \end{cscm}
    At the given label, we transform the expression to evaluate before going 
    back to \vscm{eval-dispatch}:
    \begin{cscm}
    ev-cond
        (assign exp (op cond->if) (reg exp))
        (goto (label eval-dispatch))
    \end{cscm}
    It is tempting to go to \vscm{ev-if} instead, but if the \vscm{cond} 
    expression has no clauses, \vscm{cond->if} simply returns \vscm{false}, 
    which is not an \vscm{if} expression, so this would lead to an error. 
    Several other transformations (\vscm{and}, \vscm{or}, \vscm{let}...) also do 
    not always return an expression of the same type, so I chose to always 
    return to the \vscm{eval-dispatch} label after the transformation, even if 
    in some cases we could directly jump to the right expression type.
\end{exe}

\begin{exe}[5.24]
    We can implement \vscm{cond} as a new special form by replacing the 
    instructions handling such expressions with the instructions included below.

    The instructions are a translation of the \vscm{cond->if} procedure to 
    register-machine code.

    The loop starts at the label \vscm{ev-cond-clauses}, at which point the 
    \vscm{unev} register holds the remaining clauses. We first check if there 
    are remaining clauses. If no clauses remain, either because the \vscm{cond} 
    was empty or because no predicate was true, we return \vscm{false}. Then we 
    test whether the next clause is an \vscm{else} clause:
    \begin{itemize}
	\item If not, we evaluate the next clause’s predicate, after saving all 
	    the registers that could be needed later.
	    \begin{itemize}
		\item If the predicate is true, we evaluate the clause’s actions 
		    as a sequence. We save \vscm{continue} because it should be 
		    on the stack when reaching \vscm{ev-sequence}.
		\item If the predicate is false, we update \vscm{unev} to remove 
		    the first clause and go back to the beginning of the loop.
	    \end{itemize}
	\item If so, we check whether it is the last clause:
	    \begin{itemize}
		\item If not, we raise an error.
		\item If so, we evaluate the actions, as when a true predicate 
		    is found.
	    \end{itemize}
    \end{itemize}
    \scm{ch5/5.24b.scm}
\end{exe}

\begin{exe}[5.25]
    I used some procedures for thunk manipulation from section~\ref{lazy-eval} 
    as primitive procedures: \vscm{delay-it}, \vscm{thunk?}, \vscm{thunk-exp}, 
    \vscm{thunk-env}, \vscm{evaluated-thunk?} and \vscm{thunk-value}.

    The other changes to the evaluator described in that section are implemented 
    in the explicit-control evaluator: I modified \vscm{ev-application}, 
    \vscm{apply-dispatch}, \vscm{ev-if} and the driver loop. For this, the 
    equivalent of \vscm{actual-value} in the explicit-control evaluator is 
    needed.

    The \vscm{actual-value} procedure from section~\ref{lazy-eval} just calls 
    \vscm{force-it} on the result of \vscm{eval}, so in the register machine, it 
    saves the \vscm{continue} register and goes to \vscm{force-it} at the end of 
    \vscm{eval-dispatch}. At the \vscm{force-it} label, the \vscm{continue} 
    register is restored, then the actual value of the contents of the 
    \vscm{val} register is computed, and if \vscm{val} contained a thunk it is 
    turned into an evaluated thunk.
    \begin{cscm}
    actual-value
        (save continue)
        (assign continue (label force-it))
        (goto (label eval-dispatch))
    force-it
        ;; Forces value of reg val.
        ; Saved at actual-value
        (restore continue)
        (test (op thunk?) (reg val))
        (branch (label force-thunk))
        (test (op evaluated-thunk?) (reg val))
        (branch (label force-evaluated-thunk))
        (goto (reg continue))
    force-thunk
        (assign env (op thunk-env) (reg val))
        (assign exp (op thunk-exp) (reg val))
        ; Comment out the following 3 lines to use unmemoized force-it.
        (save val) ; Save thunk to set its value later.
        (save continue)
        (assign continue (label thunk-result-forced))
        (goto (label actual-value))
    thunk-result-forced
        ; val contains the actual value.
        (assign exp (reg val))
        (restore continue)
        (restore val) ; Thunk
        (perform (op set-car!) (reg val) (const evaluated-thunk))
        (assign unev (op cdr) (reg val))
        (perform (op set-car!) (reg unev) (reg exp))
        (perform (op set-cdr!) (reg unev) (const ()))
        (assign val (reg exp))
        (goto (reg continue))
    force-evaluated-thunk
        (assign val (op thunk-value) (reg val))
        (goto (reg continue)))))
    \end{cscm}
    The above instructions implement the memoized version of \vscm{force-it}. To 
    use the non-memoized version, it is enough to comment out the lines that set 
    up the stack and the continuation so that the thunk is turned into an 
    evaluated thunk once its value has been computed: that way the 
    \vscm{thunk-result-forced} label is never reached, so no evaluated thunks 
    are created.

    With these instructions for \vscm{actual-value}, the only change needed at 
    the places where \vscm{actual-value} is used in the lazy evaluator is to 
    replace \vscm{(goto (label eval-dispatch))} with
    \vscm{(goto (label actual-value))}, which is the only change needed in 
    \vscm{read-eval-print-loop} and in \vscm{ev-if}.

    Significant changes to \vscm{ev-application} and \vscm{apply-dispatch} are 
    needed. The instructions at \vscm{ev-application} are simplified since the 
    argument accumulation is now left to \vscm{apply-dispatch}. The actual value 
    of the operator is computed and stored in \vscm{proc}, the operand 
    expressions are put in \vscm{unev}, and the rest is left to 
    \vscm{apply-dispatch}:
    \begin{cscm}
    ev-application
        (save continue)
        (save env)
        (assign unev (op operands) (reg exp))
        (save unev)
        (assign exp (op operator) (reg exp))
        (assign continue (label ev-appl-did-operator))
        (goto (label actual-value))
    ev-appl-did-operator
        (restore unev)          ; the operands
        (restore env)
        (assign proc (reg val)) ; the operator
        (goto (label apply-dispatch))
    \end{cscm}

    The beginning of \vscm{apply-dispatch} does not change much: it dispatches 
    on the procedure type. It also initializes the argument list.
    \begin{cscm}
    apply-dispatch
        ; Actual value of procedure to apply in proc, operand expressions in 
        ; unev, environment in env.
        (assign argl (op empty-arglist))
        (test (op primitive-procedure?) (reg proc))
        (branch (label primitive-apply))
        (test (op compound-procedure?) (reg proc))
        (branch (label compound-apply))
        (goto (label unknown-procedure-type))
    \end{cscm}

    Before applying a primitive procedure, the actual values of the arguments 
    must be computed and accumulated in \vscm{argl}, which was previously done 
    in \vscm{ev-application}. The instructions are almost the same as those used 
    in \vscm{ev-application} in the applicative-order evaluator, the main 
    difference is the use of \vscm{actual-value} rather than 
    \vscm{eval-dispatch} to compute each argument value.
    \begin{cscm}
    primitive-apply
        ; Accumulate actual values of arguments in argl.
        (save proc)
        (test (op no-operands?) (reg unev))
        (branch (label actual-primitive-apply))
    ev-appl-operand-loop
        (save argl)
        (assign exp (op first-operand) (reg unev))
        (test (op last-operand?) (reg unev))
        (branch (label ev-appl-last-arg))
        (save env)
        (save unev)
        (assign continue (label ev-appl-accumulate-arg))
        (goto (label actual-value))
    ev-appl-accumulate-arg
        (restore unev)
        (restore env)
        (restore argl)
        (assign argl (op adjoin-arg) (reg val) (reg argl))
        (assign unev (op rest-operands) (reg unev))
        (goto (label ev-appl-operand-loop))
    ev-appl-last-arg
        (assign continue (label ev-appl-accum-last-arg))
        (goto (label actual-value))
    ev-appl-accum-last-arg
        (restore argl)
        (assign argl (op adjoin-arg) (reg val) (reg argl))
    actual-primitive-apply
        (restore proc)
        (assign val (op apply-primitive-procedure) (reg proc) (reg argl))
        (restore continue)
        (goto (reg continue))
    \end{cscm}

    Before applying a compound procedure, the delayed arguments are accumulated. 
    Though the procedures \vscm{list-of-arg-values} and 
    \vscm{list-of-delayed-args} are very similar in section~\ref{lazy-eval}, 
    their register-machine translations are very different: for the latter no 
    actual computation besides the argument accumulation is done so we can use 
    a simple loop without using the stack at all, unlike the implementation of 
    \vscm{list-of-arg-values} above.
    \begin{cscm}
    compound-apply
        ; Accumulate delayed arguments in argl.
        (test (op no-operands?) (reg unev))
        (branch (label actual-compound-apply))
        (assign exp (op first-operand) (reg unev))
        (assign exp (op delay-it) (reg exp) (reg env))
        (assign argl (op adjoin-arg) (reg exp) (reg argl))
        (assign unev (op rest-operands) (reg unev))
        (goto (label compound-apply))
    actual-compound-apply
        (assign unev (op procedure-parameters) (reg proc))
        (assign env (op procedure-environment) (reg proc))
        (assign env (op extend-environment) (reg unev) (reg argl) (reg env))
        (assign unev (op procedure-body) (reg proc))
        (goto (label ev-sequence))
    \end{cscm}
\end{exe}

\subsection{Running the Evaluator}

\begin{exe}[5.26]
    \ \vspace{-20pt}
    \begin{enumerate}
	\item The maximum depth is 10 for $n \geq 1$. For $n = 0$ the maximum 
	depth is 8.

    \item The number of pushes for any $n \geq 0$ is equal to $35n + 29$.
    \end{enumerate}
\end{exe}

\begin{exe}[5.27]
    The results found are, for any $n \geq 1$:
    \begin{center}
	\begin{tabular}{|c|c|c|}
	    \hline
	    & Maximum depth & Number of pushes \\\hline
	    Recursive factorial & $5n + 3$ & $32n - 16$ \\\hline
	    Iterative factorial & $10$ & $35n + 29$ \\\hline
	\end{tabular}
    \end{center}
\end{exe}

\begin{exe}[5.28]
    With a non tail-recursive evaluator, both procedures now require space and 
    time that grow linearly with their input. The results can be summed up in 
    the following table, for any $n \geq 1$:
    \begin{center}
	\begin{tabular}{|c|c|c|}
	    \hline
	    & Maximum depth & Number of pushes \\\hline
	    Recursive factorial & $8n + 3$ & $34n - 16$ \\\hline
	    Iterative factorial & $3n + 14$ & $37n + 33$ \\\hline
	\end{tabular}
    \end{center}
\end{exe}


\end{document}
