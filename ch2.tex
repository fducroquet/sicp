\chapter{Building Abstractions with Data}

\section{Introduction to Data Abstraction}

\subsection{Example: Arithmetic Operations for Rational Numbers}

\begin{exe}[2.1]
    A possibility for a \vscm{make-rat} handling both positive and negative 
    arguments:
    \scm{ch2/2.01.scm}
\end{exe}

\subsection{Abstraction Barriers}

\begin{exe}[2.2]
    Exemple implementation for the representation of segments in a plane:
    \scm{ch2/2.02.scm}
\end{exe}

\begin{exe}[2.3]
    In the following implementation, a rectangle is represented by its two 
    opposite sides, which must have the same orientation. The code makes use of 
    auxiliary procedures defined below.

    I added selectors to access each of the vertices of the rectangle to be 
    able to print rectangles in a uniform format.
    \scm{ch2/2.03.1.scm}

    The procedures that compute the perimeter and area of a rectangle, and the 
    procedure that prints a rectangle, are defined thus:
    \scm{ch2/2.03.perim-area.scm}

    Another possibility is to represent a rectangle by its four vertices:
    \scm{ch2/2.03.2.scm}

    Yet another possibility is to represent a rectangle by two perpendicular 
    segments with the same origin:
    \scm{ch2/2.03.3.scm}

    The procedures \vscm{perim-rect}, \vscm{area-rect} and \vscm{print-rect} 
    work in all three cases.

    \begin{comp}
        The code above makes use of the following auxiliary procedures to check 
        that the input is correct, compute the length of a segment, and print 
        points inline for use in \vscm{print-rect}:
        \scm{ch2/2.03.aux.scm}
    \end{comp}
\end{exe}

\subsection{What Is Meant by Data?}

\begin{exe}[2.4]
    With the representation of pairs given in the exercise, \vscm{(cons x y)} is 
    a procedure that takes as its argument a procedure \vscm{m} with two 
    arguments and returns the result of the application of \vscm{m} to \vscm{x} 
    and \vscm{y}.

    \vscm{(car z)} applies the procedure \vscm{(cons x y)} to the procedure that 
    returns the first of its arguments, so \vscm{(car (cons x y))} yields 
    \vscm{x}.

    Using the substitution model, the successive steps are:
    \scm{ch2/2.04a.scm}

    The corresponding definition of \vscm{cdr} is:
    \scm{ch2/2.04b.scm}
    The method to prove that \vscm{(cdr (cons x y))} yields \vscm{y} is the same 
    as with \vscm{car}.
\end{exe}

\begin{exe}[2.5]
    If $a$ and $b$ are known, we can compute $2^a 3^b$, and since the 
    decomposition of integers as a product of primes is unique, it’s possible to 
    find $a$ and $b$ from the value of $2^a 3^b$.

    The procedures \vscm{cons}, \vscm{car}, and \vscm{cdr} corresponding to this 
    representation can be defined as:
    \scm{ch2/2.05.scm}
\end{exe}

\begin{exe}[2.6]
    The successive substitution steps to evaluate \vscm{(add-1 zero)} are:
    \scm{ch2/2.06a.scm}

    In other words, \vscm{one} is a procedure that takes a one-argument 
    procedure as its argument and returns it.

    The substitution steps to evaluate \vscm{(add-1 one)} are:
    \scm{ch2/2.06b.scm}

    In other words, \vscm{two} is a procedure that takes a one-argument 
    procedure $f$ as its argument and returns the procedure $f \circ f$ ($f$ 
    applied twice).

    From these observations, and after remarking that \vscm{zero} is a procedure 
    that takes one argument and always returns the identity procedure, we can 
    make the hypothesis that the $n$th Church numeral is a procedure that takes 
    a one-argument procedure $f$ as its argument and returns the $n$th repeated 
    application of $f$ (see exercise~\ref{1.43}). This can be proved by 
    induction.

    \begin{proof}
        We’ve already shown that it’s true for $0$, $1$ and $2$.
        Let’s assume that it’s true for a positive integer $n$.

        From the induction hypothesis, \vscm{(n f)} is the $n$th repeated 
        application of $f$, so it’s obvious that
        \vscm{(lambda (x) (f ((n f) x)))} is the $(n + 1)$th repeatead 
        application of $f$, so the result is true for $n + 1$, hence it’s true 
        for any positive integer $n$.
    \end{proof}

    To apply a function $n + m$ times, we just need to apply it $m$ times, and 
    then $n$ times more, so \vscm{+} can be defined directly as:
    \scm{ch2/2.06c.scm}
\end{exe}

\subsection{Extended Exercise: Interval Arithmetic}

Let’s first define a function that prints intervals:
\scm{ch2/2.07pre2.scm}

\begin{exe}[2.7]
    Since \vscm{make-interval} has been defined as \vscm{cons}, 
    \vscm{upper-bound} and \vscm{lower-bound} can be defined as \vscm{cdr} and 
    \vscm{car} respectively.
    \scm{ch2/2.07.scm}
\end{exe}

\begin{exe}[2.8]
    With the same reasoning as for division, the subtraction of two intervals is 
    the addition of the first with the opposite of the second.
    The subtraction procedure can thus be defined:
    \scm{ch2/2.08.scm}
\end{exe}

\begin{exe}[2.9]
    Let $[a;b]$ and $[c;d]$ be two intervals.

    Their sum is $[a + c; b + d]$. Its width is $\sfrac{(b + d) - (a + c)}{2} 
    = \sfrac{(b - a)}{2} + \sfrac{(d - c)}{2}$, in other words, the sum’s 
    width is the sum of the widths, so it depends only on the widths of the 
    intervals being added.

    The difference can be defined as the sum with the opposite, and taking the 
    opposite doesn’t change the width, so this is also true for differences.

    For multiplication and division, the width of the result also depends on 
    the values of the bounds. For instance, $[1; 2] \times [2; 3] = [2; 6]$, 
    but $[0; 1] \times [2; 3] = [0; 3]$. In both cases, we multiply two 
    intervals of width $\sfrac{1}{2}$, but the former product has a width of 
    $2$ while the latter has a width of $\sfrac{3}{2}$, so the width of the 
    product is not a function of the widths of the intervals being multiplied.

    Since division can be defined as a multiplication, this is also true for 
    division.
\end{exe}

\begin{exe}[2.10]
    The new code of \vscm{div-interval} could be:
    \scm{ch2/2.10.scm}
\end{exe}

\begin{exe}[2.11]
    There are three cases for each interval:
    \begin{itemize}
        \item the lower bound is positive or null;
        \item the upper bound is negative or null;
        \item the lower bound is negative and the upper bound is positive.
    \end{itemize}
    This results in a total of nine cases, and the only case where the smallest 
    and greatest products can’t be deduced from the signs is when both intervals 
    span zero.

    A procedure taking this suggestion into account is:
    \scm{ch2/2.11.scm}
\end{exe}

\begin{exe}[2.12]
    The procedures \vscm{make-center-percent} and \vscm{percent} can be defined 
    as:
    \scm{ch2/2.12.scm}
\end{exe}

\begin{exe}[2.13]
    Let $c_1, c_2, w_1$ and $w_2$ be the centers and widths of two intervals. We 
    assume that all numbers are positive. The lower and upper bounds of the 
    product are
    $(c_1 \pm w_1) \times (c_2 \pm w_2) = c_1 c_2 \pm (c_1 w_2 + c_2 w_1) + w_1 
    w_2$.

    Since the percentages are small, $w_1 w_2$ is negligible, and the product’s 
    width is $w \approx c_1 w_2 + c_2 w_1$.

    Additionally, if we call the percentage tolerances $p_1$ and $p_2$ 
    respectively, we have $w_i = c_i \times \sfrac{p_i}{100}$ for $i = 1, 2$.

    From there, $w \approx c_1 c_2 \times \sfrac{p_1 + p_2}{100}$, and $c_1 c_2$ 
    is the product’s center, so under the given conditions, the approximate 
    percentage tolerance of the product is the sum of the tolerances of the 
    factors.
\end{exe}

\begin{exe}[2.14]
    TODO
\end{exe}

\begin{exe}[2.15]
    TODO
\end{exe}

\begin{exe}[2.16]
    TODO
\end{exe}

\section{Hierarchical Data and the Closure Property}

\subsection{Representing Sequences}

\begin{exe}[2.17]
    The \vscm{last-pair} procedure can be defined as:
    \scm{ch2/2.17.scm}
\end{exe}

\begin{exe}[2.18]
    The \vscm{reverse} procedure can be defined as:
    \scm{ch2/2.18.scm}
\end{exe}

\begin{exe}[2.19]
    The procedures can be defined respectively as \vscm{car}, \vscm{cdr} and 
    \vscm{null?}.
    \scm{ch2/2.19.scm}

    The order of the list \vscm{coin-values} does not affect the answer produced 
    by \vscm{cc}, because \vscm{cc} gives the total number of combinations, and 
    the relation used for the computation does not depend on a particular order.
\end{exe}

\begin{exe}[2.20]
    A possible solution is:
    \scm{ch2/2.20.scm}
\end{exe}

\begin{exe}[2.21]
    The completed procedures are:
    \scm{ch2/2.21.scm}
\end{exe}

\begin{exe}[2.22]
    With the first procedure, the answer list is in reverse order because the 
    elements are added to it starting from the beginning of the initial list, 
    and the first element added to a list is at its end.

    With the second procedure, the result is not a list because \vscm{cons} is 
    called with a list as its first argument and the element to add as its 
    second argument. To add an element to a list, the order of the arguments 
    should be the opposite.
\end{exe}

\begin{exe}[2.23]
    Here is a possible implementation of \vscm{for-each}:
    \scm{ch2/2.23.scm}
\end{exe}

\subsection{Hierarchical Structures}

\begin{exe}[2.24]
    The result given by the interpreter is \vscm{(1 (2 (3 4)))}. To represent 
    the corresponding box-and-pointer structure in terms of pairs, one must use 
    the equality of \vscm{(list 1 (list 2 (list 3 4)))} and
    \vscm{(cons 1 (cons (list 2 (list 3 4)) nil))}, and similar equalities for 
    the two other lists.

    \begin{figure}
        \begin{center}
            TODO (too time-consuming…)
        \end{center}
        \caption{Box-and-pointer-structure of \vscm{(1 (2 (3 4)))}.}
    \end{figure}

    \vspace*{1cm}
    \begin{figure}
        \begin{center}
            \pstree[nodesepB=5pt]{\Tdot[tnpos=a]~{\vscm{(1 (2 (3 4)))}}}{%
            \TR[tnpos=b]{1}%
            \pstree{\Tdot[tnpos=r]~{\vscm{(2 (3 4))}}}{%
            \TR[tnpos=b]{2}%
            \pstree{\Tdot[tnpos=r]~{\vscm{(3 4)}}}{%
            \TR[tnpos=b]{3}\TR[tnpos=b]{4}}}}
        \end{center}
        \caption{Tree representation of \vscm{(1 (2 (3 4)))}.}
    \end{figure}
\end{exe}

\begin{exe}[2.25]
    If we call the three lists $x$, $y$ and $z$ respectively, the combinations 
    \vscm{(car (cdr (car (cdr (cdr x)))))}, \vscm{(car (car y))} (which can be 
    shortened to \vscm{(caar y)}), and\linebreak
    \vscm{(car (cdr (car (cdr (car (cdr (car (cdr (car (cdr (car (cdr z))))))))))))}
    or, more simply,
    \vscm{(cadr (cadr (cadr (cadr (cadr (cadr z))))))},
    all pick 7 from the lists.
\end{exe}

\begin{exe}[2.26]
    The results printed by the interpreter are \vscm{(1 2 3 4 5 6)} for
    \vscm{(append x y)}, \vscm{((1 2 3) 4 5 6)} for \vscm{(cons x y)} and
    \vscm{((1 2 3) (4 5 6))} for \vscm{(list x y)}.
\end{exe}

\begin{exe}[2.27]
    Here is a possible solution for \vscm{deep-reverse}:
    \scm{ch2/2.27.scm}
\end{exe}

\begin{exe}[2.28]
    A possible solution for \vscm{fringe} is:
    \scm{ch2/2.28.scm}
\end{exe}

\begin{exe}[2.29]
    \ \vspace{-20pt}
    \begin{itemize}
        \item[a.] The selectors can be defined as:
            \scm{ch2/2.29a.scm}
        \item[b.] The total weight can be computed with:
            \scm{ch2/2.29b.scm}
        \item[c.] Since we defined a \vscm{branch-weight} procedure in b., 
            \vscm{balanced?} can be defined simply with:
            \scm{ch2/2.29c.scm}
        \item[d.] To convert to the new representation, the only things that 
            need to be changed are the \vscm{right-branch} and 
            \vscm{branch-structure} selectors:
            \scm{ch2/2.29d.scm}
    \end{itemize}
\end{exe}

\begin{exe}[2.30]
    The two \vscm{square-list} procedures are identical to the \vscm{scale-tree} 
    procedures defined in the text, except that there is only one argument and 
    \vscm{(* tree factor)} is replaced with \vscm{(square tree)}.

    Direct definiton:
    \scm{ch2/2.30a.scm}

    Using \vscm{map} and recursion:
    \scm{ch2/2.30b.scm}
\end{exe}

\begin{exe}[2.31]
    The procedure \vscm{tree-map} can be defined without using \vscm{map}:
    \scm{ch2/2.31a.scm}
    or using it:
    \scm{ch2/2.31b.scm}
\end{exe}

\begin{exe}[2.32]
    The completed procedure is:
    \scm{ch2/2.32.scm}

    The empty set has only one subset: the empty set.

    For a non-empty (finite) set, with elements $\{a_1, …, a_n\}$, the set of 
    subsets is the reunion of the subsets not containing $a_1$ and the subsets 
    containing $a_1$, and the application $S \mapsto S \cup \{a_1\}$ is 
    a bijection between these two sets.
\end{exe}

\subsection{Sequences as Conventional Interfaces}

\begin{exe}[2.33]
    The operations can be redefined as:
    \scm{ch2/2.33.scm}
\end{exe}

\begin{exe}[2.34]
    A polynomial can be evaluated using Horner’s rule with the procedure:
    \scm{ch2/2.34.scm}
\end{exe}

\begin{exe}[2.35]
    This can be done with or without \vscm{enumate-tree}. Without that function:
    \scm{ch2/2.35.scm}
    The mapped function associates to each subtree its number of leaves: 1 if 
    the subtree has no children, i.e.\ is a leaf, \vscm{(count-leaves subtree)} 
    otherwise.
\end{exe}

\begin{exe}[2.36]
    The procedure \vscm{accumulate-n} can be defined as:
    \scm{ch2/2.36.scm}
\end{exe}

\begin{exe}[2.37]
    The matrix operation can be define as:
    \scm{ch2/2.37.scm}
\end{exe}

\begin{exe}[2.38]
    The value of \vscm{(fold-right / 1 (list 1 2 3))} is $\sfrac{3}{2}$.

    The value of \vscm{(fold-left / 1 (list 1 2 3))} is $\sfrac{1}{6}$.

    The value of \vscm{(fold-right list nil (list 1 2 3))} is the list
    \vscm{(1 (2 (3 nil)))}.

    The value of \vscm{(fold-left list nil (list 1 2 p))} is the list
    \vscm{(((nil 1) 2) 3)}.

    \vscm{fold-right} and \vscm{fold-left} produce the same values for any 
    sequence  if (and only if) \vscm{op} is commutative.

    \begin{proof}
        Suppose \vscm{op} commutative. We’ll show by induction that 
        \vscm{fold-left} and \vscm{fold-right} always produce the same results.

        \vscm{(fold-right op init nil)} and
        \vscm{(fold-left op init nil)} both produce \vscm{init}.

        Let’s assume that \vscm{fold-left} and \vscm{fold-right} produce the 
        same values for any list of length $n \geq 0$. If \vscm{sequence} is 
        a list of length $n + 1$,
        \vscm{(fold-right op init sequence)} equals\linebreak
        \vscm{(op (car sequence) (fold-right op init (cdr sequence)))}, and
        \vscm{(fold-left op init sequence)} equals
        \vscm{(op (fold-left op init (cdr sequence)) (car sequence))}. It 
        follows from the induction hypothesis and the commutativity of \vscm{op} 
        that these two values are equal.

        Hence, by induction, \vscm{fold-left} and \vscm{fold-right} always 
        produce the same results.

        Conversely, if \vscm{op} is not commutative, there exists elements 
        \vscm{a} and \vscm{b} such that \vscm{(op a b)} is different from
        \vscm{(op b a)}, and these expressions are equal respectively to
        \vscm{(fold-left op a (list b))} and to
        \vscm{(fold-right op a (list b))}, so \vscm{fold-left} and 
        \vscm{fold-right} don’t always produce the same values.
    \end{proof}
\end{exe}

\begin{exe}[2.39]
    The procedure \vscm{reverse} can be defined in terms of \vscm{fold-left} and 
    \vscm{fold-right} as:
    \scm{ch2/2.39.scm}
\end{exe}

\begin{exe}[2.40]
    The procedure \vscm{unique-pairs} and the simplified definition of 
    \vscm{prime-sum-pairs} are:
    \scm{ch2/2.40.scm}
\end{exe}

\begin{exe}[2.41]
    A solution using \vscm{unique-pairs} from the previous exercise:
    \scm{ch2/2.41.scm}
    The triples $(i, j, k)$ are in decreasing order because \vscm{unique-pairs} 
    returns pairs $(j, k)$ with $j > k$.
\end{exe}

\begin{exe}[2.42]
    A possible solution, with a position represented as the list of the numbers 
    of the lines occupied by the queen in each column. The position of the queen 
    in the first column is at the end of the list because the functions are 
    easier to write this way. I removed the \vscm{k} parameter in \vscm{safe?} 
    and \vscm{adjoin-position} because I didn’t need it.
    \scm{ch2/2.42.scm}
\end{exe}

\begin{exe}[2.43]
    In Louis’ program, each call to \vscm{(queen-cols k)} calls 
    \vscm{(queen-cols (- k 1))} 8 times, instead of only one time with the 
    program in exercise 2.42. Since there are 8 recursion levels, Louis’ program 
    will take about $8^8T$ time to solve the eight-queens puzzle.
\end{exe}

\subsection{Example: A Picture Language}

\begin{comp}
    To test the code in this section, it’s necessary to load a module including 
    graphical functions. The solution I chose here is to use DrRacket’s 
    \vscm{graphics.ss} library. It required me to adapt my solution to 
    exercise~2.46 and to slightly modify the \vscm{segments->painter} to use the 
    given viewport. The code from later exercises is sometimes needed to try out 
    some previous code.

    Code to put at the beginning of the file:
    \scm{ch2/2.44pre2.scm}
\end{comp}

\begin{exe}[2.44]
    The definition of \vscm{up-split} is:
    \scm{ch2/2.44.scm}
\end{exe}

\begin{exe}[2.45]
    The \vscm{split} procedure can be defined as:
    \scm{ch2/2.45.scm}
\end{exe}

\begin{exe}[2.46]
    A possible solution if we are not using \vscm{graphics.ss}:
    \scm{ch2/2.46a.scm}
    With \vscm{graphics.ss}, we must redefine \vscm{make-vect}, \vscm{xcor-vect} 
    and \vscm{ycor-vect}:
    \scm{ch2/2.46b.scm}
\end{exe}

\begin{exe}[2.47]
    The selectors \vscm{origin-frame} and \vscm{edge1-frame} are the same with 
    both implementations:
    \scm{ch2/2.47a.scm}
    For the first implementation:
    \scm{ch2/2.47b.scm}
    For the second implementation:
    \scm{ch2/2.47c.scm}
\end{exe}

\begin{comp}
    Sligthly modified version of \vscm{segments->painter} for use with 
    \vscm{graphics.ss}:
    \scm{ch2/2.48pre2.scm}
\end{comp}

\begin{exe}[2.48]
    Representation of segments:
    \scm{ch2/2.48.scm}
\end{exe}

\begin{exe}[2.49]
    \ \vspace{-20pt}
    \begin{itemize}
        \item[a.] The painter drawing the outline can be defined as: 
            \scm{ch2/2.49a.scm}
        \item[b.] The painter drawing an “X” can be defined as: 
            \scm{ch2/2.49b.scm}
        \item[c.] The painter drawing a diamond can be defined as: 
            \scm{ch2/2.49c.scm}
        \item[d.] The \vscm{wave} painter can be defined as: \scm{ch2/2.49d.scm}
    \end{itemize}
\end{exe}

\begin{exe}[2.50]
    The given transformations can be defined as:
    \scm{ch2/2.50.scm}
\end{exe}

\begin{exe}[2.51]
    \vscm{below} defined as a procedure analogous to the \vscm{beside} 
    procedure:
    \scm{ch2/2.51a.scm}
    In terms of \vscm{beside} and rotation operations:
    \scm{ch2/2.51b.scm}
\end{exe}

\begin{exe}[2.52]
    \ \vspace{-20pt}
    \begin{itemize}
        \item[a.] Possible modified version of \vscm{wave}:
            \scm{ch2/2.52a.scm}
        \item[b.] Modified \vscm{corner-split}:
            \scm{ch2/2.52b.scm}
        \item[c.] Modified \vscm{square-limit}:
            \scm{ch2/2.52c.scm}
    \end{itemize}
\end{exe}

\section{Symbolic Data}

\subsection{Quotation}

\begin{exe}[2.53]
    Check with an interpreter.
\end{exe}

\begin{exe}[2.54]
    A possible implementation of \vscm{equal?}:
    \scm{ch2/2.54.scm}
\end{exe}

\begin{exe}[2.55]
    The expression \vscm{''abracadabra} is equivalent to
    \vscm{(quote (quote abracadabra))}, which evaluates to
    \vscm{(quote abracadabra)}, an expression whose \vscm{car} is \vscm{quote}.
\end{exe}

\subsection{Example: Symbolic Differentiation}

\begin{exe}[2.56]
    The modified version of \vscm{deriv} and the procedures 
    \vscm{exponentiation?}, \vscm{base}, \vscm{exponent}, and 
    \vscm{make-exponentiation} can be written as:
    \scm{ch2/2.56.scm}
\end{exe}

\begin{exe}[2.57]
    The \vscm{deriv} procedure always calls \vscm{make-sum} and 
    \vscm{make-product} with only two arguments, so all that’s necessary is to 
    redefine \vscm{augend} and \vscm{multiplicand} as indicated in the text, for 
    instance:
    \scm{ch2/2.57a.scm}

    As a supplement, if we also want to be able to call \vscm{make-sum} and 
    \vscm{make-product} with more than two arguments, we can redefine 
    \vscm{make-sum} and \vscm{make-product} as well:
    \scm{ch2/2.57b.scm}
    These procedures also simplify the result very partially by grouping the 
    numerical arguments together, however expressions such as
    \vscm{(make-sum 'x (make-sum 'y 'x))} or
    \vscm{(make-sum 'x 'y 2 'x)} are not simplified.
\end{exe}

\begin{exe}[2.58]
    \ \vspace{-20pt}
    \begin{itemize}
        \item[a.] It is sufficient to redefine the following procedures:
            \scm{ch2/2.58a.scm}
        \item[b.] The following implementations supports only multiplication and 
            addition and can produce results with unnecessary parentheses.

            It uses several helper procedures:
            \begin{itemize}
                \item \vscm{(elt-or-list elts)} returns the car of elts if elts 
                    is of length 1, and the list elts otherwise.
                \item \vscm{(take-until l elt)} returns a list containing all 
                    the elements of l until the first occurrence of \vscm{elt}, 
                    excluding it. If elt is not contained in the list, the full 
                    list is returned.
                \item \vscm{(intersperse l sep)} returns a list containing all 
                    the elements of l, with sep inserted between each pair of 
                    elements.
            \end{itemize}
            \scm{ch2/2.58b.scm}
    \end{itemize}
\end{exe}

\subsection{Example: Representing sets}

\begin{exe}[2.59]
    The \vscm{union-set} operation can be defined as:
    \scm{ch2/2.59.scm}
\end{exe}

\begin{exe}[2.60]
    If duplicate elements are allowed, we can redefine \vscm{adjoin-set} and 
    \vscm{union-set} in a more efficient way:
    \scm{ch2/2.60.scm}
    The operation \vscm{adjoin-set} now has $\Theta(1)$ complexity, while 
    \vscm{union-set} has $\Theta(n)$ complexity. It’s not possible to improve 
    the performance of \vscm{element-of-set?} and \vscm{intersection-set}.

    This representation would be preferable to the non-duplicate one when there 
    is no need to worry about the size taken by the sets in memory, and when the 
    operations \vscm{adjoin-set} and \vscm{union-set} are used a lot more than 
    \vscm{element-of-set?} and \vscm{intersection-set}.
\end{exe}

\begin{exe}[2.61]
    A possible implementation of \vscm{adjoin-set} requiring on average half as 
    many steps as with the unordered representation is:
    \scm{ch2/2.61.scm}
\end{exe}

\begin{exe}[2.62]
    By using the same method as for \vscm{intersection-set}, we get 
    a $\Theta(n)$ \vscm{union-set} implementation:
    \scm{ch2/2.62.scm}
\end{exe}

\begin{exe}[2.63]
    \ \vspace{-20pt}
    \begin{itemize}
        \item[a.] Both procedures produce the same result for every tree. They 
            produce the ordered list representation of the set represented by 
            the tree. For the trees in figure 2.16, they produce the list 
            \vscm{(1 3 5 7 9 11)}.
        \item[b.] If $T(n)$ is the number of steps required to convert 
            a balanced tree to a list, with the first procedure we have: $T(n) 
            \approx 2T\left(\sfrac{n}{2}\right) + \sfrac{n}{2}$ since 
            \vscm{append} has linear complexity. By applying this formula 
            recursively, we can see that $T(n) \approx n +  \sfrac{n \log 
            n}{2}$, so that \vscm{tree->list-1} grows as $\Theta(n \log{n})$.

            With the second procedure, we have $T(n) = 2T(\sfrac{n}{2}) + 1$, so 
            \vscm{tree->list-2} grows as $\Theta(n)$.
    \end{itemize}
\end{exe}

\begin{exe}[2.64]
    \ \vspace{-20pt}
    \begin{itemize}
        \item[a.] If $n = 0$, the constructed tree is empty.

            Otherwise, one element will be the tree’s entry, so $n - 1$ elements 
            will be in the subtrees.
            $l = \lfloor \sfrac{n - 1}{2}\rfloor$ elements are put in the left 
            tree, and the remaining $r = n - 1 - l$ are put in the right tree.

            Since the list is ordered and the elements in the left tree must be 
            smaller than the entry, the first elements of the list are used to 
            build the left tree. The first remaining element is the entry, and 
            the $r$ following remaining elements are used to build the right 
            tree. The remaining elements of this last operation are also the 
            remaining elements for the current call to \vscm{partial-tree}, so 
            all that remains to be done is to put all the results together.

        \item[b.] For \vscm{partial-tree}, the number of steps $T(n)$ as 
            a function of the size of the tree to build $n$ verifies
            $T(n) \approx 2 T\left(\sfrac{n}{2}\right)$, so its growth is in 
            $\Theta(n)$, so \vscm{list->tree} has linear growth.
    \end{itemize}
\end{exe}

\begin{exe}[2.65]
    If we call \vscm{union-set-ordered-list} and 
    \vscm{intersection-set-ordered-list} the linear union and intersection 
    procedures defined for ordered lists, we can define linear procedures 
    \vscm{union-set} and \vscm{intersection-set} for binary trees as:
    \scm{ch2/2.65.scm}
\end{exe}

\begin{exe}[2.66]
    The procedure is almost the same as \vscm{element-of-set?} for sets 
    represented as binary trees:
    \scm{ch2/2.66.scm}
\end{exe}

\subsection{Example: Huffman Encoding Trees}

\begin{exe}[2.67]
    The encoded string is ADABBCA.
\end{exe}

\begin{exe}[2.68]
    A possible implementation of \vscm{encode-symbol} is:
    \scm{ch2/2.68.scm}
\end{exe}

\begin{exe}[2.69]
    A possible implementation for \vscm{successive-merge} is:
    \scm{ch2/2.69.scm}
\end{exe}

\begin{exe}[2.70]
    With a Huffman encoding tree, 84 bits are required for encoding.
    With a fixed-length code, at least 3 bits per symbol are required for an 
    alphabet of 8 symbols, and the message contains 36 symbols, so at least 108 
    bits would have been needed.
\end{exe}

\begin{exe}[2.71]
    Since $\sum_{i = 0}^{k} 2^i = 2^{k + 1} - 1$, every right (or every left) 
    branch of the tree is a leaf, and the tree has a depth of $n - 1$. So the 
    most frequent symbol is encoded with 1 bit, while the least frequent symbol 
    is encoded with $n - 1$ bits.
\end{exe}
