\chapter{Modularity, Objects, and State}

\section{Assignment and Local State}

\subsection{Local State Variables}

\begin{exe}[3.1]
    The procedure \vscm{make-accumulator} can be written:
    \scm{ch3/3.01.scm}
\end{exe}

\begin{exe}[3.2]
    The \vscm{make-monitored} procedure can be written:
    \scm{ch3/3.02.scm}
\end{exe}

\begin{exe}[3.3]
    The \vscm{make-account} procedure can be modified in the following way:
    \scm{ch3/3.03.scm}
\end{exe}

\begin{exe}[3.4]
    The procedure can be rewritten as:
    \scm{ch3/3.04.scm}
\end{exe}

\subsection{The Benefits of Introducing Assignment}

\begin{exe}[3.5]
    Using Gambit Scheme’s \vscm{random-real} procedure, that generates a random 
    real number between 0 and 1, \vscm{random-in-range} and the other procedures 
    can be written:
    \scm{ch3/3.05.scm}
\end{exe}

\begin{exe}[3.6]
    The \vscm{rand} procedure can be rewritten as:
    \scm{ch3/3.06.scm}
\end{exe}

\subsection{The Costs of Introducing Assignment}

\begin{exe}[3.7]
    I simply added a \vscm{join} action to the account returned by 
    \vscm{make-account} that creates an access with another password. I also 
    make \vscm{incorrect-password} throw an error instead of simply returning 
    a string, otherwise a call such as
    \vscm{(define new-acc (make-join account curr-pass new-pass))} with an 
    incorrect current password will affect a string value to \vscm{new-acc} 
    without reporting an error, and subsequent uses of the account will throw 
    errors because \vscm{"Incorrect password"} is not a procedure.
    \scm{ch3/3.07.scm}
\end{exe}

\begin{exe}[3.8]
    The procedure \vscm{f} returns:
    \begin{itemize}
        \item 0 if it is the first time it is called;
        \item the previous argument it was called with otherwise.
    \end{itemize}
    Thus, if we evaluate \vscm{(f 0)}, then \vscm{(f 1)}, we get 0 both times, 
    but if we evaluate \vscm{(f 1)}, then \vscm{(f 0)}, we get 0 the first 
    time and 1 the second.
    \scm{ch3/3.08.scm}
\end{exe}

\section{The Environment Model of Evaluation}

\subsection{The Rules for Evaluation}

This subsection contains no exercises.

\subsection{Applying Simple Procedures}

% Tikz styles for the environments of section 3.2.
\tikzset{env/.style={
rectangle,
rounded corners=2pt,
% border
very thick,
draw=teal!80!black,
% filling
fill=teal!10,
inner sep = 2mm,
font=\ttfamily\small,
align=left,
},
global env/.style={
env,
align=left,
inner xsep=1ex,
text width=13cm-2*\pgfkeysvalueof{/pgf/inner xsep},
minimum width=13cm,
minimum height=1.5cm
},
code/.style={
align=left,
font=\ttfamily\small,
}
}

\var{\nametoenv}{4mm}
\var{\envtonext}{3mm}
\begin{exe}[3.9]
    \begin{figure}
        \begin{tikzpicture}[>=Stealth, thick]
            \node[text width=1cm, align=right] (ge) {global\\ env};
            \node[global env, right=\nametoenv of ge] (g) {factorial};
            \node[text width=1cm, align=right, below=of ge] (e1) {E1};
            \node[env, right=\nametoenv of e1] (n1) {n:\,6};
            \node[right=\envtonext of n1] (e2) {E2};
            \node[env, right=\nametoenv of e2] (n2) {n:\,5};
            \node[right=\envtonext of n2] (e3) {E3};
            \node[env, right=\nametoenv of e3] (n3) {n:\,4};
            \node[right=\envtonext of n3] (e4) {E4};
            \node[env, right=\nametoenv of e4] (n4) {n:\,3};
            \node[right=\envtonext of n4] (e5) {E5};
            \node[env, right=\nametoenv of e5] (n5) {n:\,2};
            \node[right=\envtonext of n5] (e6) {E6};
            \node[env, right=\nametoenv of e6] (n6) {n:\,1};

            \draw[->] (ge) -- (ge.west -| g.west);
            \foreach \i in {1, ..., 6} {
                \draw[->] (e\i) -- (n\i);
                \draw[->] (n\i.north) -- (n\i.north|-g.south);
                \node[code,below=2mm of n\i] { \vscm{(if ...)} };
            }
        \end{tikzpicture}
        \caption{Environments created by evaluating \vscm{(factorial 6)} with 
        the recursive procedure. In all the environments created, the code to 
        evaluate corresponds to the body of the \vscm{factorial} procedure.}
        \label{fact_rec}
    \end{figure}

    \begin{figure}
        \begin{tikzpicture}[>=Stealth, thick]
            \node[text width=1cm, align=right] (ge) {global\\ env};
            \node[global env, right=\nametoenv of ge] (g)
            {factorial\\ fact-iter};
            \node[text width=1cm, align=right, below=of ge] (e1) {E1};
            \node[env, right=\nametoenv of e1] (n1) {n:\,6};
            \node[right=\envtonext of n1] (e2) {E2};
            \node[env, right=\nametoenv of e2] (n2) {p:\,1\\ c:\,1\\ m:\,6};
            \node[right=\envtonext of n2] (e3) {E3};
            \node[env, right=\nametoenv of e3] (n3) {p:\,2\\ c:\,2\\ m:\,6};
            \node[right=(\envtonext+.6cm) of n3] (e4) {E4};
            \node[env, right=\nametoenv of e4] (n4) {p:\,6\\ c:\,3\\ m:\,6};
            \node[right=(\envtonext+.3cm) of n4] (e5) {E5};
            \node[env, right=\nametoenv of e5] (n5) {p:\,24\\ c:\,4\\ m:\,6};
            \node[below=1.9cm of n3] (e6) {E6};
            \node[env, right=\nametoenv of e6] (n6) {p:\,24\\ c:\,5\\ m:\,6};
            \node[right=\envtonext of n6] (e7) {E7};
            \node[env, right=\nametoenv of e7] (n7) {p:\,120\\ c:\,6\\ m:\,6};
            \node[right=\envtonext of n7] (e8) {E8};
            \node[env, right=\nametoenv of e8] (n8) {p:\,720\\ c:\,7\\ m:\,6};

            \draw[->] (ge) -- (ge.west -| g.west);
            \foreach \i in {1, ..., 8} {
                \draw[->] (e\i) -- (n\i);
                \draw[->] (n\i.north) -- (n\i.north|-g.south);
            }

            \node[code, below = 2mm of n1] { \vscm{(fact-iter 1 1 n)} };
            \foreach \i in {2, ..., 8} {
                \node[code, below = 2mm of n\i] { \vscm{(if ...)} };
            }
        \end{tikzpicture}
        \caption{Environments created by evaluating \vscm{(factorial 6)} with 
        the iterative procedure. In environments E2 to E8, the code to evaluate 
        corresponds to the body of the \vscm{fact-iter} procedure.}
        \label{fact_iter}
    \end{figure}

    The environment structure created by evaluating \vscm{(factorial 6)} with 
    both versions of the procedure are shown in figures \ref{fact_rec} and 
    \ref{fact_iter}.
\end{exe}
