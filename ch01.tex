\chapter{Building Abstractions with Procedures}

\section{The Elements of Programming}

\begin{exe}
    Check with a scheme interpreter.
\end{exe}

\begin{exe}
    A minimally indented version:
    \scm{ch01/1.02a.scm}
    A heavily indented version:
    \scm{ch01/1.02b.scm}
\end{exe}

\begin{exe}
    \ \vspace{-20pt}
    \scm{ch01/1.03.scm}
\end{exe}

\begin{exe}
  If $b > 0$, \vscm{(a-plus-abs-b a b)} returns $a + b$; otherwise it returns $a 
- b$.
In other words, \vscm{(a-plus-abs-b a b)} returns $a + |b|$.
\end{exe}

\begin{exe}
    With applicative-order evaluation, the interpreter tries to evaluate 
    \vscm{(p)}, which results in an infinite loop, so the interpreter never 
    returns (or returns an error).

    With normal-order evaluation, the interpreter doesn't try to evaluate 
    \vscm{(p)} until it's really needed, but that never happens since
    \vscm{(= x 0)} returns true, so the call returns 0.
\end{exe}


% vim:filetype=tex:set expandtab
