\chapter{Building Abstractions with Procedures}

\section{The Elements of Programming}

\begin{exe} % 1.1
    Check with a scheme interpreter.
\end{exe}

\begin{exe} % 1.2
    A minimally indented version:
    \scm{ch01/1.02a.scm}
    A heavily indented version:
    \scm{ch01/1.02b.scm}
\end{exe}

\begin{exe} % 1.3
    \ \vspace{-20pt}
    \scm{ch01/1.03.scm}
\end{exe}

\begin{exe} % 1.4
  If $b > 0$, \vscm{(a-plus-abs-b a b)} returns $a + b$; otherwise it returns $a 
- b$.
In other words, \vscm{(a-plus-abs-b a b)} returns $a + |b|$.
\end{exe}

\begin{exe} % 1.5
    With applicative-order evaluation, the interpreter tries to evaluate 
    \vscm{(p)}, which results in an infinite loop, so the interpreter never 
    returns (or returns an error).

    With normal-order evaluation, the interpreter doesn't try to evaluate 
    \vscm{(p)} until it's really needed, but that never happens since
    \vscm{(= x 0)} returns true, so the call returns 0.
\end{exe}

\begin{exe} % 1.6
    When Alyssa attempts to use this to compute square roots, the program never 
    returns.

    Explication: \vscm{new-if} is an ordinary procedure, so each time it is 
    called, the evaluator tries to evaluate all of its arguments. In particular, 
    each call to \vscm{sqrt-iter} will cause one more call to \vscm{sqrt-iter}, 
    whether \vscm{(good-enough? guess x)} returns true or not, so the evaluator 
    ends up in an infinite loop.
\end{exe}

% vim:filetype=tex:set expandtab
